\documentclass{ieeeaccess}
\usepackage{cite}
\usepackage{amsmath,amssymb,amsfonts}
\usepackage{algorithm,algorithmic}
\usepackage{cases}
\usepackage{graphicx}
\usepackage{textcomp}

\def\BibTeX{{\rm B\kern-.05em{\sc i\kern-.025em b}\kern-.08em
    T\kern-.1667em\lower.7ex\hbox{E}\kern-.125emX}}
\begin{document}
\history{Date of publication xxxx 00, 0000, date of current version xxxx 00, 0000.}
\doi{10.1109/ACCESS.2017.DOI}

\title{Preparation of Papers for IEEE ACCESS}

\author{\uppercase{Te Wei}\authorrefmark{1}, \IEEEmembership{Student Member, IEEE},
\uppercase{Yunzhong Hou}\authorrefmark{2}, \uppercase{Ning Ge}\authorrefmark{3},
\IEEEmembership{Member, IEEE},\uppercase{Wei Feng}\authorrefmark{4},
\IEEEmembership{Member, IEEE}, \uppercase{and Jianhua Lu},\authorrefmark{5},
\IEEEmembership{Member, IEEE}}
\address[1]{National Institute of Standards and 
Technology, Boulder, CO 80305 USA (e-mail: author@boulder.nist.gov)}
\address[2]{Department of Physics, Colorado State University, Fort Collins, 
CO 80523 USA (e-mail: author@lamar.colostate.edu)}
\address[3]{Electrical Engineering Department, University of Colorado, Boulder, CO 
80309 USA}
\address[4]{Electrical Engineering Department, University of Colorado, Boulder, CO 
80309 USA}
\address[5]{Electrical Engineering Department, University of Colorado, Boulder, CO 
80309 USA}

\tfootnote{This paragraph of the first footnote will contain support 
information, including sponsor and financial support acknowledgment. For 
example, ``This work was supported in part by the U.S. Department of 
Commerce under Grant BS123456.''}

\markboth
{Author \headeretal: Preparation of Papers for IEEE TRANSACTIONS and JOURNALS}
{Author \headeretal: Preparation of Papers for IEEE TRANSACTIONS and JOURNALS}

\corresp{Corresponding author: First A. Author (e-mail: author@ boulder.nist.gov).}

\begin{abstract}

Unlike terrestrial cellular networks, a maritime communication system has to cover a vast area with quite limited base stations (BSs) due to the limitation of geographically available BS sites. Therefore, the system usually adopts high-powered BSs, and reducing power consumption is especially a critical issue therein. In this paper, we reduce the power consumption by utilizing the process information, which has not been considered in the previous studies. As both the wireless channels and the service demand are time-varying, it is impossible to accurately predict the complete channel state information (CSI) and service requirements. To overcome these two difficulties, we exploit the positional information of each vessel based on its specific shipping lane and timetable to estimate the slowly varying large-scale channel fading instead. Besides, we particularly focus on the delay-tolerant information distribution service. On that basis, we formulate a power consumption optimization problem, which is proved to be NP-hard. We propose an efficient algorithm to solve it in an iterative way with a polynomial time complexity. Simulation results reveal that the proposed process-oriented scheme significantly reduces the power consumption, although it has not taken the small-scale channel fading into consideration.

\end{abstract}

\begin{keywords}
Process-oriented, maritime communications, user scheduling, large-scale channel fading
\end{keywords}

\titlepgskip=-15pt

\maketitle

\section{Introduction}
\IEEEPARstart{W}{ith} the rapid development of marine activities such as marine tourism, offshore aquaculture and oceanic mineral exploration, there has been a growing demand for reliable and high-speed maritime communication services. Oceanic economic and cultural exchanges between countries, such as the Maritime Silk Road project in China, have further promoted this demand \cite{p322}\cite{p101}. In order to meet the increasing demand, several maritime communication network (MCN) projects have been developed in recent years, e.g., the BLUECOM+ project, the MarCom project, and the TRITON project \cite{p321}--\cite{p32}.
Unlike terrestrial cellular networks, a maritime communication system has quite limited geographically available base station (BS) sites. In order to cover a vast area with limited BSs, the system usually adopts high-powered BSs, which increases the operational costs of mobile network operators and poses a global threat to the environment \cite{p33}.
Accordingly, reducing power consumption is a critical issue for maritime communication.
Device-to-device (D2D) communication underlaid with cellular networks allows direct communication between mobile users \cite{p101}-\cite{p103}. With proximate communication opportunities, D2D communication may increase spectral efficiency, improve cellular coverage, as well as reduce power consumption.
Therefore, advanced wireless transmission and radio resource management techniques for D2D underlaid cellular communication system are in urgent need to solve the power reduction problem.


\subsection{Related work}
%cellular
So far, several energy-efficient techniques have been developed for terrestrial cellular networks.
In \cite{p3}, a joint antenna-subcarrier-power allocation scheme was proposed for distributed antenna systems (DASs) with limited backhaul capacity to maximize the energy efficiency while providing min-rate guaranteed services. In \cite{p6}, a matching algorithm of joint sub-channel assignment and power allocation was developed for non-orthogonal multiple access (NOMA) networks to maximize the total sum-rate with user fairness taken into consideration. In \cite{p4}, a joint power allocation and user scheduling algorithm based on dynamic programming (DP) was proposed for multi-user MIMO systems to minimize the total energy consumption under hard delay constraints. In \cite{p5}, a cross-layer cooperative user scheduling and power allocation scheme was developed for hybrid-delay services, and the fundamental tradeoff between delay and energy consumption was illustrated. More recently in \cite{p7}, a user scheduling and pilot assignment scheme was proposed for massive MIMO systems to serve the maximum number of users with guaranteed quality of service (QoS).

%D2D
There are also several power control scheme developed for terrestrial D2D underlaid cellular systems so far. Xiao Xiao et al. proposed a power optimization scheme with joint resource (i.e. subcarrier and bit allocation) allocation and mode selection in an OFDMA system with integrated D2D communications in \cite{p104}. In \cite{p105}, Namyoon Lee et al. proposed a random network model for a D2D underlaid cellular system using stochastic geometry and developed centralized and distributed power control algorithms. In \cite{p106}, a distributed power control algorithm was developed which iteratively determines the SINR targets in a mixed cellular and D2D environment and allocates transmit powers such that the overall power consumption is minimized subject to an sum-rate constraint. In \cite{p107}, a power-efficient mode selection and power allocation scheme in D2D underlaid cellular system was developed which based on exhaustive search of all possible mode combinations of the devices.

%问题
However, as both the wireless channels and the service demand are time-varying, it is impossible to accurately predict the complete CSI and users' requirements during the service process. Therefore, all of the above user scheduling schemes are state-oriented, i.e, based on the current CSI and service requirements, and the optimization is in a ${\left( {J + 1} \right) \times J}$ 2-dimensional subspace (user count $J$). As the service process information is ignored, the state-oriented schemes lose the potential gain in energy efficiency to a great extent, especially for maritime vessels with dynamic locations and service requirements.
In our previous study \cite{p108}, a process-oriented user scheduling scheme for maritime cellular-only communication system was developed. The optimization in \cite{p108} was in a  ${J \times T}$ 2-dimensional subspace (time slot count $T$).

\subsection{Contributions}
In this paper, we further explore a ${\left( {J + 1} \right) \times J \times T}$ 3-dimensional optimization subspace to reduce the power consumption for maritime D2D underlaid cellular communications by utilizing the service process information, which has not been considered in the previous studies.
The major challenge for our proposed scheme lies in the long-term prediction of the CSI, as well as the prediction of the users' requirements. We overcome these difficulties by fully utilizing the following unique features of maritime communications:
\textbf{(1)} As there are fewer scatterers on the sea than that in the terrestrial scenario, we exploit the position information of marine users to estimate and predict the slowly-varying large-scale CSI instead of the complete instantaneous CSI;
\textbf{(2)} The users' positions can be predicted based on their specific shipping lanes and timetables;
\textbf{(3)} Besides, we particularly focus on the delay-tolerant information distribution service, which is initiated and terminated when a marine user sails into and out of the BS's coverage, respectively, so that we can make long-term prediction of the users' requirements.

% maritime

Given the delay tolerant characteristic of maritime communication: mobile users focus more on reliability and data volume of cellular download rather than transmission delay, we can sacrifice delay for larger system capacity and less power consumption. Hence we focus on the delay-tolerant information distribution service in this study.
Since the shipping lanes of maritime mobile users are acquired beforehand, we can use the shipping lanes to predict long-time user locations. And the user locations are of great use in determining the long-time large-scale channel fading, according to the scarcity of scatterers on the sea, which makes it easier to estimate and predict the slowly varying large-scale channel fading. Therefore, we exploit the positional information of each vessel based on its specific shipping lane and timetable to estimate the large-scale channel fading instead of the complete CSI, as the research in \cite{p120} suggests that large-scale channel fading is a good estimate for the complete CSI.%冯伟老师文章
With delay-tolerant service assumption and large-scale channel fading estimation, we address the user requirement problem and the CSI prediction problem based on the characteristic of maritime communication system.

% d2d



% proposed method

On that basis, we formulate a power consumption optimization problem for user scheduling, aiming to minimize the power consumption while providing users with min-rate max-delay guaranteed services. The problem is proved to be NP-hard. To overcome the difficulties of solving the NP-hard problem, we decompose the problem into two simpler subproblems, and propose an efficient algorithm to solve it in an iterative way with a polynomial time complexity.
Simulation results reveal that the proposed process-oriented scheme significantly outperforms the state-oriented ones in terms of power consumption with the utilization of the service process information, although it has not taken the small-scale channel fading into consideration.



\section{System Model}

\Figure[t!](topskip=0pt, botskip=0pt, midskip=0pt){SysModel.eps}
{Maritime communication system for information
distribution service.\label{fig1}}

As shown in Fig.1, the following sections focus on the D2D underlaid cellular downlink transmission of a single-cell marine time communication system, which has an onshore BS equipped with $L$ antennas and  $J$ single-antenna users (ships) in the sea. We assume that the subcarrier bandwidth is ${B_s}$, and the total bandwidth of $N$ subcarriers is $B = N{B_s}$ .

In the studied system, D2D communications between ships use the same licensed band of cellular network, and the same air interface of the underlying cellular communication. As a result, D2D communications consumes part of the resources allocated to the cellular network, i.e., D2D communications also use the $N$ subcarriers whose bandwidths are ${B_s}$. At any given time, each D2D or cellular transmission link will use distinct subcarrier. Here in this paper by `link' we mean the transmission from BS/relay to user during a certain time period. We further assume that the $J$ single-antenna users can either receive data transmission from one source (BS/relay) or send data to another user at a time.

Without loss of generality, we assume the cell shape to be a semicircle with radius $R$. Each user sails into and out of the cell according to its shipping lane and timetable. For each user, delay-torrent service is assumed, and the total amount of the data required by the ${j^{th}}$ user is denoted by $C_j^{QoS}$. In order to simplify the problem, we only consider D2D and cellular communications of the ships in the semicircle. We also assume all the users request different data and the system has no D2D data reuse.

We further assume a modified 2-ray propagation model, since the sea surface is relatively flat. For a given subcarrier, we denote the composite channel gain from the BS/relay $i$ to the user $j$ at time slot $t$ by $\sqrt {{\beta _{i,j,t}}} {h_{i,j,t}}$. The small-scale fading vectors ${h_{i,j,t}}$ follows a complex Gaussian distribution with standard deviation ${\sigma _s} = 1$, i.e., ${h_{i,j,t}} \sim \mathbb{C}N\left( {0,{\mathbf{I}}} \right)$. The large-scale fading coefficient ${\beta _{i,j,t}}$ is expressed as
\begin{align}
{\beta _{i,j,t}} = {\left( {\frac{\lambda }{{4\pi {d_{i,j,t}}}}} \right)^2}{\left[ {2\sin \left( {\frac{{2\pi {h_t}{h_r}}}{{\lambda {d_{i,j,t}}}}} \right)} \right]^2}
\end{align}
where $\lambda $ is the carrier wavelength, ${d_{i,j,t}}$ is the distance between the BS/relay $i$ and the user $j$ at time slot $t$. The antenna height of the transmitter and the receiver are represented by $h_t$ and $h_r$ respectively.


To fully utilize the slow time-varying characteristic of the large-scale channel fading, we divide the total service time into $T$ time slots, each lasts $\Delta t$. The value $\Delta t$ is carefully chosen so that $\beta _{i,j,t}$ remains constant in each time slot $t$. Thus, we make it possible to estimate $\beta _{i,j,t}$ for $\forall t \in \left\{ {1,...,T} \right\}$ based on shipping lanes and timetable. With the large-scale channel fading (CSI) known beforehand, we can further design and implement a process-oriented scheme foe user scheduling.

The total energy consumption of the system consists of cellular (BS) transmission part and D2D transmission part. We model the energy consumption as

\begin{align}
{{E_{total}} = \sum\limits_{j = 1}^J {{E_j}}  = \sum\limits_{j = 1}^J {\left( {\sum\limits_{t = 1}^T {\sum\limits_{i = 0}^J {{P_{i,j,t}}\Delta t} } } \right)} }
\end{align}
here ${P_{i,j,t}}$ represents the average power consumed by the transimission from BS/relay $i$ to user $j$ during time slot $t$.

Our objective is to minimize the system energy consumption by means of user scheduling in cellular transmission and D2D transmission. We further denote the transmission link from BS/relay $i \in \left\{ {0,1,...,J} \right\}$($i = 0$ means BS, $i > 0$ means user relay) to user $j$ at time slot $t$ by $i \to j@t$. We assume BS/relay always use their max power $P_i^{\max }$ for transmission $i \to j@t$. For the transmission link $i \to j@t$, we denote the ratio of the used transmission time to time slot's duration by ${\eta _{i,j,t}} \in [0,1]$. ${\eta _{i,j,t}} = 0$ means no subcarrier is scheduled for the transmission $i \to j@t$, whereas ${\eta _{i,j,t}} \in \left( {0,1} \right]$ means a subcarrier is scheduled at time slot $t$ and the transmission uses ${\eta _{i,j,t}}$ of the time slot's duration. By ${C_{j,t}}$ we denote the total data volume user $j$ have by time slot $t$. Since the system has no D2D data reuse, user $j$ must have enough data ${C_{j,t}}$ in order to act as relay and transmit to another user $j'$ at $t$

Thus, we formulate the \textbf{energy consumption optimization problem} as

\begin{subequations}
\begin{align}
& \mathop {\min }\limits_{{\mathbf{{\rm H}}} \in {{\left[ {0,1} \right]}^{\left( {J + 1} \right) \times J \times T}}} \left\{ {\sum\limits_{t = 1}^T {\sum\limits_{j = 1}^J {\sum\limits_{i = 0}^J {{P_{i,j,t}}\Delta t} \,} } } \right\} \\
& {s.t.} \;\; \sum\limits_{i \ne j} {\left( {{\eta _{i,j,t}} > 0} \right)}  + \sum\limits_{j' \ne j} {\left( {{\eta _{j,j',t}} > 0} \right) \le {\text{1}}} \\
& \;\;\;\;\;\; \sum\limits_j {\sum\limits_i {\left( {{\eta _{i,j,t}} > 0} \right)} }  \le N \\
& \;\;\;\;\;\; r_{i,j,t}^{\max } = {\log _2}\left( {1 + \frac{{P_i^{\max }{\beta _{i,j,t}}{{\left| {{h_0}} \right|}^2}}}{{{\sigma ^2} + M}}} \right) \\
& \;\;\;\;\;\; {r_{i,j,t}} = r_{i,j,t}^{\max }{\eta _{i,j,t}},\; {P_{i,j,t}} = P_i^{\max }{\eta _{i,j,t}} \\
& \;\;\;\;\;\; {\left. {{{\left. {{C_{j,t}}} \right|}_{t = 0}} = 0, {C_{j,t}}} \right|_{t = T}} \ge C_j^{QoS} \\
& \;\;\;\;\;\; {C_{j,t}} = \sum\limits_{\tau  = 1}^t {\left( {\sum\limits_i {{r_{i,j,\tau }}}  - \sum\limits_{j'} {{r_{j,j',\tau }}} } \right)\Delta t} ,\; {C_{j,t}} \ge 0
\end{align}
\end{subequations}
where $P_i^{\max } = \left\{ {P_0^{\max },\left\{ {P_j^{\max }} \right\}} \right\}$ represents the maximum transmission power of BS and users, and ${P_{i,j,t}}$ represents on average how much power the transmission actual take in the time slot. ${\mathbf{{\rm H}}}{\text{ = }}{\left\{ {{\eta _{i,j,t}}} \right\}^{\left( {J + 1} \right) \times J \times T}}$ since we have to consider transmissions from ${J + 1}$ sources to $J$ targets at $T$ time slots, and our optimization is in a $\left( {J + 1} \right) \times J \times T$ 3-dimensional subspace. Constraint in (3b) guarantees that users can only receive from one source since they have single antenna. Constraint in (3c) guarantees that at most $N$ users can be severed simultaneously in the system, cellular or D2D, since there is only $N$ subcarriers. In (3e) we create two denotations ${r_{i,j,t}}$ and ${P_{i,j,t}}$ for simplicity in (3a) and (3g). (3f) and (3g) make sure that the QoS constraint is met and relays can only transmit all they have at most.


\section{Process-Oriented User Scheduling Scheme}
In this section, we focus on the reduction of system energy consumption while ensuring the users' service requirements (QoS). We decompose the optimization problem in (3) into 3 subproblems. Moreover, we proposed 3 efficient algorithms for the subproblems with polynomial time complexity to solve the NP-hard problem.

%\subsection{Problem Formulation}

\subsection{Problem Decomposition}

The problem in (3) is a discrete non-convex optimization problem and is NP-hard. Therefore, conventional methods for solving linear or convex optimization problems are no longer applicable. We decompose the problem into three simpler subproblems.

First, in \textbf{Subproblem 1: CT (Cellular Transmission)}, we consider the cellular-only transmission and ignore the subcarrier constraint.

Second, in \textbf{Subproblem 2: NCT (N-Subcarrier Cellular Transmission)}, we use an iterative algorithm to make sure the cellular-only transmission uses no more than $N$ subcarriers and get a suboptimal solution for the cellular-only system.

Last, in \textbf{Subproblem 3: DNCT (D2D-Aided N-Subcarrier Cellular Transmission)}, we consider the D2D underlaid cellular system. We use another iterative algorithm to substitute part of the cellular transmission links for cellular-\&-D2D transmission link clusters for less energy consumption. Each of the substitution link clusters consists of exact one cellular link $0 \to i'@{t_1}$ for BS to transmit data to relay ${i'}$ and one D2D relay transmission link $i' \to j@{t_2}$ for relay ${i'}$ to transmit to target user $j$. Two links $\left[ {{\rm{0}} \to i'@{t_1},i' \to j@{t_2}} \right]$ (one cellular and one D2D) in the substitution link cluster must use less energy combined than the original cellular-only link $0 \to j@{t_0}$ for improvement energy-wise. This results in that the transmission time of the two links in the substitution link cluster must be less than time slot duration $\Delta t$ ,i.e., ${\eta _{0,i',{t_1}}} \in \left( {0,1} \right)$, ${\eta _{i',j,{t_2}}} \in \left( {0,1} \right)$.

\subsection{Solution to \textbf{Subproblem 1: CT}}

For the first two subproblems, we consider a cellular-only system. We fix $i = 0$ since users can only receive data from BS.

\begin{subequations}
\begin{align}
& \mathop {\min }\limits_{{\mathbf{{\rm H}}}_{\mathbf{0}} \in {{\left[ {0,1} \right]}^{\left( {J + 1} \right) \times J \times T}}} \left\{ {\sum\limits_{t = 1}^T {\sum\limits_{j = 1}^J {{P_{0,j,t}}\Delta t} \,} } \right\} \\
& {s.t.} \;\; \sum\limits_j  {\left( {{\eta _{0,j,t}} > 0} \right)}   \le N \\
& \;\;\;\;\;\; r_{0,j,t}^{\max } = {\log _2}\left( {1 + \frac{{P_0^{\max }{\beta _{0,j,t}}{{\left| {{h_0}} \right|}^2}}}{{{\sigma ^2} + M}}} \right) \\
& \;\;\;\;\;\; {r_{0,j,t}} = r_{0,j,t}^{\max }{\eta _{0,j,t}},\; {P_{0,j,t}} = P_0^{\max }{\eta _{0,j,t}} \\
& \;\;\;\;\;\; {\left. {{{\left. {{C_{j,t}}} \right|}_{t = 0}} = 0, {C_{j,t}}} \right|_{t = T}} \ge C_j^{QoS} \\
& \;\;\;\;\;\; {C_{j,t}} = \sum\limits_{\tau  = 1}^t {{r_{0,j,\tau }}\Delta t} ,\; {C_{j,t}} \ge 0
\end{align}
\end{subequations}
$P_0^{\max }$ represents the maximum transmission power of BS. Here ${{\mathbf{{\rm H}}}_{\mathbf{0}}}{\text{ = }}{\left\{ {{\eta _{0,j,t}}} \right\}^{J \times T}}$ since the optimization is currently in a $J \times T$ 2-dimensional subspace (only one source BS) in the first two cellular-only sub problems. Constraint in (1b) is not necessary here since users can only receive data from one source, namely BS.

In the first subproblem, we optimize ${{\mathbf{{\rm H}}}_{\mathbf{0}}}$ with constraint (4c)-(4f), ignoring the subcarrier constraint in (4b). This means we assume that the BS can serve infinite number of users, and we ignore the subcarrier constraint. In this case, the optimization variables of different users in no longer correlated, and the optimal solution of this problem can be obtained by scheduling each user separately. The problem can be reduced to $\mathop {\min }\limits_{{{\mathbf{{\rm H}}}_{\mathbf{0}}} \in {{\left[ {0,1} \right]}^T}} \left\{ {\sum\limits_{t = 1}^T {{P_{0,j,t}}\Delta t} } \right\}$. Note that ${r_{0,j,t}}$ is a monotone increasing function of ${\beta _{0,j,t}}$, therefore we can obtain the optimal solution for each user by assigning time slots with best CSI. For each user, we find transmission link $0 \to j@t$ with best ${\beta _{0,j,t}}$ and set the ratio of the used transmission time ${\eta _{0,j,t} = 1}$ until the QoS constraint is met.

We further define ${{\bf{S}}_{\bf{1}}}$ as the set of chosen cellular transmission link at a specific time slot in \textbf{Subproblem 1: CT}, i.e., $\left( {0,j,t} \right) \in {\bf{S}}_{\bf{1}}$ if $0 \to j@t$ is a chosen transmission link. We propose Algorithm 1 to solve the first subproblem.

\begin{algorithm}[h]
\caption{Optimal User Scheduling for Cellular-only System Regardless of Subcarrier Count}
\label{alg:1}
\begin{algorithmic}[1]
\STATE Initialize ${{\bf{S}}_{\bf{1}}}=\phi$
\FOR{ each user $j$}
  \WHILE {${C_{j,T}} \ge {C_{j,QoS}}$ not met}
    \STATE Find $\left( {0,j,t} \right) = \arg \max \left\{ {r_{0,j,t}^{\max }} \right\}$.
    \STATE Set ${\eta _{0,j,t}} = 1$.
    \STATE Update ${C_{j,t}}$, ${P_{0,j,t}}$, ${{\bf{S}}_{\bf{1}}}={{\bf{S}}_{\bf{1}}} \cup \left\{ {\left( {0,j,t} \right)} \right\}$.
  \ENDWHILE
\ENDFOR
\end{algorithmic}
\end{algorithm}

\subsection{Solution to \textbf{Subproblem 2: NCT}}

The solution ${{\bf{S}}_{\bf{1}}}$ returned by Algorithm 1 is not a feasible one for the cellular-only system in (4a)-(4f) since (4b) has not been taken into account. We design an effective method to approach the suboptimal feasible solution ${{\bf{S}}_{\bf{2}}}$ for constraints in (4) iteratively.

As ${{\bf{S}}_{\bf{1}}}$ is the optimal solution for (4c)-(4f), the original problem in (4a)-(4f) is equivalent to minimizing the energy consumption gap between ${{\bf{S}}_{\bf{1}}}$ and the result ${{\bf{S}}_{\bf{2}}}$ in subproblem 2, and the second subproblem can be expressed as

\begin{subequations}
\begin{align}
& \mathop {\min }\limits_{{{\bf{{\rm H}}}_{\bf{0}}} \in {{\left[ {0,1} \right]}^{J \times T}}} \left\{ {\sum\limits_{t = 1}^T {\sum\limits_{j = 1}^J {\left( {\mathop {{P_{0,j,t}}}\limits_{\left( {0,j,t} \right) \in {{\bf{S}}_{\bf{2}}}}  - \mathop {{P_{0,j,t}}}\limits_{\left( {0,j,t} \right) \in {{\bf{S}}_{\bf{1}}}} } \right)} \Delta t} } \right\} \\
& {s.t.} \;\; \sum\limits_j  {\left( {{\eta _{0,j,t}} > 0} \right)}  \le N \\
& \;\;\;\;\;\; r_{0,j,t}^{\max } = {\log _2}\left( {1 + \frac{{P_0^{\max }{\beta _{0,j,t}}{{\left| {{h_0}} \right|}^2}}}{{{\sigma ^2} + M}}} \right) \\
& \;\;\;\;\;\; {r_{0,j,t}} = r_{0,j,t}^{\max }{\eta _{0,j,t}},\; {P_{0,j,t}} = P_0^{\max }{\eta _{0,j,t}} \\
& \;\;\;\;\;\; {\left. {{{\left. {{C_{j,t}}} \right|}_{t = 0}} = 0, {C_{j,t}}} \right|_{t = T}} \ge C_j^{QoS} \\
& \;\;\;\;\;\; {C_{j,t}} = \sum\limits_{\tau  = 1}^t {{r_{0,j,\tau }}\Delta t} ,\; {C_{j,t}} \ge 0
\end{align}
\end{subequations}
note that solving this subproblem is a process of adjusting the user scheduling result in ${{\bf{S}}_{\bf{1}}}$.

If the constraint (4b) isn't met in time slot ${t}$, we have to use alternative transmission links like $0 \to j@t'$ for replacement. These replacement will satisfy the N-subcarrier constraint at the cost of more energy consumption. According to the transmit-at-maximum-power policy in this system, we have to find substitutions with least system capacity impact.

To acquire the suboptimal solution ${{\bf{S}}_{\bf{2}}}$, we find transmission links in ${{\bf{S}}_{\bf{1}}}$ that have least impact on system capacity if substituted. We drop those transmission links out of ${{\bf{S}}_{\bf{2}}}$ and find substitution links to satisfy the QoS need under the N-subcarrier constraint in (4b).

The proposed iterative method is shown in Algorithm 2.

\begin{algorithm}[h]
\caption{Suboptimal User Scheduling for Cellular System}
\label{alg:1}
\begin{algorithmic}[1]
\STATE Initialize ${{\bf{S}}_{\bf{2}}}={{\bf{S}}_{\bf{1}}}$
\WHILE{ $\forall t,\sum\limits_j {{\eta _{0,j,t}}}  \le N$ not met}
  \STATE Find $\left( {0,j,t} \right) = \arg \mathop {\min }\limits_{\scriptstyle \left( {0,j,t} \right) \in {{\bf{S}}_{\bf{1}}} \atop
  \scriptstyle \left( {0,j,t'} \right) \notin {{\bf{S}}_{\bf{2}}}}  \left\{ {{r_{0,j,t}} - {r_{0,j,t'}}} \right\}$, where $\sum\limits_{j} {{\eta _{0,j,t'}}}  \le N - 1$, $\sum\limits_j {{\eta _{0,j,t}} > N} $.
  \STATE Set ${{\bf{S}}_{\bf{2}}}={{\bf{S}}_{\bf{2}}}\backslash \left\{ {\left( {0,j,t} \right)} \right\}$, ${\eta _{0,j,t}} = 0$.
  \WHILE {${{C_{j,T}} \ge {C_{j,QoS}}}$ not met}
    \STATE Find ${\left( {0,j,t} \right) = \arg \mathop {\max }\limits_{\left( {0,j,t} \right) \notin {{\bf{S}}_{\bf{2}}}} \left\{ {{r_{0,j,t}}} \right\}}$, where ${\sum\limits_j {{\eta _{0,j,t}}}  \le N - 1}$.
    \STATE Set ${\eta _{0,j,t}} = 1$.
    \STATE Update ${C_{j,t}}$, ${P_{0,j,t}}$, ${{\bf{S}}_{\bf{2}}}={{\bf{S}}_{\bf{2}}} \cup \left\{ {\left( {0,j,t} \right)} \right\}$.
  \ENDWHILE
\ENDWHILE
\end{algorithmic}
\end{algorithm}

\subsection{Solution to \textbf{Subproblem 3: DNCT}}

After solving the first two subproblems, we have already claimed an approximation of the optimal solution for the cellular-only system in a $J \times T$ subspace. In subproblem 3, we change part of the cellular transmission links into D2D transmission links for better energy efficiency. The optimization in a $\left( {J + 1} \right) \times J \times T$ subspace get the solution ${{\bf{S}}_{\bf{3}}}$. ${{\bf{S}}_{\bf{2}}}$ contains both cellular links like $0 \to j@t'$ and cellular-\&-D2D link clusters like $\left[ {{\rm{0}} \to i'@{t_1},i' \to j@{t_2}} \right]$.
Given that ${{\bf{S}}_{\bf{2}}}$ is only based on constraint (4a)-(4f), the original problem in (1a)-(1g) is equivalent to maximizing the energy consumption reduction between ${{\bf{S}}_{\bf{3}}}$ and the result ${{\bf{S}}_{\bf{2}}}$ in subproblem 2, and the third subproblem can be expressed as

\begin{subequations}
\begin{align}
& \mathop {\max }\limits_{{\bf{{\rm H}}} \in {{\left[ {0,1} \right]}^{\left( {J + 1} \right) \times J \times T}}} \left\{ {\sum\limits_{t = 1}^T {\sum\limits_{j = 1}^J {\left( {\mathop {{P_{0,j,t}}}\limits_{\left( {0,j,t} \right) \in {{\bf{S}}_{\bf{2}}}}  - \sum\limits_{i = 0}^J {\mathop {{P_{i,j,t}}}\limits_{\left( {i,j,t} \right) \in {{\bf{S}}_{\bf{3}}}} } } \right)\Delta t} } } \right\} \\
& {s.t.} \;\; \sum\limits_{i \ne j} {\left( {{\eta _{i,j,t}} > 0} \right)}  + \sum\limits_{j' \ne j} {\left( {{\eta _{j,j',t}} > 0} \right) \le {\text{1}}} \\
& \;\;\;\;\;\; \sum\limits_j {\sum\limits_i {\left( {{\eta _{i,j,t}} > 0} \right)} }  \le N \\
& \;\;\;\;\;\; r_{i,j,t}^{\max } = {\log _2}\left( {1 + \frac{{P_i^{\max }{\beta _{i,j,t}}{{\left| {{h_0}} \right|}^2}}}{{{\sigma ^2} + M}}} \right) \\
& \;\;\;\;\;\; {r_{i,j,t}} = r_{i,j,t}^{\max }{\eta _{i,j,t}},\; {P_{i,j,t}} = P_i^{\max }{\eta _{i,j,t}} \\
& \;\;\;\;\;\; {\left. {{{\left. {{C_{j,t}}} \right|}_{t = 0}} = 0, {C_{j,t}}} \right|_{t = T}} \ge C_j^{QoS} \\
& \;\;\;\;\;\; {C_{j,t}} = \sum\limits_{\tau  = 1}^t {\left( {\sum\limits_i {{r_{i,j,\tau }}}  - \sum\limits_{j'} {{r_{j,j',\tau }}} } \right)\Delta t} ,\; {C_{j,t}} \ge 0
\end{align}
\end{subequations}
here ${\bf{{\rm H}}}{\text{ = }}{\left\{ {{\eta _{i,j,t}}} \right\}^{\left( {J + 1} \right) \times J \times T}}$ since the optimization is now in a $\left( {J + 1} \right) \times J \times T$ subspace: there are $\left( {J + 1} \right)$ sources (BS/users), $J$ targets and $T$ time slots. For simplicity, by $0 \to j@{t_0}$ we denote the cellular transmission link in ${{\bf{S}}_{\bf{2}}}$ that is to be replaced by a cellular-\&-D2D link cluster in subproblem 3. Whereas the substitution link cluster $\left[ {{\rm{0}} \to i'@{t_1},i' \to j@{t_2}} \right]$ in this paper consist of exact one cellular link $0 \to i'@{t_1}$ and one D2D relay transmission link $i' \to j@{t_2}$. Two links (one cellular and one D2D) in the substitution link cluster must use less energy combined than the original cellular one.

We propose another iterative method as shown in Algorithm 3.

\begin{algorithm}[h]
\caption{Suboptimal User Scheduling for Cellular System}
\label{alg:1}
\begin{algorithmic}[1]
\STATE Initialize ${{\bf{S}}_{\bf{3}}}={{\bf{S}}_{\bf{2}}}$
\FOR{all user $j$}
  \STATE Set ${\bf{R}} = \phi $ as group for all plausible cellular-\&-D2D link clusters.
  \STATE $r_0^{\min } = \mathop {\min }\limits_{\left( {0,j,t} \right) \in {{\bf{S}}_{\bf{2}}}} \left\{ {r_{0,j,t}^{\max }{\eta _{0,j,t}}} \right\}$.
  \FOR{all time slot ${t_2}$}
    \FOR{all relays $i' \ne j$}
      \IF{${i'}$ \& $j$ \& SYSTEM are FREE at ${t_2}$ and $r_{i',j,{t_2}}^{\max } \geqslant r_0^{\min }$}
        \FOR{all time slot ${t_1}$ where ${i'}$ \& SYSTEM are FREE at ${t_1}$ and ${t_1} \ne {t_2}$ and $r_{0,i',{t_1}}^{\max } \geqslant r_0^{\min }$}
          \IF{$\left\{ \begin{array}{c}
{C_{i',{t_2} - 1}} \ge r_0^{\min }\Delta t,{\bf{if}} \,\, {t_1} > {t_2}\\
{C_{i',{t_2} - 1}} + r_{0,i',{t_1}}^{\max }\Delta t \ge r_0^{\min }\Delta t, \bf{else}
\end{array} \right.$}
            \STATE Set ${\bf{R = R}} \cup \left\{ {\left[ {\left( {0,i',{t_1}} \right),\left( {i',j,{t_2}} \right)} \right]} \right\}$.
          \ENDIF
        \ENDFOR
      \ENDIF
    \ENDFOR
  \ENDFOR
  \WHILE{${\bf{R}} \ne \phi $}
    \STATE Update $r_0^{\min } = \mathop {\min }\limits_{\left( {0,j,t} \right) \in {{\bf{S}}_{\bf{2}}}} \left\{ {r_{0,j,t}^{\max }{\eta _{0,j,t}}} \right\}$.
    \IF{${\eta _{0,i',{t_1}}} + \frac{{r_0^{\min }}}{{r_{0,i',{t_1}}^{\max }}} \leqslant 1$ and ${\eta _{i',j,{t_2}}} + \frac{{r_0^{\min }}}{{r_{i',j,{t_2}}^{\max }}} \leqslant 1$ where $\left[ {\left( {0,i',{t_1}} \right),\left( {i',j,{t_2}} \right)} \right] \in \bf{R}$ and ${i'}$ \& $j$ \&  SYSTEM are FREE at ${t_2}$ and ${i'}$ \& SYSTEM are FREE at ${t_1}$}
      \STATE $\Delta P = {P_{0,j,{t_0}}} - \left( {{P_{0,i',{t_1}}} + {P_{i',j,{t_2}}}} \right)$
    \ELSE{}
      \STATE $\Delta P =  - \infty $
    \ENDIF
    \IF{$\max \left\{ {\Delta P} \right\} > 0$}
      \STATE $\left[ {\left( {0,j,{t_0}} \right),\left( {0,i',{t_1}} \right),\left( {i',j,{t_2}} \right)} \right] = \arg \max \left\{ {\Delta P} \right\}$.
      \STATE Set ${\eta _{0,j,{t_0}}} = 0,{\eta _{0,i',{t_1}}} +  = \frac{{r_0^{\min }}}{{r_{0,i',{t_1}}^{\max }}},{\eta _{i',j,{t_2}}} +  = \frac{{r_0^{\min }}}{{r_{i',j,{t_2}}^{\max }}}$.
      \STATE Update ${{\bf{S}}_{\bf{3}}} \leftarrow {{\bf{S}}_{\bf{3}}}\backslash \left\{ {\left( {0,j,{t_0}} \right)} \right\},{{\bf{S}}_{\bf{3}}} \leftarrow {{\bf{S}}_{\bf{3}}} \cup \left\{ {\left[ {\left( {0,i',{t_1}} \right),\left( {i',j,{t_2}} \right)} \right]} \right\}$, update ${C_{j,t}},{C_{i',t}}$ and ${P_{0,j,{t_0}}},{P_{0,i',{t_1}}},{P_{i',j,{t_2}}}$.
      \STATE $\bf{R} \leftarrow \bf{R}\backslash \left\{ {\left[ {\left( {0,i',{t_1}} \right),\left( {i',j,{t_2}} \right)} \right]} \right\}$.
    \ELSE
      \STATE Break.
    \ENDIF
  \ENDWHILE
\ENDFOR
\end{algorithmic}
\end{algorithm}


For each user $j$, we first record all plausible cellular-\&-D2D link clusters like $\left[ {{\rm{0}} \to i'@{t_1},i' \to j@{t_2}} \right]$ in a temporary set $\bf{R}$. Here `plausible' means that the single antenna constraint in (6b) and the N-subcarrier constraint in (6c) are satisfied and both links in the cluster are at speed greater than the that of the original links. Further exploration will be conducted in the plausible cellular-\&-D2D link cluster set $\bf{R}$.

Once we have the plausible set $\bf{R}$, we check the cellular link and the D2D link in each cluster to see if they are capable of substitution, i.e., whether there are enough time unused in those link to complete the transmission in the original cellular link $0 \to j @{t_0}$. Of all the cellular-\&-D2D link clusters that pass the test, we find the combination of cluster $\left[ {{\rm{0}} \to i'@{t_1},i' \to j@{t_2}} \right]$ and original link $0 \to j @{t_0}$ that save most power, remove $0 \to j @{t_0}$ from ${{\bf{S}}_{\bf{3}}}$ and move $\left[ {{\rm{0}} \to i'@{t_1},i' \to j@{t_2}} \right]$ from $\bf{R}$ to ${{\bf{S}}_{\bf{3}}}$. Continue those steps until the plausible cluster set $\bf{R}$ become empty or there are no power gain from substitution.



Here in Algorithm 3 ``${i'}$ \& $j$ \& SYSTEM are FREE at ${t_2}$'' means that
\begin{subnumcases}
{}
\sum\limits_{{i^*} \ne j} {\left( {{\eta _{{i^*},j,{t_2}}} > 0} \right)}  + \sum\limits_{{j^*} \ne j} {\left( {{\eta _{j,{j^*},{t_2}}} > 0} \right)}  \le 1\\
\sum\limits_{{i^*} \ne i'} {\left( {{\eta _{{i^*},i',{t_2}}} > 0} \right)}  + \sum\limits_{{j^*} \ne i'} {\left( {{\eta _{i',{j^*},{t_2}}} > 0} \right)}  \le 1 \\
\sum\limits_{{j^*}} {\sum\limits_{{i^*}} {\left( {{\eta _{{i^*},{j^*},{t_2}}}} \right)} }  \le N
\end{subnumcases}
and ``${i'}$ \& SYSTEM are FREE at ${t_1}$'' means that
\begin{subnumcases}
{}
{\sum\limits_{{i^*} \ne i'} {\left( {{\eta _{{i^*},i',{t_1}}} > 0} \right)}  + \sum\limits_{{j^*} \ne i'} {\left( {{\eta _{i',{j^*},{t_1}}} > 0} \right) \le 1}}\\
{\sum\limits_{{j^*}} {\left( {\sum\limits_{{i^*}} {\left( {{\eta _{{i^*},{j^*},{t_1}}} > 0} \right)} } \right)}  \le N}
\end{subnumcases}
therefore the system constraint in (6b) and (6c) is met.

\section{Simulation Results}

In this section, we provide numerical results for the cellular-only method in the first two subproblems and the proposed D2D method in the third subproblem, as well as a reference greedy cellular-only method, which based on current CSI only. The reference method optimize the cellular-only systme based on current CSI, therefore the optimization is in a 1-dimensional subspace $J$, rather than the $J \times T$ 2-dimensional subspace for subproblem 1 and 2 or the $\left( {J + 1} \right) \times J \times T$ 3-dimensional subspace in subproblem 3. In the reference cellular-only method, in each time slot, we find and choose $N$ ships that have highest transmission speed under given BS broadcast power and current CSI.

As for the simulation parameters, the BS is located in the central position at the plane, while the ships traverse along two intersecting shipping lanes. Moreover, the two lanes have same amount of ships. Ships leave the harbors every 15 minutes, and all sail at the speed of 36km/h. We assume that the system uses a carrier frequency of 1.9GHz , and has 3 subcarriers, which have identical bandwidth 2MHz. The BS power for cellular-only transmission is set to be 10W whereas the ships' D2D transmission power are 1W, since they are arguably smaller in size. The antenna height of the BS and the ships is 100m and 10m respectively. The power density of the additive white Gaussian noise is -110dBm/Hz.



%\begin{figure} [htb]
%\begin{center}
%\includegraphics*[width=9cm]{Cqos.eps}
%\end{center}
%\vspace*{-4mm} \caption{Average energy consumption per user $E_{avg}$ versus the QoS constraint ${C_{QoS}}$}\label{fig:2}
%\vspace*{-2mm}
%\end{figure}


\Figure[t!](topskip=0pt, botskip=0pt, midskip=0pt){Cqos.eps}
{Average energy consumption per user $E_{avg}$ versus the QoS constraint ${C_{QoS}}$.\label{fig2}}


Fig.2 shows the bit-wise average power consumption under different QoS constraint.

As we can see, our proposed D2D method outmatches the cellular-only method and the reference method, especially when there is a smaller QoS constraint. When the QoS constraint is 1Gbits/user, the D2D method consummates 50\% less energy than the cellular-only method. The proposed D2D method's energy consumption approaches the cellular-only method as the QoS constraint gets larger. This is because the cellular-only part in first two subproblems might take up too many time slots and left the D2D method in the third subproblem few time slots with feasible D2D links to choose from.

The reference method's energy consumption decreases as the QoS constraint get larger, while the proposed methods' energy consumptions increase. The decrease in reference method's energy consumption is because the reference method is a greedy one, and it aims to meet the QoS constraint as soon as possible. When the QoS constraint is smaller, the reference method may choose many time slots with relatively low ${\beta _{0,j,t}}$ and can still satisfy the QoS constraint. When the QoS constraint gets larger, the reference method has to choose more time slots, and therefore the ratio of time slots with relatively low ${\beta _{0,j,t}}$ to total chosen time slots decreases. Thus, the reference method's energy consumption per user per Gbit decreases.

The rise in proposed methods' energy consumption is because the proportion of chosen time slots with relatively low speed gets larger when the QoS constraint increase. Moreover, the reference method can only meet the QoS constraint of 30Gbits/user while our proposed method can serve as much as 90Gbits/user.

%\begin{figure} [htb]
%\begin{center}
%\includegraphics*[width=9cm]{Tranges.eps}
%\end{center}
%\vspace*{-4mm} \caption{Average energy consumption per user $E_{avg}$ versus the percentage of pre-acquired CSI} \label{fig:3}
%\vspace*{-2mm}
%\end{figure}

\Figure[t!](topskip=0pt, botskip=0pt, midskip=0pt){Tranges.eps}
{Average energy consumption per user $E_{avg}$ versus the percentage of pre-acquired CSI.\label{fig3}}



Fig.3 demonstrates the relationship between average energy consumption and the percentage of time slots whose CSI we can acquire in advance. The QoS constraint here is 1Gbits/user.
When we can only acquire present CSI, our proposed method retrogresses to the reference method. The longer can we predict the CSI, the more feasible transmission time slots we can choose from in our method and therefore the more improvement we can get from our process-oriented D2D and cellular-only methods.

%\begin{figure} [htb]
%\begin{center}
%\includegraphics*[width=9cm]{Ns.eps}
%\end{center}
%\vspace*{-4mm} \caption{Average energy consumption per user $E_{avg}$ versus the number of subcarriers} \label{fig:4}
%\vspace*{-2mm}
%\end{figure}

\Figure[t!](topskip=0pt, botskip=0pt, midskip=0pt){Ns.eps}
{Average energy consumption per user $E_{avg}$ versus the number of subcarriers.\label{fig4}}



Average energy consumption versus number of subcarriers is shown in Fig.4. Half of the ships in the system hold still in the BS coverage, while the other half still traverse along the shipping lanes at 36km/h. The QoS constraint here is 10Gbits/user for traversing ships and 1Gbits/user for ships that hold still.

When there is only 1 subcarrier, our proposed D2D method's average energy consumption is very close to the cellular-only method. This is because the QoS constraint is relatively large and hence cellular-only method takes up too many time slots since there being only 1 subcarrier. As a result, there are few time slots available for the D2D optimization.

Our proposed D2D method gets better when there are more subcarriers. The reference cannot meet the QoS need until there are more than 5 subcarriers. Since the reference method is a greedy one and aims to meet the QoS need as soon as possible, its average energy consumption gets larger as the subcarriers increases.


\section{Conclusion}\label{sec:4}

In this paper, we focused on the process-oriented user scheduling in a D2D underlaid D2D maritime communication system. Our aim is to minimize the average downlink transmission energy while providing users with min-rate max-delay guaranteed services. By utilizing each users' positional information acquired from their specific shipping lanes, we make it possible to estimate the large-scale channel fading during the whole service process. Based on that, we formulate the power consumption optimization problem by partitioning the total service duration into time solts. Further, we decompose the NP-hard problem into 3 subproblems. By solving the first two subproblem we acquire a sub-optimal solution for cellular system in a 2-dimensional optimization subspace. During the 3rd subproblem, we further explore the 3-dimensional subspace with D2D transmission links. The iterative algorithm we proposed can solve the 3 subproblems with a polynomial time complexity. Simulation results show that the proposed process-oriented schemes significantly enhances the system performance in terms of energy consumption.


\appendices

Appendixes, if needed, appear before the acknowledgment.

\section*{Appendix A \\ Proof of Theorem 1}

\begin{subnumcases}
{}
A(k,(k - 1)M + m) =  {r_{k,m}^{\max }}\\
A(t + K,(k - 1)M + m) = -1 \\
A(\text {others}) = 0
\end{subnumcases}


\begin{align}
\mathbf{A}=\left[ \begin{matrix}
   1 & 1 & 0 & 1 & 0  \\
   1 & 0 & -1 & 0 & -1  \\
\end{matrix} \right], \text{and} \; \mathbf{b}=\left[ \begin{matrix}
   1  \\
   -1  \\
\end{matrix} \right]
\end{align}


\section*{Acknowledgment}

This work was partially supported by the National Basic Research Program of China under grant No. 2013CB329001, and
the National Science Foundation of China under grant No. 91638205 and grant No. 61621091.

\section{Reference}

\begin{itemize}

\item \emph{Basic format for books:}\\
J. K. Author, ``Title of chapter in the book,'' in \emph{Title of His Published Book, x}th ed. City of Publisher, (only U.S. State), Country: Abbrev. of Publisher, year, ch. $x$, sec. $x$, pp. \emph{xxx--xxx.}\\
See \cite{b1,b2}.

\item \emph{Basic format for periodicals:}\\
J. K. Author, ``Name of paper,'' \emph{Abbrev. Title of Periodical}, vol. \emph{x, no}. $x, $pp\emph{. xxx--xxx, }Abbrev. Month, year, DOI. 10.1109.\emph{XXX}.123456.\\
See \cite{b3}--\cite{b5}.

\item \emph{Basic format for reports:}\\
J. K. Author, ``Title of report,'' Abbrev. Name of Co., City of Co., Abbrev. State, Country, Rep. \emph{xxx}, year.\\
See \cite{b6,b7}.

\item \emph{Basic format for handbooks:}\\
\emph{Name of Manual/Handbook, x} ed., Abbrev. Name of Co., City of Co., Abbrev. State, Country, year, pp. \emph{xxx--xxx.}\\
See \cite{b8,b9}.

\item \emph{Basic format for books (when available online):}\\
J. K. Author, ``Title of chapter in the book,'' in \emph{Title of
Published Book}, $x$th ed. City of Publisher, State, Country: Abbrev.
of Publisher, year, ch. $x$, sec. $x$, pp. \emph{xxx--xxx}. [Online].
Available: \underline{http://www.web.com}\\
See \cite{b10}--\cite{b13}.

\item \emph{Basic format for journals (when available online):}\\
J. K. Author, ``Name of paper,'' \emph{Abbrev. Title of Periodical}, vol. $x$, no. $x$, pp. \emph{xxx--xxx}, Abbrev. Month, year. Accessed on: Month, Day, year, DOI: 10.1109.\emph{XXX}.123456, [Online].\\
See \cite{b14}--\cite{b16}.

\item \emph{Basic format for papers presented at conferences (when available online): }\\
J.K. Author. (year, month). Title. presented at abbrev. conference title. [Type of Medium]. Available: site/path/file\\
See \cite{b17}.

\item \emph{Basic format for reports and handbooks (when available online):}\\
J. K. Author. ``Title of report,'' Company. City, State, Country. Rep. no., (optional: vol./issue), Date. [Online] Available: site/path/file\\
See \cite{b18,b19}.

\item \emph{Basic format for computer programs and electronic documents (when available online): }\\
Legislative body. Number of Congress, Session. (year, month day). \emph{Number of bill or resolution}, \emph{Title}. [Type of medium]. Available: site/path/file\\
\textbf{\emph{NOTE: }ISO recommends that capitalization follow the accepted practice for the language or script in which the information is given.}\\
See \cite{b20}.

\item \emph{Basic format for patents (when available online):}\\
Name of the invention, by inventor's name. (year, month day). Patent Number [Type of medium]. Available: site/path/file\\
See \cite{b21}.

\item \emph{Basic format}\emph{for conference proceedings (published):}\\
J. K. Author, ``Title of paper,'' in \emph{Abbreviated Name of Conf.}, City of Conf., Abbrev. State (if given), Country, year, pp. \emph{xxxxxx.}\\
See \cite{b22}.

\item \emph{Example for papers presented at conferences (unpublished):}\\
See \cite{b23}.

\item \emph{Basic format for patents}$:$\\
J. K. Author, ``Title of patent,'' U.S. Patent \emph{x xxx xxx}, Abbrev. Month, day, year.\\
See \cite{b24}.

\item \emph{Basic format for theses (M.S.) and dissertations (Ph.D.):}
\begin{enumerate}
\item J. K. Author, ``Title of thesis,'' M.S. thesis, Abbrev. Dept., Abbrev. Univ., City of Univ., Abbrev. State, year.
\item J. K. Author, ``Title of dissertation,'' Ph.D. dissertation, Abbrev. Dept., Abbrev. Univ., City of Univ., Abbrev. State, year.
\end{enumerate}
See \cite{b25,b26}.

\item \emph{Basic format for the most common types of unpublished references:}
\begin{enumerate}
\item J. K. Author, private communication, Abbrev. Month, year.
\item J. K. Author, ``Title of paper,'' unpublished.
\item J. K. Author, ``Title of paper,'' to be published.
\end{enumerate}
See \cite{b27}--\cite{b29}.

\item \emph{Basic formats for standards:}
\begin{enumerate}
\item \emph{Title of Standard}, Standard number, date.
\item \emph{Title of Standard}, Standard number, Corporate author, location, date.
\end{enumerate}
See \cite{b30,b31}.

\item \emph{Article number in~reference examples:}\\
See \cite{b32,b33}.

\item \emph{Example when using et al.:}\\
See \cite{b34}.

\end{itemize}

\begin{thebibliography}{10}

\bibitem{p32}
M. Zhou, V. D. Hoang, H. Harada, et al., ``TRITON: High-Speed Maritime Wireless Mesh Network,''
\emph{IEEE Wireless Communications}, vol. 20, no. 5, pp.~134-142, 2013.

\bibitem{p322}
T. Roste, K. Yang, F. Bekkadal, ``Coastal Coverage for Maritime Broadband Communications'',
\emph{MTS/IEEE OCEANS - Bergen}, pp.~1-8, 2013.

\bibitem{p33}
S. Buzzi, Chih-Lin I, T. E. Klein, H. V. Poor, et al., ``A Survey of Energy-Efficient Techniques for 5G Networks and Challenges Ahead'',
\emph{IEEE Journal on Selected Areas in Communications (JSAC)}, vol. 34, no. 4, pp.~697-709, 2016.

\bibitem{p3}
X. Li, X. Ge, X. Wang, et al., ``Energy Efficiency Optimization: Joint Antenna-Subcarrier-Power Allocation in OFDM-DASs'',
\emph{IEEE Transactions on Wireless Communications (TWC)}, vol. 15, no. 11, pp.~7470-7483, 2016.

\bibitem{p6}
B. Di, L. Song, Y. Li, ``Sub-Channel Assignment, Power Allocation, and User Scheduling for Non-Orthogonal Multiple Access Networks'',
\emph{IEEE Transactions on Wireless Communications (TWC)}, vol. 15, no. 11, pp.~7686-7698, 2016.

\bibitem{p4}
L. Shan, R. Miura, ``Energy-Efficient Scheduling under Hard Delay Constraints for Multi-User MIMO System'',
\emph{International Symposium on Wireless Personal Multimedia Communications (WPMC)}, pp.~696-699, 2014.

\bibitem{p5}
S. Cao, Q. Cui, Y. Shi, H. Wang, X. Ma, ``Cross-Layer Cooperative Delay-Energy Tradeoff Scheme for Hybrid Services in Cellular Networks'',
\emph{IEEE 79th Vehicular Technology Conference (VTC Spring)}, pp.~1-5, 2014.

\bibitem{p7}
X. Xiong, B. Jiang, X. Gao, X. You, ``QoS-Guaranteed User Scheduling and Pilot Assignment for Large-Scale MIMO-OFDM Systems'',
\emph{IEEE Transactions on Vehicular Technology (TVT)}, vol. 65, no. 8, pp.~6275-6289, 2016.

%\bibitem{p8}
%Rahul Singh, Alexander Stolyar, ``MaxWeight scheduling: Smoothness of the service process'',
%\emph{IEEE 35th Annual IEEE International Conference on Computer Communications (INFOCOM)}, pp.~1-9, 2016.

\bibitem{p0}
Sumayya Balkees P A, K. Sasidhar, S. Rao, ``A Survey Based Analysis of Propagation Models over the Sea'',
\emph{International Conference on Advances in Computing, Communications and Informatics (ICACCI)}, pp.~69-75, 2015.

\bibitem{p1}
Y. Zhao, J. Ren, and X. Chi, ``Maritime Mobile Channel Transmission Model Based on ITM'',
\emph{2nd International Symposium on Computer, Communication, Control and Automation (3CA)}, Atlantis Press, 2013.

\bibitem{p2}
J. C. Reyes-Guerrero, M. Bruno, L. A. Mariscal, A. Medouri, ``Buoy-to-Ship Experimental Measurements over Sea at 5.8 GHz near Urban Environments'',
\emph{11th Mediterranean Microwave Symposium (MMS)}, pp.~320-324, 2011.

\bibitem{p22}
C. He, G. Y. Li, F. Zheng, X. You, ``Power Allocation Criteria for Distributed Antenna Systems'',
\emph{IEEE Transactions on Vehicular Technology (TVT)}, vol. 64, no. 11, pp.~5083-5090, 2015.

\bibitem{p41}
H. Shin, J. H. Lee, ``Capacity of Multiple-Antenna Fading Channels: Spatial Fading Correlation, Double Scattering, and Keyhole'',
\emph{IEEE Transactions on Information Theory}, vol. 49, no. 10, pp.~2636-2647, 2003.

\bibitem{p9}
F. Fernandes, A. Ashikhmin, T. L. Marzetta, ``Inter-Cell Interference in Noncooperative TDD Large Scale Antenna Systems'',
\emph{IEEE Journal on Selected Areas in Communications (JSAC)}, vol. 31, no. 2, pp.~192-201, 2013.

\bibitem{p11}
N. Souto, R. Dinis, ``MIMO Detection and Equalization for Single-Carrier Systems Using the Alternating Direction Method of Multipliers'',
\emph{IEEE Signal Processing Letters (SPL)}, vol. 23, no. 12, pp.~1751-1755, 2016.

\bibitem{p123}
H. Zhang, C. Jiang, N. C. Beaulieu, X. Chu, X. Wen, M. Tao, ``Resource Allocation in Spectrum-Sharing OFDMA Femtocells With Heterogeneous Services'',
\emph{IEEE Transactions on Communications}, vol. 62, no. 7, pp.~2366-2377, 2014.

\bibitem{p8}
T. Yang, H. Liang, N. Cheng, R. Deng, X. Shen, ``Efficient Scheduling for Video Transmissions in Maritime Wireless Communication Networks'',
\emph{IEEE Transactions on Vehicular Technology (TVT)}, vol. 64, no. 9, pp.~4215-4229, 2015.

\bibitem{p10}
T. H. Cormen, ``Introduction to Algorithms'', MIT press, 2009.

\end{thebibliography}

\begin{IEEEbiography}[{\includegraphics[width=1in,height=1.25in,clip,keepaspectratio]{a1.png}}]{First A. Author} (M'76--SM'81--F'87) and all authors may include 
biographies. Biographies are often not included in conference-related
papers. This author became a Member (M) of IEEE in 1976, a Senior
Member (SM) in 1981, and a Fellow (F) in 1987. The first paragraph may
contain a place and/or date of birth (list place, then date). Next,
the author's educational background is listed. The degrees should be
listed with type of degree in what field, which institution, city,
state, and country, and year the degree was earned. The author's major
field of study should be lower-cased. 

The second paragraph uses the pronoun of the person (he or she) and not the 
author's last name. It lists military and work experience, including summer 
and fellowship jobs. Job titles are capitalized. The current job must have a 
location; previous positions may be listed 
without one. Information concerning previous publications may be included. 
Try not to list more than three books or published articles. The format for 
listing publishers of a book within the biography is: title of book 
(publisher name, year) similar to a reference. Current and previous research 
interests end the paragraph. The third paragraph begins with the author's 
title and last name (e.g., Dr.\ Smith, Prof.\ Jones, Mr.\ Kajor, Ms.\ Hunter). 
List any memberships in professional societies other than the IEEE. Finally, 
list any awards and work for IEEE committees and publications. If a 
photograph is provided, it should be of good quality, and 
professional-looking. Following are two examples of an author's biography.
\end{IEEEbiography}

\begin{IEEEbiography}[{\includegraphics[width=1in,height=1.25in,clip,keepaspectratio]{a2.png}}]{Second B. Author} was born in Greenwich Village, New York, NY, USA in 
1977. He received the B.S. and M.S. degrees in aerospace engineering from 
the University of Virginia, Charlottesville, in 2001 and the Ph.D. degree in 
mechanical engineering from Drexel University, Philadelphia, PA, in 2008.

From 2001 to 2004, he was a Research Assistant with the Princeton Plasma 
Physics Laboratory. Since 2009, he has been an Assistant Professor with the 
Mechanical Engineering Department, Texas A{\&}M University, College Station. 
He is the author of three books, more than 150 articles, and more than 70 
inventions. His research interests include high-pressure and high-density 
nonthermal plasma discharge processes and applications, microscale plasma 
discharges, discharges in liquids, spectroscopic diagnostics, plasma 
propulsion, and innovation plasma applications. He is an Associate Editor of 
the journal \emph{Earth, Moon, Planets}, and holds two patents. 

Dr. Author was a recipient of the International Association of Geomagnetism 
and Aeronomy Young Scientist Award for Excellence in 2008, and the IEEE 
Electromagnetic Compatibility Society Best Symposium Paper Award in 2011. 
\end{IEEEbiography}

\begin{IEEEbiography}[{\includegraphics[width=1in,height=1.25in,clip,keepaspectratio]{a3.png}}]{Third C. Author, Jr.} (M'87) received the B.S. degree in mechanical 
engineering from National Chung Cheng University, Chiayi, Taiwan, in 2004 
and the M.S. degree in mechanical engineering from National Tsing Hua 
University, Hsinchu, Taiwan, in 2006. He is currently pursuing the Ph.D. 
degree in mechanical engineering at Texas A{\&}M University, College 
Station, TX, USA.

From 2008 to 2009, he was a Research Assistant with the Institute of 
Physics, Academia Sinica, Tapei, Taiwan. His research interest includes the 
development of surface processing and biological/medical treatment 
techniques using nonthermal atmospheric pressure plasmas, fundamental study 
of plasma sources, and fabrication of micro- or nanostructured surfaces. 

Mr. Author's awards and honors include the Frew Fellowship (Australian 
Academy of Science), the I. I. Rabi Prize (APS), the European Frequency and 
Time Forum Award, the Carl Zeiss Research Award, the William F. Meggers 
Award and the Adolph Lomb Medal (OSA).
\end{IEEEbiography}

\EOD

\end{document}
