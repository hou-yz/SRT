%\documentclass[conference]{IEEEtran}
  %\IEEEoverridecommandlockouts
  \documentclass[journal]{IEEEtran}
    % correct bad hyphenation here
    %\documentclass[10pt,draftclsnofoot, onecolumn]{IEEEtran}
    %\documentclass[10pt, draftclsnofoot,onecolumn]{IEEEtran}
    \hyphenation{op-tical net-works semi-conduc-tor}
    \usepackage{graphicx,amssymb,lineno}
    \usepackage{amsmath,amsfonts,amssymb}
    \usepackage{cases}
    \newtheorem{theorem}{Theorem}
    \newtheorem{lemma}{Lemma}
    \newtheorem{definition}{Definition}
    \usepackage{algorithm}
    \usepackage{caption}
    \usepackage{algorithmic}
    \usepackage[usenames]{color}
    \usepackage{subfigure}
    \usepackage{graphicx}
    
    \setlength{\columnsep}{0.22in}
    
   %\renewcommand{\thealgorithm}{}
  \DeclareCaptionFormat{myformat}{#3}
  \captionsetup[algorithm]{format=myformat}
  
  
   \begin{document}
   \title{User Scheduling for Maritime Ship-to-Ship/Shore Communications Based on Large-Scale CSI}
   %\author{Michael~Shell,~\IEEEmembership{Member,~IEEE,} John~Doe,~\IEEEmembership{Fellow,~OSA,} and~Jane~Doe,~\IEEEmembership{Life~Fellow,~IEEE}
   \author{Yunzhong~Hou, Te~Wei,~\IEEEmembership{Student Member,~IEEE,}  Wei~Feng,~\IEEEmembership{Member,~IEEE,} \\ Yunfei~Chen,~\IEEEmembership{Senior Member,~IEEE,}  and Ning~Ge,~\IEEEmembership{Member,~IEEE}
    
   
    %E-mail: \{houyz14, wei-t14\}@mails.tsinghua.edu.cn, \{fengwei, gening\}@tsinghua.edu.cn, yunfei.chen@warwick.ac.uk%
   
   \thanks{Y. Hou, T. Wei, W. Feng, and N. Ge are with Tsinghua National Laboratory for Information Science and Technology, Beijing 100084, China. (E-mail: \{houyz14, wei-t14\}@mails.tsinghua.edu.cn, \{fengwei, gening\}@tsinghua.edu.cn.)}
   \thanks{Y. Chen is with School of Engineering, University of Warwick, Coventry CV4 7AL, U.K. (E-mail: yunfei.chen@warwick.ac.uk)}
   }
   %\address[1]{Tsinghua National Laboratory for Information Science and Technology, Tsinghua University, Beijing 100084, P. R. China}
   %\address[2]{School of Engineering, University of Warwick, Coventry CV4 7AL, U.K.}
   
   
   %\markboth
   %{Y. Hou \headeretal: Energy Efficient User Scheduling for Maritime Ship-to-Ship/Shore Communications}
   %{Y. Hou \headeretal: Energy Efficient User Scheduling for Maritime Ship-to-Ship/Shore Communications}
   
   %\corresp{Corresponding author: Wei Feng (e-mail: fengwei@tsinghua.edu.cn).}
   
    
   
   \maketitle
   
  \begin{abstract}
   
   %The maritime ship-to-shore communication system has to cover a vast area with a limited number of base stations (BSs) due to the geographical restriction of BS sites. Therefore, its energy consumption is usually much larger than terrestrial cellular networks. 
   %To reduce its energy consumption, ship-to-ship relay communication is introduced to the system since it allows ships to act like relays and enables direct communication between neighboring vessels. However, there are more transmitters to choose from in a ship-to-ship/shore system, which bring more challenges for user scheduling. 
   %As a crucial technique for saving energy, user scheduling depends strongly on the channel state information (CSI), which is difficult to acquire in maritime scenarios due to the time-varying channel fading. Note that in maritime scenarios, the channel is dominated by the large-scale CSI, which can be predicted through the position information of each vessel based on its specific shipping lane and timetable. Thus, we use the large-scale CSI instead of the complete CSI to avoid the heavy overhead. 
   %Based on predicted large-scale CSI, 
   In this paper, we investigate user scheduling for a maritime communication system with both ship-to-ship and ship-to-shore links. Different from previous studies, we consider a more practical scenario that only the slowly-varying large-scale channel state information (CSI) is known at the transmitter. The large-scale CSI can be predicted through the position information of each vessel based on its specific shipping lane and timetable. Thus, the total system overhead can be greatly reduced, which is quite crucial for maritime communications.
   On that basis, we formulate a user scheduling optimization problem, aiming to minimize the total energy consumption with guaranteed quality of service (QoS). 
   To solve the NP-hard problem, we develop a progressive algorithm, which only requires polynomial computational complexity. Simulation results demonstrate that the proposed scheme significantly reduces the energy consumption by up to 67\% over existing methods. 
   %To solve the problem, we simplify it in a progressive way. Accordingly, we develop a 3-step algorithm, which can solve the problem with polynomial computational complexity. Simulation results reveal that the proposed large-scale-CSI-based scheme significantly reduces the energy consumption by up to 67\% over existing methods.
   
   \end{abstract}
   
   \begin{IEEEkeywords}
   Maritime ship-to-ship/shore communication, user scheduling, large-scale channel state information (CSI), progressive algorithm
   %, greedy method
   % mobility enhanced,
   \end{IEEEkeywords}
  
   
   \section{Introduction}\label{sec:1}
  
   The rapid development of marine industries, as well as the economic and cultural exchanges between littoral states, have promoted the demand for reliable and high-speed maritime communication services. In recent years, several maritime communication network projects have been developed in order to meet this increasing demand, e.g., the BLUECOM+ project, the MarCom project, and the TRITON project \cite{p321}--\cite{p32}. 
   
   Unlike terrestrial cellular networks, a maritime ship-to-shore communication system has quite limited geographically available base station (BS) sites. Thus, maritime communication systems usually adopt high-powered BSs so as to cover a vast area with limited BSs. This high-powered BS strategy increases the operational costs of mobile network operators and poses a global threat to the environment \cite{p33}.
   Accordingly, reducing energy consumption becomes a critical issue for maritime communications. 
   To reduce system energy consumption, we introduce the idea of ship-to-ship communications, where vessels act like relays in the on-shore data distribution network. Several previous maritime communication network projects \cite{p321}--\cite{p32} have already included ship-to-ship communications to extend the coverage of the systems. Nevertheless, to the best of the authors' knowledge, the area still remains undiscovered where ship-to-ship relay transmissions are introduced to reduce system energy consumption. 
  
   Undoubtedly, the ship-to-ship transmission will reduce system energy consumption with direct relay transmission between neighboring vessels. 
   However, it introduces more transmitters (BS/relays) and brings more challenges for user scheduling. % (see Appendix A)
   User scheduling, as an important perspective for energy saving, depends heavily on channel state information (CSI). However, it is rather costly to acquire perfect CSI in maritime communication systems, due to the excessive system overhead including pilot overhead and feedback overhead \cite{p403}-\cite{p407}. %Most of the previous user scheduling schemes overlook the difficulties in obtaining full CSI in maritime scenarios. 
   
   %terrestrial
   So far, the majority of energy-efficient user scheduling techniques focused on terrestrial cellular networks. 
   Based on the utilization degree of CSI, terrestrial user scheduling schemes can be classified into three categories. The first one required no CSI, such as the simple but efficient round-robin scheme for fair queuing \cite{p51}. The second one exploited statistical and outdated CSI, as studied in \cite{p52} and \cite{p53}. The third one assumed full CSI, and utilized the instantaneous CSI for user scheduling on a minuscule time scale, i.e., in each coherence time \cite{p4}-\cite{p7}. 
   In maritime scenarios, however, obtaining perfect CSI for user scheduling brings large overheads and is not cost-effective.
   Moreover, since the large-scale CSI can be predicted in maritime scenarios (which will be explained later), statistical and outdated CSI is not the best choice, either. 
  
   %maritime
   As for maritime user scheduling with ship relays, limited works have been done. %concentrated on energy efficiency. 
   Both \cite{p300} and \cite{p301} focused on monitoring videos uploading via maritime communication networks. They focus on user scheduling of ship-to-shore communication with the store-carry-and-forward relay mechanism.  
   %Towards video packets store-carry-and-forward scheduling in maritime wideband communication
   %Efficient Scheduling for Video Transmissions in Maritime Wireless Communication Networks
   In \cite{p303}, a scheduling model was developed to provide the communication path of the fewest routing times to the moving ships that are far apart, which has reduced the space link resources consumption. 
   %Research on optimum cooperative relay model for moving targets based on ant colony algorithm
   Nevertheless, these works were also based on the assumption that the channel capacity is known, and did not consider the real-world challenges in acquiring full CSI. 
    
   
   
   %when mmwave communications meet network densification
   %问题-》帽子
   %For the following reasons, current studies on user scheduling require systematic redesign and should be based on large-scale CSI instead of full CSI. 
   
   %1.海域模型,大尺度主导
   %\textbf{1.} As there are fewer scatterers on the sea than that in the terrestrial scenario, the large-scale channel fading becomes the dominant factor for the maritime channel \cite{p403}. Hence, we can use the position information of vessels to predict large-scale CSI and avoid the heavy overhead for full CSI; 
   %when mmwave communications meet network densification
   
   %2.航线·位置信息
   %\textbf{2.} %Different from the random trails of human beings in terrestrial scenarios where the previous studies focused on,
   %Most vessels have specific fixed shipping-lanes and timetables that can be acquired beforehand, and their position information can be easily predicted. From the position information, we can predict large-scale CSI for the whole service duration. We can have further gain by considering the whole service process instead of the short timescale as in terrestrial scenarios. 
   
  
   In this paper, we take the advantage of maritime channel features to redesign user scheduling without full CSI. In general, the maritime channel has fewer scatterers, so the large-scale channel fading becomes the dominant factor for the maritime channel \cite{p403}. Moreover, most vessels have specific fixed shipping-lanes and timetables which can be used to acquire their position information. From the position information, we can predict large-scale CSI for the whole service duration. Particularly, we formulate a maritime user scheduling optimization problem based on large-scale CSI, aiming to minimize the system energy consumption. To overcome the difficulties in choosing transmitters brought forward by ship-to-ship transmission, we progressively simplify the optimization problem. We further design a 3-step algorithm for the approximate solution to the user scheduling problem. 
  
   \section{System Model and Problem Formulation}\label{sec:2}
   
   \begin{figure} [ht!]
   \begin{center}
   \includegraphics*[width=8.8cm]{slotted.png}
   \end{center}
   \vspace*{-4mm} 
   \caption{${K_0}\Delta \tau $ after the service begin, ship A and B are located as shown in (a). 
   In (a), (b) and (c), the BS transmit directly to B. Later, in (d) and (e), B relay the data to A. In the end, through relay transmission, ship A can meet QoS need while avoiding high energy consumption in directly communication to BS.}\label{fig:1}
   \vspace*{-4mm} 
   \end{figure}
  
   
   \subsection{System Parameters}
   
   As shown in Fig. 1, the following sections focus on the user scheduling of an frequency division multiple access (FDMA) downlink transmission system. There are both ship-to-ship and ship-to-shore links in the considered system. The system has one on-shore BS and $J$ single-antenna users that can either act like receivers or like transmitters (relays). In general, we denote the transmitters (BS/relays) as $i \in \left\{ {0,1,...,J} \right\}$, the receivers (users) as $j \in \left\{ {1,...,J} \right\}$. $i=0$ means the transmitter is BS, whereas $i>0$ means the transmitter is one of the $J$ ships that can act like relays. We assume that there are $N$ subcarriers, and the subcarrier bandwidth is ${B_s}$. $P_i = \left\{ {P_0,\left\{ {P_j} \right\}} \right\}$ represents the fixed transmit power of BS ($i=0$) or ship relays ($i>0$) on any given subcarrier. In this paper, we only consider two-hop half-duplex `ship-to-ship/shore communications' for simplicity. 
   
   In the considered system, ship-to-ship transmissions use the same licensed band of ship-to-shore transmissions (i.e. one of the $N$ subcarriers), and the same air interface of the ship-to-shore transmissions. 
   Based on the half-duplex assumption, the $J$ single-antenna users can either receive data from one transmitter or transmit data to another user at any given time. 
   Also, each transmission link uses distinct subcarrier. Here in this paper by `link' we mean the transmission from BS/relay to a receiver during a certain time period. 
   
   Without loss of generality, we assume the on-shore BS coverage shape to be a semicircle. 
   In order to simplify the problem, we only consider transmissions within the semicircle. We also assume that all the users request different data and the system has no data reuse. 
   Each user sails into and out of the semicircle according to its shipping lane and timetable. 
   Since maritime users focus more on data volume rather than transmission delay, we particularly focus on the delay-tolerant information distribution service, which is initiated and terminated when a marine user sails into and out of the BS's coverage, respectively. 
   The QoS constraint, i.e., the data volume required by the ${j^{th}}$ user, is denoted by $V_j^{\rm{QoS}}$. 
   
   \subsection{Large-Scale CSI}
   
   In terrestrial scenarios, according to the multipath effect, signals are well scattered and the small-scale fading factor has a significant impact on the channel. However, in maritime scenarios, due to the scarcity of scatterers, the large-scale fading factor becomes dominant. Therefore, we focus on large-scale CSI in this paper. 
   
   We assume a modified 2-ray propagation model for the maritime channel, since the sea surface is relatively flat \cite{p0}--\cite{p2}. For a given subcarrier, we denote the composite channel gain from the BS/relay $i$ to the user $j$ at time $\tau $ by $\sqrt {{\beta _{i,j,\tau }}} {h_{i,j,\tau }}$. The small-scale fading vectors ${h_{i,j,\tau }}$ follow a complex Gaussian distribution with standard deviation ${\sigma _s} = 1$, i.e., ${h_{i,j,\tau }} \sim \mathcal{CN}(0, \mathbf{I})$. The large-scale fading coefficient ${\beta _{i,j,\tau }}$ is expressed as
   \begin{align}
   {\beta _{i,j,\tau }} = {\left( {\frac{\lambda }{{4\pi {d_{i,j,\tau }}}}} \right)^2}{\left[ {2\sin \left( {\frac{{2\pi {h_t}{h_r}}}{{\lambda {d_{i,j,\tau }}}}} \right)} \right]^2} ,
   \end{align}
   where $\lambda $ is the carrier wavelength, ${d_{i,j,\tau }}$ is the distance between the BS/relay $i$ and the user $j$ at time $\tau $. The antenna height of the transmitter and the receiver are represented by $h_t$ and $h_r$, respectively. 
   
   To fully utilize the slowly-varying characteristic of the large-scale channel fading, we divide the total system service time into $T$ time slots, each lasting $\Delta \tau$. The value $\Delta \tau$ is carefully chosen so that $\beta _{i,j,\tau }$ remains constant in each time slot $t$ (ignoring ship movements during each time slot). Thus, we make it possible to acquire large-scale CSI $\beta _{i,j,t} = \mathbb{E} \left [ {\beta _{i,j,\tau }} \right ]$ for $\forall t \in \left\{ {1,...,T} \right\}$ from position information based on shipping-lanes and timetable. Any further denotation of CSI in this paper refers to the predicted `large-scale CSI' for the whole service duration unless specified. 
  
   To avoid the heavy overhead, we replace the perfect CSI with large-scale CSI and use predicted slowly-varying large-scale CSI to predict transmission speed (capacity) as shown in (2a)-(2c). We justify our replacement by simulations in Section IV. Let us denote ${\gamma _{i,j,t }} = {\raise0.7ex\hbox{${{P_{i} }{\beta _{i,j,t }}}$} \!\mathord{\left/
    {\vphantom {{{P_{i} }{\beta _{i,j,t }}} {{\sigma ^2}}}}\right.\kern-\nulldelimiterspace}
   \!\lower0.7ex\hbox{${{\sigma ^2}}$}}$ for simplicity, where ${P_{i}}$ represents the transmit power of BS/relay $i$. The transmission speed in this paper can therefore be simplified as
   \begin{subequations}
   \begin{align}
   {r_{i,j,t}} & = {{B_s}{{\log }_2}\left( {1 + \frac{{{P_{i} }{\beta _{i,j,t }}{{\left| {{h_{i,j,t }}} \right|}^2}}}{{{\sigma ^2}}}} \right)},\\
   & = {B_s}{\log }_2 \left( {1 + {\gamma _{i,j,t }}{{\left| {{h_{i,j,t }}} \right|}^2}} \right),\\
   & = {B_s}\left( {{{\log }_2}e} \right){e^{\frac{1}{{{\gamma _{i,j,t }}}}}}\int_1^\infty {\frac{1}{u}{e^{ - \frac{u}{{{\gamma _{i,j,t }}}}}}du} .
   \end{align}
   \end{subequations}
   The transmission speed in (2c) is derived based on previous study \cite{p41}. The impact of this replacement (assuming that ship $j$ stays in the same position and $\beta _{i,j,\tau }$ remains constant in each time slot $t$) is further discussed in Section IV. 
   %With the long-term large-scale channel fading (CSI) known beforehand, we can further design and implement a user scheduling scheme.
  
   For simplicity, we denote the link from transmitter $i \in \left\{ {0,1,...,J} \right\}$ (BS/relay, $i = 0$ means BS, $i > 0$ means relay) to receiver $j$ (user) at time slot $t$ by $i \to j@t$. Since we only consider two-hop link-pairs, a pair of the substitution ship-to-ship/shore links $\left[ {0 \to i'@{t_1},i' \to j@{t_2}} \right]$ consists of exact a ship-to-shore part $0 \to i'@{t_1}$ for BS to transmit data to relay ${i'}$, and a ship-to-ship part $i' \to j@{t_2}$ for relay ${i'}$ to transmit to receiver user $j$. 
   
   
   \subsection{Problem Formulation}
   
   The total energy consumption can be written as the sum of transmission energy to each user. 
   Therefore the energy consumption of the whole system is
   \begin{align}
    {E_{\rm{total}} = \sum\limits_{j = 1}^J {{E_j}} = \sum\limits_{j = 1}^J {\left( {\sum\limits_{i=0}^{J}{\sum\limits_{t = 1}^T {{P_{i}}\delta _{i,j,t}} } \Delta \tau  }\right)}}. 
    \end{align}
   By $\delta _{i,j,t} \in \left\{ {0,1} \right\}$ we denote whether there is a link $i \to j @ t$ and a subcarrier is scheduled for the link. $ {{\delta _{i,j,t}}} = 0$ means there is no transmission from BS/relay $i$ to user $j$ at time slot $t$, while $ {{\delta _{i,j,t}}} = 1$ means there is a transmission link $i \to j @ t$ on a certain subcarrier. Moreover, $\forall j,\forall t,{\delta _{j,j,t}} \equiv 0$, since we do not allow receiving transmissions from oneself. 
   Our objective is to minimize the system energy consumption by means of user scheduling in the ship-to-ship/shore communication system. 
   
   We denote the total data volume that user $j$ currently has at time slot $t$ by ${V_{j,t}}$. ${V_{j,t}}$ can be written as the sum of the received data volume in each time slot minus the sum of relayed data volume in each time slot. 
   \begin{align}
    {V_{j,t}} = \sum\limits_{\tau = t_j^{{\rm{In}}}}^t {\left( {\sum\limits_i {{r_{i,j,\tau }\delta _{i,j,\tau}}} - \sum\limits_{j'} {{r_{j,j',\tau }\delta _{j,j',\tau}}} } \right) \Delta \tau } .
   \end{align}
   The time slots when user $j$ enters and leaves the BS's coverage are denoted by $t_j^{{\rm{In}}}$, $t_j^{{\rm{Out}}}$, respectively. In the considered system, based on delay-tolerant assumption, $t_j^{{\rm{In}}}$, $t_j^{{\rm{Out}}}$ equal the service beginning time slot and ending time slot for user $j$. 
   Since the system has no data reuse, user relay $j$ must have enough data ${V_{j,t}}$ to relay and transmit, i.e., ${V_{j,t} \ge 0}$. 
  
   Although different users require different data in our system, it is possible to avoid keeping track of what BS/relays transmit to each user at any given time. To do this, we have to record all links we have chosen in a link set ${\mathbf{S}}$. ${\mathbf{S}}$ can be acquired before ships enter the BS coverage based on our proposed algorithm. The elements in ${\mathbf{S}}$ can be either ship-to-shore-only link $\left\{0 \to i'@{t_1}\right\}$ or ship-to-ship/shore link-pair $\left\{\left[ {0 \to i'@{t_1},i' \to j@{t_2}} \right]\right\}$. 
   With the link set known beforehand, we can get the transmission speed of each link in the set, and determine what and how much to transmit on each link at any given time. Thus we can avoid keeping track of what BS/relays transmit to each user at any given time slot. 
   
   Thus, we formulate the energy consumption optimization problem as
   \begin{subequations}
   \begin{align}
   & \mathop {\min }\limits_{{\left\{ {{\delta _{i,j,t}}} \right\}^{\left( {J + 1} \right) \times J \times T}}}  {\sum\limits_{i = 0}^J {\sum\limits_{j = 1}^J {\sum\limits_{t = 1}^{T} {{P_{i}}\delta _{i,j,t} \Delta \tau } } } }  ,\\
    {s.t.} \;\; &\sum\limits_{i \ne j} {{{\delta _{i,j,t}}} } + \sum\limits_{j' \ne j} { {{\delta _{j,j',t}}} \le {1}} ,\\
    \;\;\;\;\;\; &\sum\limits_i {\sum\limits_j { {{\delta _{i,j,t}}} } } \le N ,\\
    \;\;\;\;\;\; &{{{\left. {{V_{j,t}}} \right|}_{t = t_j^{{\rm{In}}}}} = 0, \left. {{V_{j,t}}} \right|_{t = t_j^{{\rm{Out}}}}} \ge V_j^{\rm{QoS}}, {V_{j,t}} \ge 0.
   \end{align}
   \end{subequations}
   We have to consider transmissions from ${J + 1}$ transmitters (BS/relays) to $J$ receivers (users) at $T$ time slots in problem (5). 
   %, and our optimization is in a $\left( {J + 1} \right) \times J \times T$ 3-dimensional subspace. 
   Half-duplex constraint (5b) guarantees that each user has access to at most one BS/user at a given time, and serves either as a transmitter or as a receiver. The constraint in (5c) guarantees that at most $N$ users can be severed simultaneously in the system, by BS or relays, since there are only $N$ subcarriers. (5d) makes sure that the QoS constraint is met and relays cannot transmit more than what they have currently.
   
   %\textit{Theorem 1:} The problem in (4) is NP-hard.
   
   %\textit{Proof:} See Appendix A. 
   
   \section{User Scheduling for Maritime Ship-to-Ship/Shore Communication}\label{sec:3}
   
   In this section, we focus on the user scheduling problem in (5), which reduces system energy consumption while ensuring QoS. After the analysis of the original NP-hard problem in (5), we give an optimal solution to a simplified problem and designed a dedicated utility function for the greedy substitution afterward. Then, we propose a progressive algorithm that can return an approximate solution to the optimization problem in (5) with polynomial time complexity. 
   
   \subsection{Analysis of the Original Problem}
   
   The original problem in (5) is an Integer Programming problem, and it is NP-hard. %\cite{p1000}. 
   Thus, achieving the optimal solution for it is not practical (exponential complexity). To avoid the difficulties, we first loosen some constraints in (5) and get an optimal solution for this simplified problem. Then based on the optimal solution to this simplified problem, we gradually add back the constraints to approximate the solution to the original problem. Through the greedy approximation, we use a dedicated utility function focused on minimizing the energy consumption. 
   
   Since the original problem involves various factors, and the factors are correlated, we have to endure the exponential complexity in searching for the optimal solution. However, if we loosen the constraint in (5b) and (5c), and only consider $i=0$ rather than $i \in \left\{ {0,1,...,J} \right\}$, then simplified problem (6) no-longer have correlated factors. 
   \begin{subequations}
    \begin{align}
    & \mathop {\min }\limits_{\left\{ {{\delta _{0,j,t}}} \right\}}  {\sum\limits_{j = 1}^J {\sum\limits_{t = 1}^{T} {\delta _{0,j,t} P_{0} \Delta \tau } } }  ,\\
     {s.t.} \;\; &{{{\left. {{V_{j,t}}} \right|}_{t = t_j^{{\rm{In}}}}} = 0,\left. { {V_{j,t}}} \right|_{t = t_j^{{\rm{Out}}}}} \ge V_j^{\rm{QoS}}.
    \end{align}
   \end{subequations}
   In this case, since the optimization variables of different users are no longer correlated, the optimal solution here can be obtained by choosing $\delta _{0,j,t}$ for each $j$ separately. Note that the equation (4) now can be re-written as ${V_{j,t}=\sum\limits_{\tau = t_j^{{\rm{In}}}}^t {r_{0,j,t} \delta _{0,j,t}}\Delta \tau}$. The objective function in (6a), on the other hand, is proportional to ${\sum \limits_{j = 1}^J {\sum\limits_{t = 1}^{T} {\delta _{0,j,t} }}}$. Therefore we can obtain the optimal solution for (6) by choosing fewest links. 
   We further define ${{\mathbf{S}}_{\mathbf{1}}}$ as the set of chosen links in problem (6), i.e., $\left( {0,j,t} \right) \in {\mathbf{S}}_{\mathbf{1}}$ if ${\delta _{i,j,t} = 1}$. 
   
   The optimal solution to problem (6) can be written as $E_{\rm{total}}^{\min}=\sum\limits_{j = 1}^J \mathop {\min }\limits_{{\left\{ {{\delta _{0,j,t}}} \right\}}} {\sum\limits_{t = 1}^T {{P_{0}}\delta _{0,j,t} \Delta \tau } } $. 
  
   
  
   Next, we add back the constraint (5b) and consider $i \in \left\{ {0,1,...,J} \right\}$. $V_{j,t}$ has to be calculated according to (4) due to the multiple choice of $i$. 
  
   Since we now allow more choice for $i$ in $\delta _{i,j,t}$, a smaller $E_{\rm{total}}$ is expected. Denote the result we get here as ${{\mathbf{S}}_{\mathbf{2}}}$. Therefore, we can approximate the original objective (5a) by maximizing the energy consumption reduction between ${{\mathbf{S}}_{\mathbf{2}}}$ and ${{\mathbf{S}}_{\mathbf{1}}}$. 
   %This time, we greedily change part of the ship-to-shore links in ${{\mathbf{S}}_{\mathbf{1}}}$ into fewer ship-to-ship/shore link-pairs for lower system energy consumption. 
   %Since the introduction of ship-to-ship links brings forward energy reduction by reducing the number of ship-to-shore links, 
   \begin{subequations}
   \begin{align}
    &\mathop {\max }\limits_{{\left\{ {{\delta _{i,j,t}}} \right\}}}  {{\sum\limits_{j = 1}^J \sum\limits_{t = 1}^{T}{\left( {\mathop {{P_{0}}\delta _{0,j,t} }\limits_{\left( {0,j,t} \right) \in {{\mathbf{S}}_{\mathbf{1}}}} - \sum\limits_{i = 0}^J {\mathop {{P_{i}}\delta _{i,j,t}}\limits_{\left( {i,j,t} \right) \in {{\mathbf{S}}_{\mathbf{2}}}} } } \right) \Delta \tau } } } ,\\
    {s.t.} \;\; &\sum\limits_{i \ne j} {{{\delta _{i,j,t}}} } + \sum\limits_{j' \ne j} { {{\delta _{j,j',t}}} \le {1}} ,\\
    \;\;\;\;\;\; &{{{\left. {{V_{j,t}}} \right|}_{t = t_j^{{\rm{In}}}}} = 0, \left. {{V_{j,t}}} \right|_{t = t_j^{{\rm{Out}}}}} \ge V_j^{\rm{QoS}}, {V_{j,t}} \ge 0.
  \end{align}
  \end{subequations}
   Noted that, with the re-introduction of the constraints, the problem (7) becomes an integer programming problem and is NP-hard. We try to avoid the exponential complexity for the optimal solution and settled on the approximation provided by a greedy substitution based on the optimal solution to problem (6).  
  

  At last, we add back the constraint in (5c) to complete our approximation to the original problem in (5). 
  %The solution ${{\mathbf{S}}_{\mathbf{1}}}$ returned by step1 is not a practical one for the ship-to-shore system in (5a)-(5c) since (4b) has not been taken into account. We design an effective method to iteratively get the approximate solution ${{\mathbf{S}}_{\mathbf{2}}}$ which is applicable for the ship-to-shore system. 
  
  Since the re-introduction of the constraint (5c) brings forward more limitation, we have to sacrifice some performance to meet all the constraints. 
  We minimize the energy consumption increase term in (8a) for a minimal sacrifice in performance. 
  \begin{subequations}
    \begin{align}
      &\mathop {\min }\limits_{\left\{ {{\delta _{i,j,t}}} \right\}}  {\sum\limits_{i = 0}^J {\sum\limits_{j = 1}^J {\sum\limits_{t = 1}^T {\left( {\mathop {{\delta _{i,j,t}}}\limits_{\left( {i,j,t} \right) \in {{\bf{S}}_{\bf{3}}}}  - \mathop {{\delta _{i,j,t}}}\limits_{\left( {i,j,t} \right) \in {{\bf{S}}_{\bf{2}}}} } \right){P_i}\Delta \tau } } } }  ,\\
      {s.t.} \;\; &\sum\limits_{i \ne j} {{{\delta _{i,j,t}}} } + \sum\limits_{j' \ne j} { {{\delta _{j,j',t}}} \le {1}} ,\\
      \;\;\;\;\;\; &\sum\limits_i {\sum\limits_j { {{\delta _{i,j,t}}} } } \le N ,\\
      \;\;\;\;\;\; &{{{\left. {{V_{j,t}}} \right|}_{t = t_j^{{\rm{In}}}}} = 0, \left. {{V_{j,t}}} \right|_{t = t_j^{{\rm{Out}}}}} \ge V_j^{\rm{QoS}}, {V_{j,t}} \ge 0.
    \end{align}
  \end{subequations}
  Since the optimal solution for the NP-hard problem in (8) requires polynomial complexity, we greedily approximate the optimal solution for the ship-to-ship/shore system by minimizing the energy consumption increase between ${{\mathbf{S}}_{\mathbf{2}}}$ and the final result ${{\mathbf{S}}_{\mathbf{3}}}$. %The utility function in (8) is also used here to decide which link to replace in our greedy method. 
  
  By solving the problems (6) - (8) in sequence, each based on its predecessor, we complete the approximation to the original problem in (5). Moreover, we can avoid the exponential complexity for the optimal solution by greedy substitution and approximation. 

   
    
  %%%%%%%%%%%%%%%%%%%%%%%%%%%%%%%%%%%%%%%%%%%%%%%%%%
  
   \subsection{Proposed Progressive Algorithm}
   
   %In stage-1, we simply consider the ship-to-shore transmission and ignore the subcarrier constraint, so as to focus solely on the simple scheduling based on large-scale CSI that we can predict. We can get an optimal result for this simple problem by choosing links with best CSI. 
   
   %In stage-2, we consider the maritime ship-to-ship/shore communication system. We substitute part of the ship-to-shore links we get in stage-1 for two-hop ship-to-ship/shore link-pairs $\left[ {0 \to i'@{t_1},i' \to j@{t_2}} \right]$ for less energy consumption. Since the introduction of relays brings forward great difficulties in user scheduling, we use a greedy method in stage-2. %We consider the subcarrier constraint here or otherwise, the number of links in the same time slot may further exceed the subcarrier constraint due to the introduction of ship-to-ship links. 
   
   %Last, in stage-3, we use a greedy algorithm based on the result returned by stage-2 to make sure that our user scheduling is applicable, i.e., the subcarrier constraint is met. Since this constraint has not been considered in stage-1, we make adjustments in stage-3 to get an approximation of the applicable solution for the ship-to-shore system. 
  
   %Eventually, after all, three steps, we approximate the optimal solution for the original problem in (5). Since the transmission speed for all $\left( J+1 \right) \times J \times T$ links can be predicted based on large-scale CSI, the 3-step algorithm can be carried out using only large-scale CSI. In Section IV we prove the validity of our progressive algorithm by comparing the energy consumptions. 
   
   
   %For stage-1, we concentrate on large-scale CSI by considering the most simple scenario: a ship-to-shore-only system without subcarrier constraint. We fix transmitter $i = 0$ since users can only receive data from on-shore BS. 
   
   %We only optimize $\left\{ {{\delta _{0,j,t}}} \right\}^{J \times T}$ since in stage-1, there is only one transmitter. 
   %since the optimization is currently in a $J \times T$ 2-dimensional subspace (the transmitter dimension degenerates since there is only one transmitter, namely BS) 
   %Half-duplex constraint in (5b) and ${V_{j,t}} \ge 0$ in (5d) are not necessary here since users can only receive data from BS, and cannot act like relays. We also drop the $N$-subcarrier constraint in (5c) since we assume that the BS can serve infinite number of users. 
   
   In this subsection, we design a progressive algorithm for the original problem in (5). Based on the optimal solution for problem (6), we use a dedicated utility function to complete the greedy substitution for the problems (7) and (8). 
  
   First, we get an optimal solution for problem (6) by finding link $0 \to j@t$ with best ${r _{0,j,t}}$ and set ${\delta _{0,j,t} = 1}$ for each user until the QoS constraint (6b) is met. 
   
   
   Next, we use a greedy method to approximate the optimal solution in problem (7). 

   Aiming to maximize the system energy reduction in (7a) through greedy substitution, we propose a utility function for any given link (or link pair), which can indicate the performance gain form the substitution. During the greedy substitution process, we find link-pairs with the highest utility function values to substitute the original links with lowest utility function values. The utility function can be written as the reciprocal of composite power-to-rate ratio. 
   \begin{subnumcases}
    {}
    \frac{1}{U\left(0 \to j @ {t_0}\right)} = \frac{P_0}{r_{0,j,{t_0}}},\\
    \frac{1}{U\left(0 \to i'@{t_1}, i'\to j@{t_2} \right)} = {\frac{P_0}{r_{0,i',{t_1}}} + \frac{P_{i'}}{r_{i',j,{t_2}}}} . 
   \end{subnumcases}
   A higher utility function value means a lower composite power-to-rate ratio, which implies the link (link-pair) takes less energy to achieve the same transmission speed. If a link-pair has higher utility function value, i.e., $ U\left(0 \to i'@{t_1}, i'\to j@{t_2} \right) > U\left(0 \to j @ {t_0}\right)$, then we substitute the original link into this ship-to-ship/shore link-pair (one link from BS to relay, the other from relay to user). We do this to maximize the difference term (7a) since 
   \begin{subequations}
   \begin{align}
    \Delta E  & =  \left[ {P_0}- \left( {\frac{{{P_0}}}{{{r_{0,i',{t_1}}}}} + \frac{{{P_{i'}}}}{{{r_{i',j,{t_2}}}}}} \right){r_{0,j,{t_0}}} \right] \Delta \tau,\\
    & = \frac{{r_{0,j,{t_0}}}\Delta \tau}{U\left(0 \to j @ {t_0}\right)} - \frac{{r_{0,j,{t_0}}}\Delta \tau}{U\left(0 \to i'@{t_1}, i'\to j@{t_2} \right)}  .
   \end{align}
   \end{subequations}
  % In addition, we introduce $I\left[ x \right]$, the indicative function where $I\left[ x \right] = \left\{ \begin{array}{l}
  %   1,x > 0,\\
  %   0,\rm{else}.
  %   \end{array} \right.$
   As we can see, by substituting the original link with link-pair with highest utility function value, we can maximize the energy reduction term in (7a). Thus, we can design a greedy substitution method with the help of the utility function in (9) for problem (7). 
  

   Since the two parts of the link pair may have different transmission speed, i.e., $r_{0,i',{t_1}} \ne r_{i',j,{t_2}}$, we denote the actual transmission speeds used in the link-pair by 
   \begin{align}
   &r_{0,i',{t_1}}^{\rm{desired}}=r_{i',j,{t_2}}^{\rm{desired}}=\min \left\{ r_{0,i',{t_1}},r_{i',j,{t_2}} \right\}. 
   \end{align}
   We also used $r_{i,j,{t}}^{\rm{desired}}$ instead of $r_{i,j,{t}}$ in (4) to update the data volume any user has at given time slot. 
   The remaining available transmission speed for any link $i \to j @ t$ is expressed as
   \begin{align}
    &r_{i,j,{t}}^{\left( k+1 \right),\rm{remain}} = r_{i,j,{t}}^{\left( k \right),\rm{remain}} - r_{i,j,{t}}^{\rm{desired}}, 
   \end{align}
   where $\left( k \right)$ denotes the $k^{th}$ update for link $i \to j @ t$. 
   
   The algorithm is carried out as follows.
   
   For each user $j$, we first record all plausible ship-to-ship/shore link-pairs like $\left[ {0 \to i'@{t_1},i' \to j@{t_2}} \right]$ in a temporary set $\mathbf{R}$. Here `plausible' means that the constraint in (7b) 
   %and the $N$-subcarrier constraint in (5c) are 
   is satisfied, plus both parts of the link-pairs have higher transmission speed than the original links. Further exploration will be conducted in the plausible ship-to-ship/shore link-pair set $\mathbf{R}$. We assume that in each ship-to-ship/shore link-pair, the BS to relay part and relay to user part transmit the same volume of data.
  
   Once we have the substitution set $\mathbf{R}$, while the constraint in (7b) and ${V_{j,t}} \ge 0$ in (7c) are met%(which are guaranteed by (11) - (13) as showed later)
   , we add ship-to-ship/shore link-pairs with highest utility function value $U\left(0 \to i'@{t_1}, i'\to j@{t_2} \right)$ to ${{\mathbf{S}}_{\mathbf{2}}}$. Preparing for substitution, we try to meet the QoS constraint in (7c) only with ship-to-ship/shore link-pairs, i.e., ${\left. {{V_{j,t}}} \right|_{t = t_j^{{\rm{Out}}}}} \ge 2{V^{\rm{QoS}}}$ (we may substitute all the original links and only have ship-to-ship/shore link-pairs for user $j$). 
   Continue those steps until there is no plausible link-pair to user $j$ in $\mathbf{R}$. %or there is no gain energy-wise from substitution, i.e., there are no ship-to-ship/shore link-pairs that have lower composite power-to-rate ratio $ {\frac{{{P_0}}}{{{r_{0,i',{t_1}}}}} + \frac{{{P_{i'}}}}{{{r_{i',j,{t_2}}}}}} $ than the original links. 
  
   After this, we complete our substitution by removing elements in ${{\mathbf{S}}_{\mathbf{2}}}$ with relatively utility function value as long as the QoS constraint is met. If the element we removed is an original link, then it means we substitute it with link-pairs. If the element we removed is a link-pair, then it suggests that this link-pair is no better than the original link, and we do not substitute. 
  
   
   \begin{algorithm}[ht]
   \caption{Algorithm for solving the problem in (7)}
   \begin{algorithmic}[1]
   \STATE Initialize ${{\mathbf{S}}_{\mathbf{2}}}={{\mathbf{S}}_{\mathbf{1}}}$
   \STATE Initialize ${\mathbf{R}} = \phi $ as group for all plausible ship-to-ship/shore link-pairs.
   \STATE Find all ship-to-ship/shore link-pairs combination that have higher transmission speed than the original ship-to-shore-only links. Store them in ${\mathbf{R}}$.
   \FOR{all user $j$}
    \WHILE{${\left. {{V_{j,t}}} \right|_{t = t_j^{{\rm{Out}}}}} < 2{V^{\rm{QoS}}}$}
     \IF{there is no relay link with $j$ as target in  ${\mathbf{R}}$ }
      \STATE Break.
     \ENDIF
     \STATE Find ship-to-ship/shore link-pair in ${\mathbf{R}}$ with highest utility function value. 
     \STATE Update $r_{0,i',{t_1}}^{\rm{desired}}$, $r_{i',j,{t_2}}^{\rm{desired}}$.
     \IF{`${V_{i',{t_2}}}$ is ENOUGH' AND `${i'}$ \& $j$ are FREE at ${t_2}$' AND `${i'}$ is FREE at ${t_1}$'}
      \STATE Add them from ${\mathbf{R}}$ to ${{\mathbf{S}}_{\mathbf{2}}}$.
      \STATE Update $r_{0,i',{t_1}}^{\rm{remain}}$, $r_{i',j,{t_2}}^{\rm{remain}}$.
     \ELSE
     \STATE Drop them from ${\mathbf{R}}$. 
     \ENDIF
     \ENDWHILE
   \ENDFOR
  
   \WHILE{${\left. {{V_{j,t}}} \right|_{t = t_j^{{\rm{Out}}}}} \ge {V^{\rm{QoS}}}$}
     \STATE Find element in ${{\mathbf{S}}_{\mathbf{2}}}$ with lowest utility function value. 
     \STATE Remove it from ${{\mathbf{S}}_{\mathbf{2}}}$ to complete the substitution. 
     
   \ENDWHILE
   \end{algorithmic}
   \end{algorithm}
   
   
   In order to make sure that system constraint ${V_{j,t}} \ge 0$ in (7c) is met, we have ``${V_{i',{t_2}}}$ is ENOUGH'' in the algorithm, which means 
   \begin{subnumcases}
   {}%\begin{align}
   {V_{j',{t_2} - 1}} \ge r_0^{\min }\Delta \tau ,{\text{if}}\;{t_1} > {t_2},\\
   {V_{j',{t_2} - 1}} + r_{0,j',{t_1}}^{\rm{desired}}\Delta \tau \ge r_0^{\min }\Delta \tau ,{\text{else}}.
   %\end{align}
   \end{subnumcases}
   In the above algorithm, ``${i'}$ \& $j$ are FREE at ${t_2}$'' means that
   \begin{subnumcases}
   {}
   \sum\limits_{{i^*} \ne j} { {\delta _{{i^*},j,{t_2}}}} + \sum\limits_{{j^*} \ne j} {{\delta _{j,{j^*},{t_2}}}} - \delta_{i',j,{t_2}} =0,\\
   \sum\limits_{{i^*} \ne i'} {{\delta _{{i^*},i',{t_2}}}} + \sum\limits_{{j^*} \ne i'} {{\delta _{i',{j^*},{t_2}}}} -\delta_{i',j,{t_2}} =0,\\
   r_{i',j,{t_2}}^{\rm{remain}} \ge r_{i',j,{t_2}}^{\rm{desired}}.
   \end{subnumcases}
   And ``${i'}$ is FREE at ${t_1}$'' means that
   \begin{subnumcases}
   {}
   {\sum\limits_{{i^*} \ne i'} {{\delta _{{i^*},i',{t_1}}}} + \sum\limits_{{j^*} \ne i'} {{\delta _{i',{j^*},{t_1}}} - \delta_{0,i',{t_1}}=0}},\\
   r_{0,i',{t_1}}^{\rm{remain}} \ge r_{0,i',{t_1}}^{\rm{desired}}.
   \end{subnumcases}
   Through (14) and (15) the constraint (7b) and ${V_{j,t}} \ge 0$ in (7c) are met for the ship-to-ship/shore substitution link-pairs. 
  
  
   
   Last, we propose the following algorithm.
   
   If the constraint (8c) is not satisfied in time slot ${t}$, we find the element in ${{\mathbf{S}}_{\mathbf{2}}}$ that have the lowest utility function value. We drop those elements out of ${{\mathbf{S}}_{\mathbf{2}}}$ and find substitution to satisfy the QoS need under the constraint in (8c) with minimal energy addition. The substitution process is similar to the algorithm for solving the problem in (7), thus it is omitted here. 
   
   \begin{algorithm}[ht]
   \caption{Algorithm for solving the problem in (10)}
   \begin{algorithmic}[1]
   \STATE Initialize ${{\mathbf{S}}_{\mathbf{3}}}={{\mathbf{S}}_{\mathbf{2}}}$
   \FOR{ all $t$}
    \IF{$\sum\limits_j {{\eta _{i,j,t}}} \le N$ not met}
     \STATE Find element in ${{\mathbf{S}}_{\mathbf{3}}}$ that have the lowest utility function value. 
     \STATE Remove it from ${{\mathbf{S}}_{\mathbf{3}}}$.
     \WHILE {${{V_{j,T}} \ge {V_{j,QoS}}}$ not met}
      \STATE Find ship-to-shore-only link or ship-to-ship/shore link-pair (in the remaining ${\mathbf{R}}$) with highest utility function value. 
      \STATE Add it/them to ${{\mathbf{S}}_{\mathbf{3}}}$ (similar to the algorithm for solving the problem in (7)). 
     \ENDWHILE
    \ENDIF
   \ENDFOR
   \end{algorithmic}
   \end{algorithm}

   Based on the optimal solution to problem in (6), by carrying out the algorithm for solving the problems in (7) and (8), we complete the progressive algorithm for the approximation of the original problem in (5).
  
  
   \subsection{Complexity Analysis}
  
   %First we get the optimal solution to the simplified problem, and then we gradually add back the constraints and use a dedicated utility function to help us finish our substitution. 
   The progressive algorithm we proposed has polynomial complexity as shown below. 
   
   When carrying out the progressive algorithm, the transmission speed (capacity) is pre-calculated based on predicted large-scale CSI. Since we have to consider links from $J+1$ transmitters to $J$ receivers in $T$ time slots, it takes $\left( {J + 1} \right) \times J \times T$ calculations. The time complexity here is $O\left[ {\left( {J + 1} \right)JT} \right] = O\left( {{J^2}T} \right)$. 
   %Actually, since we only consider the transmissions for the users within the BS coverage, the calculations can be reduced to $O\left[ \left( {J + 1} \right) \times J \times \max \left\{ {t_j^{{\rm{Out}}} - t_j^{{\rm{In}}}} \right\} \right]$ complexity.
  
   First, to achieve the optimal solution to problem (6), we have to go through all $J$ users. For each user $j \in \left\{ {1,...,J} \right\}$, it takes $T$ iterations worst case to get the optimal solution for (6). Finding a ship-to-shore-only link with highest transmission speed is an $O\left( \log \left( T \right) \right)$ operation, whereas adding a link to ${{\mathbf{S}}_{\mathbf{1}}}$ is a cheap $O\left( 1 \right)$ operation. The total complexity is $O\left( J T\log \left( T \right) \right)$.
  
   To complete the approximation for problem (7), first, we have to acquire the `plausible' ship-to-ship/shore link set ${\mathbf{R}}$. %(ship-to-ship/shore link-pairs in ${\mathbf{R}}$ have higher transmission speed than the optimal ship-to-shore-only links). %During this process, we store the composite power-to-rate ratio for the links, which is a $O\left( 1 \right)$ operation. 
   Then, we go through all $J$ users, adding ship-to-ship/shore link-pairs trying to meet the QoS constraint only with ship-to-ship/shore link-pairs. There are at most $J \times T^2$ links in ${\mathbf{R}}$ for each user. Hence, finding the ship-to-ship/shore link with highest utility function value for user $j$ is an $O\left( \log \left( JT^2 \right) \right)$ operation. 
   After the addition of ship-ship/shore link pairs there are at most $J \times T$ link-pairs (since the link-pairs have higher transmission speed than the original ship-to-shore-only links) and 
   $J \times T$ ship-to-shore-only links. 
   %At last, we go through all ${2J}T$ elements ($JT$ from link pairs and $JT$ from original links) in ${{\mathbf{S}}_{\mathbf{2}}}$. 
   Among all ${2J}T$ elements we find element with lowest utility function value, and remove them from ${{\mathbf{S}}_{\mathbf{2}}}$. After the removal there are at most $\frac{{{P_0}}}{{{P_0} + \overline {{P_i}} }} { J} T$ links in ${{\mathbf{S}}_{\mathbf{2}}}$ (the energy consumption of ${{\mathbf{S}}_{\mathbf{2}}}$ is lower than that of ${{\mathbf{S}}_{\mathbf{1}}}$). 
   The removal of links has an $O\left[ \frac{{{P_0}}}{{{P_0} + \overline {{P_i}} }} { J} T \log \left( { {2J} T} \right)  \right]$ complexity. 
   Thus, the overall complexity is $O\left[ {JT \log \left( JT^2 \right)} \right]$. 
  
   Last, to complete the approximation for problem (8), there are at most $ J \times T$  links inherit from ${{\mathbf{S}}_{\mathbf{2}}}$. Thus, there are at most $J \times T$ links to remove from ${{\mathbf{S}}_{\mathbf{3}}}$ to satisfy the subcarrier constraint. This has an $O\left( JT\log \left( JT \right) \right)$ worst case complexity. Similar to the prior part, adding links so that the QoS constraint is satisfied takes at most $O\left( NT\log \left( JT^2 \right) \right)$ operations since there are at most $NT$ links in the end. Thus, the time complexity is $O\left[ JT \log \left( JT \right) + NT \log \left( {JT^2} \right) \right]$.
  
   In total, the overall time complexity of the 3-step algorithm is 
   \begin{align}
    &  O\left[  J^2 T  + { \left({N+J}\right) T \log \left( JT^2 \right)} \right].
   \end{align}
   As can be seen from the above analysis, the pre-processing of the transmission speed and proposed 3-step algorithm all have polynomial complexity. The total complexity is also polynomial, which is much smaller than the $O\left[ 2^{\left( J+1 \right)JT} \right]$ exponential complexity for the optimal solution. 
   
   \section{Simulation Results}\label{sec:4}
   
   In this section, we provide numerical results for the proposed ship-to-ship/shore 3-step progressive method, as well as a referential round-robin method. 
   For the referential ship-to-shore round-robin method, we have zero information about CSI, thus we find up to $N$ ships in a round-robin method in each time slot. 
   
   
   
   As for the system settings, the on-shore BS is located in the center of the plane and have a semicircular coverage shape in the sea, while the ships traverse along two intersecting shipping-lanes. 
   % since we focus on passenger ships scenarios in this study. Moreover, passenger ship assumption suits our study since their shipping-lanes are fixed and their position information can be easily determined. 
   Ships (users) leave the harbors every 15 minutes, and all sail at the speed of 36km/h. The time slot duration here is $\Delta \tau = 60$s. The QoS constraint is 1Gbits/ship if not specified. We assume that the system uses a carrier frequency of 1.9GHz, and has 32 subcarriers, each having a bandwidth of 2MHz. The on-shore BS's transmit power is 10W on any subcarrier, whereas the vessels' relay transmit power is 1W on any subcarrier since they are arguably smaller in size. The antenna height of the BS and the ships are 100m and 10m, respectively. The power density of the additive white Gaussian noise is ${-140}$dBm/Hz. 
   
   Of all the following simulations in Fig. 2 - 4, %the `round-robin' method refers to the ship-to-shore-only method based on zero CSI. 
   %The legend `large-scale CSI' means the algorithms carried out based on large-scale CSI that can be practically obtained through shipping-lanes and timetables. 
   the legend `(genius-aided) full CSI' actually means the assumption that we can know full CSI for the whole service duration in advance. %, which is impossible
   They are brought into the following simulations to examine the feasibility of our large-scale CSI method. 
   
   
   As shown later in simulations, having full CSI indeed will be most feasible. In ship-to-ship/shore systems, the difference in system energy consumption between large-scale CSI replacement and the genius-aided `full CSI' is around 5\% - 15\%. These errors here come from assuming that ship $j$ stays in the same position and $\beta _{i,j,\tau }$ remains constant during each time slot $t$, as well as not knowing the full CSI.
   This 5\% - 15\% error in ship-to-ship/shore systems shows that the large-scale CSI replacement in (2c) is quite acceptable. 
    
   
   
   First, in Fig. 2, we study the impact of only considering a part of the service duration in user scheduling. 
   Next, we investigate the impact of different service needs in Fig. 3. 
   Last, in Fig. 4, we investigate the impact of different transmit powers, since our algorithm mainly focuses on user scheduling with fixed transmit power on given subcarrier. 
   
   
   \begin{figure} [htb]
   \begin{center}
   \includegraphics*[width=8.8cm]{Tranges.eps}
   \end{center}
   \vspace*{-4mm} 
   \caption{System performance (energy consumption) when only considering a part of the system service duration in user scheduling algorithm} \label{fig:2}
   \vspace*{-2mm} 
   \end{figure}
   
   %\Fig.[t!](topskip=0pt, botskip=0pt, midskip=0pt){Tranges.png}
   %{Average energy consumption per user $E_{avg}$ versus the percentage of pre-acquired CSI.\label{fig4}}
   
   Fig. 2 demonstrates the relationship between average energy consumption and the ratio of ${T^{\rm{US}}}$ to total service time duration. Here ${T^{\rm{US}}}$ represents the user scheduling time duration, i.e., we only consider at most $T^{\rm{US}}$ time slots in user scheduling. 
   
   As we can see, the genius-aided full CSI curves are better than our large-scale CSI replacement energy-wise generally. Our proposed ship-to-ship/shore method outmatches the referential method, especially in ideal conditions (which means we consider full-service duration, ${T^{\rm{US}}}=T$). 
   %Since we can endure a slightly larger algorithm complexity (resulting from larger $T^{\rm{US}}$), we can increase  $T^{\rm{US}}$ in exchange for better performance energy-wise. 
   The long-term (larger $T^{\rm{US}}$) large-scale CSI result we get is better than the shorter-term (smaller $T^{\rm{US}}$) genius-aided full CSI result, which justifies our large-scale replacement. 
   
   When ${T^{\rm{US}}}$ approximates the total service time duration, we have maximum benefit from user scheduling. The large-scale CSI ship-to-ship/shore method consummates 60\% less energy than the round-robin method. 
   %Under simulation settings, due to geographical closeness and better CSI, users tend to finish their communications with BS or relays during the first 70\% of their staying time inside the service zone (semicircle). The improvement is steady until the ratio of ${T^{\rm{US}}}$ to total service time becomes less than 0.7, which is the sweet-spot for the tradeoff between complexity and improvements. Thus we can set ${T^{\rm{US}}=0.7T}$ for relatively large improvements at a relatively small complexity cost. 
   Energy consumption rises as the ratio of ${T^{\rm{US}}}$ to total service time shrinks, mainly because we aim to satisfy as much as possible in fewer time slots. When ${T^{\rm{US}}} = 0$, i.e. having zero CSI, our proposed method retrogresses to the referential method. 
   
   
   
   \begin{figure} [htb]
   \begin{center}
   \includegraphics*[width=8.8cm]{Cqos.eps}
   \end{center}
   \vspace*{-4mm} 
   \caption{System performance (energy consumption) under different QoS constraint (data volume constraint).}\label{fig:3}
   \vspace*{-2mm} 
   \end{figure}
   
   
   %\Fig.[htb]{Cqos.png}
   %{Average energy consumption per user $E_{avg}$ versus the QoS constraint ${V_{\rm{QoS}}}$.\label{fig3}}
   
   Fig. 3 shows the bit-wise average energy consumption under different QoS constraint (desired data volume for each user).
   Our proposed ship-to-ship/shore method outmatches the referential method. We achieve the best performance improvement when the QoS constraint is 500Mbits/user. The ship-to-ship/shore method consummates 67\% less energy than the referential method. When the QoS constraint is 3Gbits/user, we have the worst performance improvement, which is 40\% over the referential method energy-wise.
   
   The referential method's bit-wise energy consumption changes irregularly since it has zero information about CSI and chooses transmission links in a round-robin way. The proposed method's energy consumptions, on the other hand, increase as the data volume desired becomes larger. The rise in proposed method's energy consumption is because we end up choosing the time slots with low transmission speed in order to meet the increasing QoS constraint. This results in a larger energy consumption per user per Gbit. 
   
   
   
   \begin{figure} [htb]
   \begin{center}
   \includegraphics*[width=8.8cm]{snrs.eps}
   \end{center}
   \vspace*{-4mm} 
   \caption{System performance (energy consumption) under different BS's transmit power on given subcarrier.}\label{fig:5}
   \vspace*{-2mm} 
   \end{figure}
  
   Fig. 4 shows average energy consumption versus BS's transmit power under given QoS constraint. The transmit power ratio of BS and ship relays remains 10 during the change, i.e. the transmit powers of ship relays also change in direct proportion to the BS's transmit power. 
   As we can see, the energy consumption first decreases then increases. 
   The energy consumption increases first since the rise in transmit power results in the rise of transmission speed, and therefore the data distribution service can be done more quickly and in better CSI conditions. 
   The overall energy consumption then increases since the $log\left({}\right)$ operator in channel capacity (transmission speed) makes the increase in SNR less feasible when SNR is relatively large. Moreover, due to the greedy progressive methods for solving the user scheduling problem, the greedy choice we made focusing on links with best transmission speed also results in the increase energy-wise. 
  % The inflection point is larger in ship-to-ship/shore system since the transmission speed gain from lower ship relay transmit power is larger than higher BS transmit power, thus similarly the relay transmission can be done more quickly and in better CSI conditions. 
   
   By going through different transmit powers, we can find the optimal transmit power in any specific scenario. In this case, the optimal transmit power is around 8W for the BS (0.8W for the ship relays), smaller than the 10W BS assumption we first made. The optimal transmit power is also slightly higher in large-scale CSI based user scheduling schemes than genius-aided perfect CSI scenario. 
   
   %To sum up, the followings can be seen from the simulations. 
   %First, considering a longer time in user scheduling surely will improve system performance, but the sweet spot is 70\% of service duration. 
   %Second, a larger QoS data volume constraint will result in larger bit-wise energy consumption. 
   %Last, we can go through different powers to choose the optimal transmit power for given QoS constraint. As for performance, the total energy improvement is up to 67\%, while the worst case improvement is 46\%. 
  
   In section IV, we give the performance impact of different user scheduling time durations, as well as different QoS constraints. 
   Besides, by going through different powers, we can choose the optimal transmit power for given QoS constraint. 
   
   
   As shown by simulations, the approximate solution shows its potential as well as its simplicity. The proposed large-scale-CSI-based scheme can improve the performance by up to 67\% energy-wise, while only suffers 5\% - 15\% larger energy consumption over the genius-aided-full-CSI-based scheme. Thus the large-scale CSI replacement and the proposed 3-step progressive algorithm are justified. 
  
   %The main limitation of our progressive approximation comes from the inaccuracy in the large-scale CSI prediction. %We assumed that the ships sail strictly according to their shipping lanes and timetables. 
   %If there are any deviations in the ship routes and speeds, our large-scale CSI prediction will be inaccurate, and the performance of the proposed large-scale-CSI-based user scheduling algorithm will worsen. 
  
   
   \section{Conclusion}\label{sec:5}
   
   In this paper, we have proposed a user scheduling algorithm in a maritime ship-to-ship/shore communication system, aiming to reduce the system energy consumption. 
   In order to reduce the system energy consumption, we introduced ship-to-ship relay communication. 
   %By exploiting large-scale CSI for the whole service duration, we proposed our user scheduling algorithms. 
   Since large-scale CSI is the dominant factor in maritime scenarios, and the large-scale CSI can be easily predicted based on shipping lanes and timetables, we used large-scale CSI instead of full CSI to avoid the heavy overhead. Furthermore, we progressively approximated the user scheduling problem through a 3-step approximation, which has a polynomial time complexity. Simulations have shown that the proposed large-scale-CSI-based scheme significantly reduces the energy consumption by up to 67\% over existing methods.
   %Moreover, we have found the sweet spot of the user scheduling time and complexity tradeoff, which is 70\% of service duration.
   %In addition, a larger QoS data volume constraint will result in larger bit-wise energy consumption. 
  
   
  
   %It should be noted that transmit power allocation and user scheduling jointly influence the system energy consumption. In this paper, we have assumed fixed transmit power on any given subcarrier, and repeated the user scheduling algorithm under different power constraints to get the optimal transmit power. In the near future, we plan to design a joint power allocation and user scheduling algorithm for the maritime ship-to-ship/shore communication system. 
  
   
   
   \section*{Acknowledgement}
   
   This work was partially supported by the National Science Foundation of China under Grant No. 61771286, Grant No. 91638205, Grant No. 61701457, and Grant No. 61621091. %The authors Yunzhong Hou and Te Wei contributed equally to this work.
   
   
   \begin{thebibliography}{10}
    
    %maritime
    %\bibitem{p322}
    %T. Roste, K. Yang, and F. Bekkadal, ``Coastal coverage for maritime broadband communications,'' in
    %\emph{Proc. MTS/IEEE--Bergen}, Jun. 2013, pp.~1--8.
    
    %\bibitem{p101}
    %D. Liu, Y. Yu, C. Wen, and Z. Zhang, ``The GDUT Maritime Silk Road project (2014--2015) as a case study for VSMM in museum settings in China,'' in
    %\emph{Proc. Intern. Conf. Virtual System \& Multimedia}, Oct. 2016, pp.~1--9.
    
    \bibitem{p321}
    R. Campos, T. Oliveira, N. Cruz, A. Matos, and J. M. Almeida,
    ``BLUECOM+: Cost-effective broadband communications in remote ocean areas,'' in
    \emph{Proc. OCEANS}, Apr. 2016, pp.~1--6.
    
    \bibitem{p323}
    F. Bekkadal, ``Innovative maritime communications technologies,'' in
    \emph{Proc. Intern. Conf. Microwaves, Rader, \& Wireless Commun.}, June 2010, pp.~1--6.
    
    \bibitem{p32}
    M. Zhou, et al., ``TRITON: high-speed maritime wireless mesh network,''
    \emph{IEEE Wireless Commun.}, vol. 20, no. 5, pp.~134--142, 2013.
    
    
    \bibitem{p33}
    S. Buzzi, et al., ``A survey of energy-efficient techniques for 5G networks and challenges ahead,''
    \emph{IEEE J. Sel. Areas Commun.}, vol. 34, no. 4, pp.~697--709, 2016.
    
    \bibitem{p51}
    M. Shreedhar and G. Varghese, ``Efficient fair queuing using deficit round-robin,''
    \emph{IEEE/ACM Trans. Networking}, vol. 4, no. 3, pp.~375--385, 1996.
    
    \bibitem{p52}
    Q. Cao, Y. Sun, Q. Ni, S. Li, and Z. Tan, ``Statistical CSIT aided user scheduling for broadcast MU-MISO system,''
    \emph{IEEE Trans. Veh. Tech.}, vol. 66, no. 7, pp.~6102--6114, 2017.
    
    \bibitem{p53}
    J. Wang, M. Matthaiou, S. Jin, and X. Gao, ``Precoder design for multiuser MISO systems exploiting statistical and outdated CSIT,''
    \emph{IEEE Trans. Commun.}, vol. 61, no. 11, pp.~4551--4564, 2013.
    
    %\bibitem{p3}
    %X. Li, et al., ``Energy efficiency optimization: joint antenna-subcarrier-power allocation in OFDM-DASs,''
    %\emph{IEEE Trans. Wireless Commun.}, vol. 15, no. 11, pp.~7470--7483, 2016.
    
    %\bibitem{p6}
    %B. Di, L. Song, and Y. Li, ``Sub-channel assignment, power allocation, and user scheduling for non-orthogonal multiple access networks,''
    %\emph{IEEE Trans. Wireless Commun.}, vol. 15, no. 11, pp.~7686--7698, 2016.
    
    \bibitem{p4}
    L. Shan and R. Miura, ``Energy-efficient scheduling under hard delay constraints for multi-user MIMO System,'' in
    \emph{Proc. Intern. Symp. Wireless Personal Multimedia Commun.}, Sept. 2014, pp.~696--699.
    
    \bibitem{p5}
    S. Cao, Q. Cui, Y. Shi, H. Wang, and X. Ma, ``Cross-layer cooperative delay-energy tradeoff scheme for hybrid services in cellular networks,'' in
    \emph{Proc. IEEE Veh. Tech. Conf.}, May 2014, pp.~1--5.
    
    \bibitem{p7}
    X. Xiong, B. Jiang, X. Gao and X. You, ``QoS-guaranteed user scheduling and pilot assignment for large-scale MIMO-OFDM systems,''
    \emph{IEEE Trans. Veh. Tech.}, vol. 65, no. 8, pp.~6275--6289, 2016.
    
    %\bibitem{p8}
    %Rahul Singh, Alexander Stolyar, ``MaxWeight scheduling: Smoothness of the service process'',
    %\emph{IEEE 35th Annual IEEE International Conference on Computer Communications (INFOCOM)}, pp.~1-9, 2016.
    
   
    % 
    \bibitem{p300}
    T. Yang, H. Liang, N. Cheng, and X. Shen, 
    ``Towards video packets store-carry-and-forward scheduling in maritime wideband communication,'' in
    \emph{IEEE Global Commun. Conf. (GLOBECOM)}, Dec. 2013, pp.~4032--4037.
   
    \bibitem{p301}
    T. Yang, H. Liang, N. Cheng, R. Deng, and X. Shen, ``Efficient scheduling for video transmissions in maritime wireless communication networks,'' 
    \emph{IEEE Trans. Veh. Tech.}, vol. 64, no. 9, pp.~4215--4229, 2015.
   
    %\bibitem{p302}
    %A. Bejan, R. Gibbens, Y. Lim, and D. Towsley, 
    %``A performance analysis study of multipath routing in a hybrid network with mobile users,'' in 
    %\emph{Proc. of the 2013 25th International Teletraffic Congress (ITC)}, Sept. 2013, pp.~1--9.
   
    \bibitem{p303}
    Y. Bai, Y. Zhai, and D. Wang, 
    ``Research on optimum cooperative relay model for moving targets based on ant colony algorithm,'' in
    \emph{4th International Conference on Information, Cybernetics and Computational Social Systems (ICCSS)}, Nov. 2017, pp.~539--543.
   
   
    %\bibitem{p400}
    %T. Wei, W. Feng, N. Ge, and J. Lu, ``Optimized time-shifted pilots for maritime massive MIMO communication systems,'' in
    %\emph{Proc. 26th Wireless \& Optical Commun. Conf.}, Apr. 2017, pp. 1--5.
   
   
    %\bibitem{p402}
    %M. T. Zhou, H. Harada, P. Y. Kong, and J. S. Pathmasuntharama, ``Interference range analysis and scheduling among three-hop neighborhood in maritime WiMAX mesh networks,'' in 
    %\emph{Proc. IEEE Wireless Commun. \& Networking Conf.}, Apr. 2010, pp. 1--6.
   
    \bibitem{p403}
    W. Feng, Y. Wang, D. Lin, N. Ge, J. Lu, and S. Li, ``When mmWave communications meet network densification: a scalable interference coordination perspective,''
    \emph{IEEE J. Sel. Areas Commun. }, vol. 35, no. 7, pp. 1459--1471, 2017.
   
    \bibitem{p404}
    W. Feng, Y. Chen, N. Ge, and J. Lu, ``Optimal energy-efficient power allocation for distributed antenna systems with imperfect CSI,''
    \emph{IEEE Trans. Veh. Tech.}, vol. 65, no. 9, pp. 7759-7763, 2016.
   
    \bibitem{p405}
    W. Feng, Y. Wang, N. Ge, J. Lu, and J. Zhang, ``Virtual MIMO in Multi-Cell Distributed Antenna Systems: Coordinated Transmissions with Large-Scale CSIT,'' 
    \emph{IEEE J. Sel. Areas in Commun.}, vol. 31, no. 10, pp. 2067-2081, 2013.
   
    \bibitem{p406}
    Y. Chen, W. Feng, R. Shi, and N. Ge, ``Pilot-based channel estimation for AF relaying using energy harvesting,'' \emph{IEEE Trans. Veh. Tech.}, vol. 66, no. 8, pp.~6877--6886, 2017. 
  
    \bibitem{p407}
    Y. Zhang, W. Feng, and N. Ge, ``Pilot power adaptation for tomographic channel estimation in distributed MIMO systems,'' \emph{IET Communications}, vol. 11, no.1, pp.~112--118, 2017. 
   
    \bibitem{p410}
    J. S. Pathmasuntharam, J. Jurianto, P. Y. Kong, Y. Ge, M. Zhou, and R. Miura, ``High speed maritime ship-to-ship/shore mesh networks,'' in
    \emph{Proc. 7th Intern. Conf. ITS Telecommun.}, June 2007, pp. 1--6.
   
    %\bibitem{p411}
    %R. L. Moe, "Networking and ship-to-shore ship-to-ship communication," 
    %\emph{OCEANS '88. A Partnership of Marine Interests}, pp. 532-536 vol.2 , 1988.
   
    %\bibitem{p420}
    %R. Hemmecke, M. Köppe, J. Lee, and R. Weismantel, ``Nonlinear Integer Programming,'' in
    %\emph{Jünger M. et al. (eds) 50 Years of Integer Programming 1958-2008}, Springer, Berlin, Heidelberg, 2010.
   
   
    
   %\bibitem{p120}
   %M. Jung, K. Hwang, and S. Choi, ``Joint mode selection and power allocation scheme for power-efficient device-to-device (D2D) communication,'' in
   %\emph{Proc. IEEE Veh. Tech. Conf.}, May 2012, pp.~1--5.
   
   %\bibitem{p230}
   %H. Shin and J. H. Lee, ``Capacity of multiple-antenna fading channels: spatial fading correlation, double scattering, and keyhole'',
   %\emph{IEEE Trans. Info. Theory}, vol. 49, no. 10, pp.~2636--2647, 2003.
   
   \bibitem{p0}
   S. Balkees P A, K. Sasidhar, and S. Rao, ``A survey based analysis of propagation models over the sea,'' in
   \emph{Proc. Intern. Conf. Advances in Computing, Commun. \& Informatics}, Aug. 2015, pp.~69--75.
   
   \bibitem{p1}
   Y. Zhao, J. Ren, and X. Chi, ``Maritime mobile channel transmission model based on ITM,''
   \emph{Proc. Intern. Symp. Computer Commun. Control \& Automation}, vol. 68, no. 3, pp.~378--383, 2013.
   
   \bibitem{p2}
   J. C. Reyes-Guerrero, M. Bruno, L. A. Mariscal, and A. Medouri, ``Buoy-to-ship experimental measurements over sea at 5.8 GHz near urban environments,'' in
   \emph{Proc. Mediterranean Microwave Symp.}, Sept. 2011, pp.~320--324.
   
   
    %\bibitem{p22}
    %C. He, G. Y. Li, F. Zheng, X. You, ``Power Allocation Criteria for Distributed Antenna Systems'',
    %\emph{IEEE Transactions on Vehicular Technology (TVT)}, vol. 64, no. 11, pp.~5083-5090, 2015.
   
    \bibitem{p41}
    H. Shin and J. H. Lee, ``Capacity of multiple-antenna fading channels: spatial fading correlation, double scattering, and keyhole'',
    \emph{IEEE Trans. Info. Theory}, vol. 49, no. 10, pp.~2636--2647, 2003.
   
   % \bibitem{p9}
   % F. Fernandes, A. Ashikhmin, T. L. Marzetta, ``Inter-Cell Interference in Noncooperative TDD Large Scale Antenna Systems'',
   % \emph{IEEE Journal on Selected Areas in Communications (JSAC)}, vol. 31, no. 2, pp.~192-201, 2013.
   
    % \bibitem{p11}
    % N. Souto, R. Dinis, ``MIMO Detection and Equalization for Single-Carrier Systems Using the Alternating Direction Method of Multipliers'',
    % \emph{IEEE Signal Processing Letters (SPL)}, vol. 23, no. 12, pp.~1751-1755, 2016.
   
    % \bibitem{p123}
    % H. Zhang, C. Jiang, N. C. Beaulieu, X. Chu, X. Wen, M. Tao, ``Resource Allocation in Spectrum-Sharing OFDMA Femtocells With Heterogeneous Services'',
    % \emph{IEEE Transactions on Communications}, vol. 62, no. 7, pp.~2366-2377, 2014.
   
    % \bibitem{p8}
    % T. Yang, H. Liang, N. Cheng, R. Deng, X. Shen, ``Efficient Scheduling for Video Transmissions in Maritime Wireless Communication Networks'',
    % \emph{IEEE Transactions on Vehicular Technology (TVT)}, vol. 64, no. 9, pp.~4215-4229, 2015.
    
  
    \bibitem{p1000}
    R. Karp ``Reducibility Among Combinatorial Problems'', in 
    \emph{Complexity of Computer Computations} New York: Plenum. pp. 85–103, 1972
    
   \end{thebibliography}
   
   
   \end{document}
   