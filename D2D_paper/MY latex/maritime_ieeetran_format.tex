\documentclass[conference]{IEEEtran}
  %\IEEEoverridecommandlockouts
  %\documentclass[journal]{IEEEtran}
  % correct bad hyphenation here
  %\documentclass[10pt,draftclsnofoot, onecolumn]{IEEEtran}
  %\documentclass[10pt, draftclsnofoot,onecolumn]{IEEEtran}
  \hyphenation{op-tical net-works semi-conduc-tor}
  \usepackage{graphicx,amssymb,lineno}
  \usepackage{amsmath,amsfonts,amssymb}
  \usepackage{cases}
  \newtheorem{theorem}{Theorem}
  \newtheorem{lemma}{Lemma}
  \newtheorem{definition}{Definition}
  \usepackage{algorithm}
  \usepackage{algorithmic}
  \usepackage[usenames]{color}
  \usepackage{subfigure}
  %\usepackage{graphicx}
  
  \setlength{\columnsep}{0.22in}
  
 \begin{document}
 \title{User Scheduling for Maritime Ship-to-Ship/Shore Communications based on Large-Scale CSI}
 
 \author{\IEEEauthorblockN{Yunzhong~Hou\IEEEauthorrefmark{1}, Te~Wei\IEEEauthorrefmark{1}, Wei~Feng\IEEEauthorrefmark{1}, Ning~Ge\IEEEauthorrefmark{1}, and Yunfei~Chen\IEEEauthorrefmark{2} \\}
 \IEEEauthorblockA{\IEEEauthorrefmark{1}Tsinghua National Laboratory for Information Science and Technology, Beijing 100084, China \\
 \IEEEauthorrefmark{2}School of Engineering, University of Warwick, Coventry CV4 7AL, U.K.}
  E-mail: \{houyz14, wei-t14\}@mails.tsinghua.edu.cn, \{fengwei, gening\}@tsinghua.edu.cn, yunfei.chen@warwick.ac.uk%
 }
 
 %\address[1]{Tsinghua National Laboratory for Information Science and Technology, Tsinghua University, Beijing 100084, P. R. China}
 %\address[2]{School of Engineering, University of Warwick, Coventry CV4 7AL, U.K.}
 
 
 %\markboth
 %{Y. Hou \headeretal: Energy Efficient User Scheduling for Maritime Ship-to-Ship/Shore Communications}
 %{Y. Hou \headeretal: Energy Efficient User Scheduling for Maritime Ship-to-Ship/Shore Communications}
 
 %\corresp{Corresponding author: Wei Feng (e-mail: fengwei@tsinghua.edu.cn).}
 
 \maketitle
 
 \begin{abstract}
 
 The maritime ship-to-shore communication system has to cover a vast area with a limited number of base stations (BSs) due to the geographical restriction of BS sites. Therefore, its energy consumption is usually much larger than terrestrial cellular networks. 
 To reduce its energy consumption, ship-to-ship relay communication is introduced to the system since it allows ships to act like relays and enables direct communication between neighboring vessels. However, there are more transmitters to choose from in a ship-to-ship/shore system, which bring more challenges for user scheduling. 
 As a crucial technique for saving energy, user scheduling depends strongly on the channel state information (CSI), which is difficult to acquire in maritime scenarios due to the time-varying channel fading. Note that in maritime scenarios, the channel is dominanted by the large-scale CSI, which can be predicted through the position information of each vessel based on its specific shipping lane and timetable. Thus, we use the large-scale CSI instead of the complete CSI to avoid the heavy overhead. 
 %With the aid of large-scale CSI, we can have maximum benefit from the mobility of ships by proper user scheduling. 
 Based on predicted large-scale CSI, we formulate a user scheduling optimization problem, aiming to minimize the total energy consumption with guaranteed quality of service (QoS). To approximate the original problem, we develop a 3-step progressive offline approach and propose three consecutive algorithms for the approximation. The algorithms only require polynomial computational complexity. Simulation results reveal that the proposed offline user scheduling scheme based on large-scale CSI significantly reduces the energy consumption by up to 71\% over the referential method.
 \end{abstract}
 
 \begin{IEEEkeywords}
 Maritime ship-to-ship/shore communication, user scheduling, large-scale channel state information (CSI), progressive approach
 %, greedy method
 % mobility enhanced,
 \end{IEEEkeywords}
 
 
 \maketitle
 
 \section{Introduction}\label{sec:1}

 The rapid development of marine industries, as well as the economic and cultural exchanges between littoral states, has promoted the demand for reliable and high-speed ship-to-shore maritime communication services. In recent years, several maritime communication network projects have been developed in order to meet this increasing demand, e.g., the BLUECOM+ project, the MarCom project, and the TRITON project \cite{p321}--\cite{p32}. 
 
 Unlike terrestrial cellular networks, a maritime ship-to-shore communication system has quite limited geographically available BS sites. Thus, maritime communication systems usually adopt high-powered BSs so as to cover a vast area with limited BSs. This high-powered BS strategy increases the operational costs of mobile network operators and poses a global threat to the environment \cite{p33}.
 Accordingly, reducing energy consumption becomes a critical issue for maritime communications. %In addition to ship-to-shore direct communications, we introduce the idea of ship-to-ship relay transmissions, which may reduce the total energy consumption by exploiting direct communications between neighboring users. Unfortunately, the introduction of ship-to-ship relay communications greatly increases the difficulties in user scheduling. As an important perspective for saving energy, user scheduling has attracted increasingly worldwide attention. However, user scheduling heavily depends on CSI, whereas perfect CSI is hard to acquire in maritime scenarios. In this paper, we are devoted to addressing these difficulties by proposing a progressive approach based on large-scale CSI. 
 To reduce system energy consumption, we introduce the idea of ship-to-ship relay communications, where vessels act like relays in the on-shore data distribution network. Several previous maritime communication network projects \cite{p321}--\cite{p32} have already included ship-to-ship communications to extend the coverage of the systems. Nevertheless, to the best of the authors' knowledge, the area still remains undiscovered where ship-to-ship relay transmissions are introduced to reduce system energy consumption. 

 Undoubtedly, the ship-to-ship relay transmission will reduce system energy consumption with direct relay transmission between neighboring vessels. 
 However, it introduces more transmitters (BS/relays) and brings more challenges for user scheduling. % (see Appendix A)

 User scheduling, as an important perspective for energy saving, depends heavily on CSI. However, it is rather costly to acquire perfect CSI in maritime communication systems, due to the excessive system overhead including pilot overhead and feedback overhead \cite{p403}-\cite{p405}. Most of the previous user scheduling schemes suffer from the difficulties in obtaining full CSI in maritime scenarios. 
 
 %terrestrial
 So far, the majority of energy-efficient user scheduling techniques focused on terrestrial cellular networks. %, and CSI is a crucial factor therein. 
 %such as the proportional fairness based schemes in \cite{p61}--\cite{p63}, the signal-to-leakage-interference-plus-noise ratio-based methods in \cite{p64}--\cite{p66}, the coordinated scheduling with cyclic beamforming in \cite{p67}\cite{p68}, and the iterative algorithms in \cite{p69}\cite{p70}. 
 Based on the utilization degree of CSI, terrestrial user scheduling schemes can be classified into three categories. The first one required no CSI, such as the simple but efficient round-robin scheme for fair queuing \cite{p51}. The second one exploited statistical and outdated CSI, as studied in \cite{p52} and \cite{p53}. The third one assumed full CSI, and utilized the instantaneous CSI for user scheduling on a minuscule time scale, i.e., in each coherence time \cite{p4}-\cite{p7}. 
 %A joint power allocation and user scheduling algorithm based on dynamic programming (DP) was proposed for multi-user MIMO systems to minimize the total energy consumption under hard delay constraints in \cite{p4}. 
 %In \cite{p5}, a cross-layer cooperative user scheduling and power allocation scheme was developed for hybrid-delay services, and the fundamental tradeoff between delay and energy consumption was illustrated. 
 %Lately, a user scheduling and pilot assignment scheme for massive MIMO systems was proposed in \cite{p7} to serve the maximum number of users with guaranteed QoS. 
 In maritime scenarios, however, obtaining perfect CSI for user scheduling brings large overheads and is not cost-effective.
 Moreover, since the large-scale CSI can be predicted in maritime scenarios (which will be explained later), statistical and outdated CSI is not the best choice, either. 

 %maritime
 As for maritime user scheduling with ship relays, limited works have been done. %concentrated on energy efficiency. 
 Both \cite{p300} and \cite{p301} focused on monitoring videos uploading via maritime communication networks. They focus on user scheduling of ship-to-shore relay communication with the store-carry-and-forward mechanism.  
 %Towards video packets store-carry-and-forward scheduling in maritime wideband communication
 %Efficient Scheduling for Video Transmissions in Maritime Wireless Communication Networks
 In \cite{p303}, a scheduling model was developed to provide the communication path of the fewest routing times to the moving ships that are far apart, which has reduced the space link resources consumption. 
 %Research on optimum cooperative relay model for moving targets based on ant colony algorithm
 Nevertheless, these works were also based on the assumption that the channel capacity is known, and did not consider the real-world challenges in acquiring full CSI. 
 %In \cite{p302}, the authors studied the performance of a multipath TCP controller and demonstrated how path diversity can be implicitly utilized to spread flows across available paths. 
 %A performance analysis study of multipath routing in a hybrid network with mobile users
 %In \cite{p400}, an efficient user scheduling algorithm aiming to optimize the pilot power under the average power constraint was proposed. 
 %In \cite{p402}, transmission of MAC control messages and data packets within the three-hop neighborhood is investigated for the purpose of minimizing interference. 
 
 
 
 %when mmwave communications meet network densification
 %问题-》帽子
 For the following reasons, current studies on user scheduling require systematic redesign, and should be based on large-scale CSI instead of full CSI. 
 
 %1.海域模型,大尺度主导
 \textbf{1.} As there are fewer scatterers on the sea than that in the terrestrial scenario, the large-scale channel fading becomes the dominant factor for the maritime channel \cite{p403}. Hence, we can use the position information of vessels to predict large-scale CSI and avoid the heavy overhead for full CSI; 
 %when mmwave communications meet network densification
 
 %2.航线·位置信息
 \textbf{2.} %Different from the random trails of human beings in terrestrial scenarios where the previous studies focused on,
 Most vessels have specific fixed shipping-lanes and timetables that can be acquired beforehand, and their position information can be easily predicted. From the position information, we can predict large-scale CSI for the whole service duration. We can have further gain by considering the whole service process instead of the short timescale as in terrestrial scenarios. 
 
 Therefore, we formulate the optimization problem for maritime user scheduling based on large-scale CSI, aiming to minimize the system energy consumption. % while providing users with delay-tolerant data distribution services. 
 To overcome the difficulties in choosing transmitters brought forward by relay transmission, we progressively approximate the optimal solution. We further propose 3 consecutive algorithms for our 3-step progressive approach. The algorithms we proposed all have polynomial time complexity.
 
 %\subsection{Organization and Notation}
 %文章组织
 %The rest of the paper is organized as follows.
 
 %Section II introduces the system model, where a multi-user maritime ship-to-ship/shore communication system is considered, and the formulation of the optimization problem for user scheduling is presented. 
 %In Section III, the problem is progressively approximated in a 3-step progressive approach. 
 %Section IV presents simulation results along with further discussions. 
 %Finally, Section V gives the concluding remarks. 
 
 %Throughout this paper, lightface symbols represent scalars, while boldface symbols denote vectors, matrices or sets. ${\mathbf{I}}$ represents an identity matrix, $\mathbb{E}[x]$ denote the expectation of $x$, and $\mathcal{CN}(0, {\sigma}^2)$ denotes the complex Gaussian distribution with zero mean and ${\sigma}^2$ variance. %$I\left[ x \right]$ is the indicative function where $I\left[ x \right] = \left\{ \begin{array}{l}
 % 1,x > 0,\\
 % 0,else.
 % \end{array} \right.$
 %$[x]^{+}\triangleq{\mathop {\max }(x,0)}$. $\lfloor x \rfloor$ and $\lceil x \rceil$ denote the largest integer not greater than $x$ and the smallest integer not less than $x$, respectively. ${\mathbf{A}}^T$ and ${\mathbf{A}}^H$ represent the transpose and the transpose conjugate of ${\mathbf{A}}$, respectively. 
 
 \section{System Model}\label{sec:2}
 
 \begin{figure} [htb]
 \begin{center}
 \includegraphics*[width=8.8cm]{system.png}
 \end{center}
 \vspace*{-4mm} 
 \caption{Maritime ship-to-ship/shore communication system for data distribution service.}\label{fig:1}
 \vspace*{-4mm} 
 \end{figure}
 
 
 \subsection{System Parameters}
 
 As shown in Fig. 1, the following sections focus on the user scheduling of an FDMA downlink transmission of a single-BS maritime ship-to-ship/shore communication system. In the system, there are $J$ single-antenna users that can either act like receivers or like transmitters (relays). In general, we denote the transmitters (BS/relays) as $i \in \left\{ {0,1,...,J} \right\}$, the receivers (users) as $j \in \left\{ {1,...,J} \right\}$. $i=0$ means the transmitter is BS, whereas $i>0$ means the transmitter is one of the $J$ ship that can act like relays. We assume that there are $N$ subcarriers, and the subcarrier bandwidth is ${B_s}$. $P_i = \left\{ {P_0,\left\{ {P_j} \right\}} \right\}$ represents the fixed transmit power of BS ($i=0$) or ship relays ($i>0$) on any subcarrier. In this paper, we only consider two-hop half-duplex `ship-to-ship/shore communications' for simplicity. 
 
 In the studied system, ship-to-ship transmissions use the same licensed band of ship-to-shore transmissions (i.e. one of the $N$ subcarriers), and the same air interface of the ship-to-shore transmissions. 
 At any given time, each ship-to-ship or ship-to-shore links will use distinct subcarrier. Here in this paper by `link' we mean the transmission from BS/relay to a receiver during a certain time period. 
 %Given that we only consider half-duplex transmission, the $J$ single-antenna users can either receive data from one transmitter (BS/relay) or send data to another user (act as a relay) at any given time. 
 
 Without loss of generality, we assume the on-shore BS coverage shape to be a semicircle. 
 In order to simplify the problem, we only consider transmissions within the semicircle. We also assume that all the users request different data and the system has no ship-to-ship link data reuse. 
 % with radius $R$. 
 Each user sails into and out of the semicircle according to its shipping lane and timetable. 
 Since maritime users focus more on data volume rather than transmission delay, we particularly focus on the delay-tolerant information distribution service, which is initiated and terminated when a marine user sails into and out of the BS's coverage, respectively. 
 The QoS constraint, i.e., the data volume required by the ${j^{th}}$ user, is denoted by $V_j^{\rm{QoS}}$. Moreover, the delay-tolerant assumption can bring forward great potential for user scheduling. 
 
 \subsection{Large-Scale CSI}
 
 In terrestrial scenarios, according to the multipath effect, signals are well scattered and the small-scale fading factor has a significant impact on the channel. However, in maritime scenarios, due to the scarcity of scatterers, the large-scale fading factor becomes dominant. Therefore, we focus on large-scale CSI in this paper.
 We assume a modified 2-ray propagation model for the maritime channel, since the sea surface is relatively flat \cite{p0}--\cite{p2}. For a given subcarrier, we denote the composite channel gain from the BS/relay $i$ to the user $j$ at time $\tau $ by $\sqrt {{\beta _{i,j,\tau }}} {h_{i,j,\tau }}$. The small-scale fading vectors ${h_{i,j,\tau }}$ follow a complex Gaussian distribution with standard deviation ${\sigma _s} = 1$, i.e., ${h_{i,j,\tau }} \sim \mathcal{CN}(0, \mathbf{I})$. The large-scale fading coefficient ${\beta _{i,j,\tau }}$ is expressed as
 \begin{align}
 {\beta _{i,j,\tau }} = {\left( {\frac{\lambda }{{4\pi {d_{i,j,\tau }}}}} \right)^2}{\left[ {2\sin \left( {\frac{{2\pi {h_t}{h_r}}}{{\lambda {d_{i,j,\tau }}}}} \right)} \right]^2} ,
 \end{align}
 where $\lambda $ is the carrier wavelength, ${d_{i,j,\tau }}$ is the distance between the BS/relay $i$ and the user $j$ at time $\tau $. The antenna height of the transmitter and the receiver are represented by $h_t$ and $h_r$, respectively. 
 
 To fully utilize the slowly-varying characteristic of the large-scale channel fading, we divide the total system service time into $T$ time slots, each lasting $\Delta \tau$. The value $\Delta \tau$ is carefully chosen so that $\beta _{i,j,\tau }$ remains constant in each time slot $t$ (ignoring ship movements during each time slot). Thus, we make it possible to acquire $\beta _{i,j,t} = \mathbb{E} \left [ {\beta _{i,j,\tau }} \right ]$ for $\forall t \in \left\{ {1,...,T} \right\}$ from position information based on shipping-lanes and timetable. 
 In this paper we replace the perfect CSI with large-scale CSI and use predicted large-scale CSI to predict transmission speed (capacity) as shown in (2a)-(2c). We justify our replacement by simulations in Section IV. Let us denote ${\gamma _{i,j,t }} = {\raise0.7ex\hbox{${{P_{i} }{\beta _{i,j,t }}}$} \!\mathord{\left/
  {\vphantom {{{P_{i} }{\beta _{i,j,t }}} {{\sigma ^2}}}}\right.\kern-\nulldelimiterspace}
 \!\lower0.7ex\hbox{${{\sigma ^2}}$}}$ for simplicity, where ${P_{i}}$ represents the transmit power from BS/relay $i$ to receivers. The channel capacity or transmission speed in this paper can therefore be simplified as
 \begin{subequations}
 \begin{align}
 {r_{i,j,t}} & = {{B_s}{{\log }_2}\left( {1 + \frac{{{P_{i} }{\beta _{i,j,t }}{{\left| {{h_{i,j,t }}} \right|}^2}}}{{{\sigma ^2}}}} \right)},\\
 & = {B_s}{\log }_2 \left( {1 + {\gamma _{i,j,t }}{{\left| {{h_{i,j,t }}} \right|}^2}} \right),\\
 & = \left( {{{\log }_2}e} \right){e^{\frac{1}{{{\gamma _{i,j,t }}}}}}\int_1^\infty {\frac{1}{u}{e^{ - \frac{u}{{{\gamma _{i,j,t }}}}}}du} .
 \end{align}
 \end{subequations}
 The transmission speed in (2c) is derived based on previous study \cite{p41}. Any further denotation of CSI in this paper refers to the predicted `large-scale CSI' for the whole service duration unless specified. The impact of this replacement (assuming that ship $j$ stays in the same position and $\beta _{i,j,\tau }$ remains constant in each time slot $t$) is further discussed in Section IV. 
 %With the long-term large-scale channel fading (CSI) known beforehand, we can further design and implement a user scheduling scheme.

 For simplicity, we denote the link from transmitter $i \in \left\{ {0,1,...,J} \right\}$ (BS/relay, $i = 0$ means BS, $i > 0$ means relay) to receiver $j$ (user) at time slot $t$ by $i \to j@t$. Since we only consider two-hop links, each of the substitution ship-to-ship/shore links $\left[ {0 \to i'@{t_1},i' \to j@{t_2}} \right]$ consists of exact a ship-to-shore part $0 \to i'@{t_1}$ for BS to transmit data to relay ${i'}$, and a ship-to-ship part $i' \to j@{t_2}$ for relay ${i'}$ to transmit to receiver user $j$. 
 
 
 \subsection{Problem Formulation}
 
 The total energy consumption can be written as the sum of transmission energy to each user. %For each user, they receive transmissions from different transmitter $i$ in each time slot $t$. 
 Therefore the energy consumption in this system is
 \begin{align}
  {{E_{\rm{total}}} = \sum\limits_{j = 1}^J {{E_j}} = \sum\limits_{j = 1}^J {\left( {\sum\limits_{i=0}^{J}{\sum\limits_{t = 1}^T {{P_{i}}\delta _{i,j,t}} } \Delta \tau  }\right)}}. 
  \end{align}
 %where $\Delta \tau $ represents the transmission duration from transmitter $i$ to receiver $j$ during time slot $t$. %average power consumed by the transmission from BS/relay $i$ to user $j$ during time slot $t$.
 By $\delta _{i,j,t} \in \left\{ {0,1} \right\}$ we denote whether a subcarrier is scheduled for the link $i \to j @ t$. $ {{\delta _{i,j,t}}} = 0$ means there is no transmission from BS/relay $i$ to user $j$ at time slot $t$, while $ {{\delta _{i,j,t}}} = 1$ means there is a transmission link $i \to j @ t$ on a certain subcarrier. Moreover, $\forall j,\forall t,{\delta _{j,j,t}} \equiv 0$, since we do not allow receiving transmissions from oneself. 
 Our objective is to minimize the system energy consumption by means of user scheduling in the ship-to-ship/shore communication system. 
 %For the link $i \to j@t$, 
 %Thus, the used transmission time $\Delta t_{i,j,t}$ for a given link $i \to j @ t$ can be denoted by 
 %\begin{align}
 % {t_{i,j,t}} = \left( {\Delta \tau } \right){\delta _{i,j,t}}.
 %\end{align}
 % and the transmission use ${\eta _{i,j,t}}$ of the transmitter's max transmit power. 
 
 We denote the total data volume that user $j$ currently has at time slot $t$ by ${V_{j,t}}$. ${V_{j,t}}$ can be written as the sum of the received data volume in each time slot minus the sum of relayed data volume in each time slot. 
 \begin{align}
  {V_{j,t}} = \sum\limits_{\tau = t_j^{{\rm{In}}}}^t {\left( {\sum\limits_i {{r_{i,j,\tau }\delta _{i,j,\tau}}} - \sum\limits_{j'} {{r_{j,j',\tau }\delta _{j,j',\tau}}} } \right) \Delta \tau } .
 \end{align}
 The time slots when user $j$ enters and leaves the BS's coverage are denoted by $t_j^{{\rm{In}}}$, $t_j^{{\rm{Out}}}$, respectively. In the considered system, based on delay-tolerant assumption, $t_j^{{\rm{In}}}$, $t_j^{{\rm{Out}}}$ equal the service beginning time slot and ending time slot for user $j$. 
 Since the system has no ship-to-ship link data reuse, user relay $j$ must have enough data ${V_{j,t}}$ to relay and transmit, i.e., ${V_{j,t} \ge 0}$. 

 Although different users require different data in our system, it is possible to avoid keeping track of what BS/relays transmit to each user at any given time. To do this, we have to record all ship-to-ship/shore links we have chosen in a link set ${\mathbf{S}}$. ${\mathbf{S}}$ can be acquired before ships enter the BS coverage based on our proposed algorithms. With the link set known beforehand, we can get the transmission speed of each link in the set, and determine what and how much to transmit to each user at any given time slot.
 
 Thus, we formulate the energy consumption optimization problem as
 \begin{subequations}
 \begin{align}
 & \mathop {\min }\limits_{{\left\{ {{\delta _{i,j,t}}} \right\}^{\left( {J + 1} \right) \times J \times T}}} \left\{ {\sum\limits_{i = 0}^J {\sum\limits_{j = 1}^J {\sum\limits_{t = 1}^{T} {{P_{i}}\delta _{i,j,t} \Delta \tau } } } } \right\} ,\\
  {s.t.} \;\; &\sum\limits_{i \ne j} {{{\delta _{i,j,t}}} } + \sum\limits_{j' \ne j} { {{\delta _{j,j',t}}} \le {1}} ,\\
  \;\;\;\;\;\; &\sum\limits_i {\sum\limits_j { {{\delta _{i,j,t}}} } } \le N ,\\
  \;\;\;\;\;\; &{{{\left. {{V_{j,t}}} \right|}_{t = t_j^{{\rm{In}}}}} = 0, \left. {{V_{j,t}}} \right|_{t = t_j^{{\rm{Out}}}}} \ge V_j^{\rm{QoS}}, {V_{j,t}} \ge 0.
 \end{align}
 \end{subequations}
 We have to consider transmissions from ${J + 1}$ transmitters (BS/relays) to $J$ receivers (users) at $T$ time slots in problem (5). 
 %, and our optimization is in a $\left( {J + 1} \right) \times J \times T$ 3-dimensional subspace. 
 Half-duplex constraint (5b) guarantees that each user has access to at most one BS/user at a given time, and serves either as a transmitter or as a receiver. The constraint in (5c) guarantees that at most $N$ users can be severed simultaneously in the system, by BS or relays, since there are only $N$ subcarriers. (5d) makes sure that the QoS constraint is met and relays cannot transmit more than what they have currently.
 
 Given the difficulties in solving the Linear Integer Programming problem in (5), we propose the following progressive approach. 
 %\textit{Theorem 1:} The problem in (4) is NP-hard.
 
 %\textit{Proof:} See Appendix A. 
 
 \section{User Scheduling for Maritime Ship-to-Ship/Shore Communication}\label{sec:3}
 
 In this section, we focus on the user scheduling problem, which reduces system energy consumption while ensuring QoS. We progressively approach the optimization problem in (5) through 3 consecutive algorithms with polynomial time complexity.
 
 %\subsection{Problem Formulation}
 
 \subsection{Progressive Offline Approach of the Original-Problem}
 
 The original problem in (5) involves various factors, and achieving the optimal solution for it is not practical. 
 In order to approximate the optimal solution, we first loosen some constraints in (5), and then gradually add them back to approximate the original problem through a 3-step progressive offline approach, with each step based on its predecessor's result. 
 
 In step-1, we focus on user scheduling enabled by large-scale CSI, since the key factor of our scheduling is the utilization of large-scale CSI that we can predict. We simply consider the ship-to-shore transmission and ignore the subcarrier constraint. We can get an optimal result for this simple problem by choosing links with best CSI. 
 
 In step-2, we consider the maritime ship-to-ship/shore communication system. We substitute part of the ship-to-shore links we get in step-1 for two-hop ship-to-ship/shore links $\left[ {0 \to i'@{t_1},i' \to j@{t_2}} \right]$ for less energy consumption. Since the introduction of relays brings forward great difficulties in user scheduling, we use a greedy method in step-2. 
 %The ship-to-ship/shore links  (one ship-to-shore and one ship-to-ship) in the substitution link set must use less energy combined than the original ship-to-shore link $0 \to j@{t_0}$ for improvement energy-wise. 
 
 Last, in step-3, we use a greedy algorithm based on the result returned by step-2 to make sure that our user scheduling is applicable, i.e., the subcarrier constraint is met for the links. Since this constraint has not been considered in step-1, we make adjustments in step-3 to get an approximation of the applicable solution for the ship-to-shore system. 

 Eventually, after all three steps, we approximate the optimal solution for the original problem in (5). Since the transmission speed for all $\left( J+1 \right) \times J \times T$ links can be predicted based on large-scale CSI, the 3 consecutive algorithms can be carried out in a offline manner, i.e., they can be done before the start of service. In Section IV we prove the validity of our progressive approach by comparing the energy consumptions. 
 
 \subsection{Step-1}
 
 For the first step, we concentrate on large-scale CSI by considering the most simple scenario: a ship-to-shore only system without subcarrier constraint. We fix transmitter $i = 0$ since users can only receive data from on-shore BS. 
 \begin{subequations}
 \begin{align}
 & \mathop {\min }\limits_{\left\{ {{\delta _{0,j,t}}} \right\}} \left\{ {\sum\limits_{j = 1}^J {\sum\limits_{t = 1}^{T} {{P_{0}}\delta _{0,j,t} \Delta \tau } \,} } \right\} ,\\
  {s.t.} \;\; &{{{\left. {{V_{j,t}}} \right|}_{t = t_j^{{\rm{In}}}}} = 0,\left. { {V_{j,t}}} \right|_{t = t_j^{{\rm{Out}}}}} \ge V_j^{\rm{QoS}}, {V_{j,t}} \ge 0.
 \end{align}
 \end{subequations}
 We only optimize $\left\{ {{\delta _{0,j,t}}} \right\}^{J \times T}$ since in step-1, there is only one transmitter. 
 %since the optimization is currently in a $J \times T$ 2-dimensional subspace (the transmitter dimension degenerates since there is only one transmitter, namely BS) 
 Half-duplex constraint in (5b) is not necessary here since users can only receive data from BS, and cannot act like relays. We also drop the $N$-subcarrier constraint in (5c) since we assume that the BS can serve infinite number of users. 
 
 In the first step, we optimize ${\delta _{0,j,t}}$ only with constraint (6b). In this case, the optimization variables of different users are no longer correlated, and the optimal solution here can be obtained by scheduling each user separately. The problem in (6) can be reduced to $\mathop {\min }\limits_{{\left\{ {{\delta _{0,j,t}}} \right\}}} \left\{ {\sum\limits_{t = 1}^T {{P_{0}}\delta _{0,j,t} \Delta \tau } } \right\}$. Note that ${r_{0,j,t}}$ is a monotone increasing function of ${\beta _{0,j,t}}$, therefore we can obtain the optimal solution by choosing links with best CSI (transmission speed) for each user. 
 
 We further define ${{\mathbf{S}}_{\mathbf{1}}}$ as the set of chosen ship-to-shore link at a specific time slot in problem (6), i.e., $\left( {0,j,t} \right) \in {\mathbf{S}}_{\mathbf{1}}$ if ${\delta _{i,j,t} = 1}$. 
 
 For each user, we find link $0 \to j@t$ with best ${r _{0,j,t}}$ and set ${\delta _{0,j,t} = 1}$ until the QoS constraint is met.
 
 \begin{algorithm}[ht]
  \caption{Optimal User Scheduling for Maritime Ship-to-Shore System Regardless of Subcarrier Constraint}
  \label{alg:1}
  \begin{algorithmic}[1]
  \STATE Initialize ${{\mathbf{S}}_{\mathbf{1}}}=\phi$
  \FOR{all user $j$}
   \WHILE{${\left. {{V_{j,t}}} \right|_{t = t_j^{{\rm{Out}}}}} < {V^{\rm{QoS}}}$}
    \STATE Find ship-to-shore link with highest transmission speed. 
    \STATE Add them to ${{\mathbf{S}}_{\mathbf{1}}}$.
    \ENDWHILE
  \ENDFOR
 \end{algorithmic}
 \end{algorithm}

 \subsection{Step-2}
 
 In step-2 of the progressive approach, we greedily change many ship-to-shore links into fewer ship-to-ship/shore links with higher transmission speed for lower energy consumption. We record the result we get in step-2 as ${{\mathbf{S}}_{\mathbf{2}}}$. After step-1 we get ${{\mathbf{S}}_{\mathbf{1}}}$, which is an optimal solution for problem (6). ${{\mathbf{S}}_{\mathbf{1}}}$ only contains ship-to-shore links like $0 \to j@t'$, while ${{\mathbf{S}}_{\mathbf{2}}}$ also contains ship-to-ship/shore links like $\left[ {0 \to i'@{t_1},i' \to j@{t_2}} \right]$. 
 
 Since the introduction of ship-to-ship links brings forward energy reduction by reducing the number of ship-to-shore links, we can approximate the original problem in (5a)-(5d) by maximizing the energy consumption reduction between ${{\mathbf{S}}_{\mathbf{2}}}$ and ${{\mathbf{S}}_{\mathbf{1}}}$ by 
 \begin{align}
   &\mathop {\max }\limits_{{\left\{ {{\delta _{i,j,t}}} \right\}}} \left\{ {{\sum\limits_{j = 1}^J \sum\limits_{t = 1}^{T}{\left( {\mathop {{P_{0}}\delta _{0,j,t} }\limits_{\left( {0,j,t} \right) \in {{\mathbf{S}}_{\mathbf{1}}}} - \sum\limits_{i = 0}^J {\mathop {{P_{i}}\delta _{i,j,t}}\limits_{\left( {i,j,t} \right) \in {{\mathbf{S}}_{\mathbf{2}}}} } } \right) \Delta \tau } } } \right\}.
 \end{align}
 In step-2, we maximize the difference term (7) to approximate the optimal solution. 

 Aiming to maximize the system energy reduction, for each user, we greedily choose ship-to-ship/shore links with lowest energy consumption under given transmission speed, i.e., the composite power-to-rate ratio $ {\frac{{{P_0}}}{{{r_{0,i',{t_1}}}}} + \frac{{{P_{i'}}}}{{{r_{i',j,{t_2}}}}}} $. If the ship-to-ship/shore links have lower composite power-to-rate ratio, i.e., $ {\frac{{{P_0}}}{{{r_{0,i',{t_1}}}}} + \frac{{{P_{i'}}}}{{{r_{i',j,{t_2}}}}}} < \frac{{{P_0}}}{{{r_{0,j,{t_0}}}}}$, then we substitute the original link into ship-to-ship/shore links (one link from BS to relay, the other from relay to user). We do this to maximize the difference term (7) since 
 \begin{align}
   &\Delta E  = \left[ {P_0}- \left( {\frac{{{P_0}}}{{{r_{0,i',{t_1}}}}} + \frac{{{P_{i'}}}}{{{r_{i',j,{t_2}}}}}} \right){r_{0,j,{t_0}}} \right] \Delta \tau.
 \end{align}

 The algorithm in step-2 is carried out as follows.
 
 For each user $j$, we first record all plausible ship-to-ship/shore links like $\left[ {0 \to i'@{t_1},i' \to j@{t_2}} \right]$ in a temporary set $\mathbf{R}$. Here `plausible' means that the half-duplex constraint in (5b) and the $N$-subcarrier constraint in (5c) are satisfied, plus both parts of the links have higher transmission speed than the original links. Further exploration will be conducted in the plausible ship-to-ship/shore link set $\mathbf{R}$. We assume that in each ship-to-ship/shore links, the BS to relay part and relay to user part transmit the same volume of data.

 Once we have the substitution set $\mathbf{R}$, we add ship-to-ship/shore links to the system until we can meet the QoS constraint only with ship-to-ship/shore links, i.e., ${\left. {{V_{j,t}}} \right|_{t = t_j^{{\rm{Out}}}}} \ge 2{V^{\rm{QoS}}}$ (we may substitute all the original links and only have ship-to-ship/shore links for user $j$). Continue those steps until the plausible link set $\mathbf{R}$ become empty or there is no gain energy-wise from substitution, i.e., there are no ship-to-ship/shore links that have lower composite power-to-rate ratio $ {\frac{{{P_0}}}{{{r_{0,i',{t_1}}}}} + \frac{{{P_{i'}}}}{{{r_{i',j,{t_2}}}}}} $ than the original links. 
 After this, we complete our substitution by removing original links with relatively higher power-to-rate ratio ${\frac{P_0}{r_{0,j,{t_0}}}}$.

 %Once we have the plausible set $\mathbf{R}$, we find the ship-to-ship/shore substitution links $\left[ {0 \to i'@{t_1},i' \to j@{t_2}} \right]$ that have best highest transmission speed. Then we iteratively find original ship-to-shore link $0 \to j@{t_0}$ with lowest transmission speed, remove it from ${{\mathbf{S}}_{\mathbf{2}}}$, replace it with ship-to-ship/shore substitution links and add the substitution links to ${{\mathbf{S}}_{\mathbf{2}}}$. 
 %Continue this iteration till the substitution links $\left[ {0 \to i'@{t_1},i' \to j@{t_2}} \right]$ have transmitted as much as they can (${\eta _{0,i',{t_1}}=1}$ or ${\eta _{i',j,{t_2}}=1}$), we remove $\left[ {0 \to i'@{t_1},i' \to j@{t_2}} \right]$ from $\mathbf{R}$ and choose a new pair of substitution links with best CSI in the reamining set $\mathbf{R}$. 
 

 %To denote the remaining transmission capability in each greedy substitution iteration, we introduce $\eta _{i,j,t}=\frac{{V_{i,j,t}^{{\rm{used}}}}}{{{r_{i,j,t}}\Delta \tau }}$ as the ratio of the data volume already transmitted (${V_{i,j,t}^{{\rm{used}}}}$) to the total data volume the link $i \to j @ t$ can transmit. During the substitution iteration, the data volume is conserved, and $\eta _{0,i',t_1},\eta _{i',j,t_2} $ can be calculated based on the data volume of the link to be substituted $\Delta {V_{0,j,t_0}^{{\rm{subs}}}}=r_{0,j,t_0}\Delta \tau $. 
 %Thus we can update $\eta _{0,i',t_1}$ and $\eta _{i',j,t_2} $ by 
 %\begin{subequations}
 % \begin{align}
 %  {\eta _{0,i',t_1}^{\left( {n+1} \right)}} & = {\eta _{0,i',t_1}^{\left( n \right)}} + \frac{V_{0,j,t_0}^{{\rm{subs}}}}{{{r_{0,i',{t_1}}}\Delta \tau }} = {\eta _{0,i',t_1}^{\left( n \right)}} + \frac{{{r_{0,j,{t_0}}}}}{{{r_{0,i',{t_1}}}}},\\
 %  {\eta _{i',j,t_2}^{\left( {n+1} \right)}} & = {\eta _{i',j,t_2}^{\left( n \right)}} + \frac{V_{0,j,t_0}^{{\rm{subs}}}}{{{r_{i',j,{t_2}}}\Delta \tau }} = {\eta _{i',j,t_2}^{\left( n \right)}} + \frac{{{r_{0,j,{t_0}}}}}{{{r_{i',j,{t_2}}}}}.
 % \end{align}
 %\end{subequations} 
 %where the superscript $^{\left( n \right)}$ means the value after $n^{th}$ iteration. 
 %$\eta _{i,j,t}=0$ means the link have not been used, while $\eta _{i,j,t}=1$ means the link is all used up. With $\eta _{i,j,t} \in \left[{0,1}\right]$, we can easily tell if a link is used up or it can still be used for substitution. 
 
 
 
 %The energy consumption reduction in each iteration can be expressed as 
 %\begin{align}
 %   \Delta E & = \left[ {P_0}- \left( {\frac{{{P_0}}}{{{r_{0,i',{t_1}}}}} + \frac{{{P_{i'}}}}{{{r_{i',j,{t_2}}}}}} \right){r_{0,j,{t_0}}} \right] \Delta \tau.
 %\end{align}
 %Note that the term $ {\frac{{{P_0}}}{{{r_{0,i',{t_1}}}}} + \frac{{{P_{i'}}}}{{{r_{i',j,{t_2}}}}}} $ in (9) can be regarded as weighted sum of the reciprocal of the transmission speeds, we can use this term to indicate the transmission speed of the substitution links. Furthermore, we can use this term to determine potential substitution gain energy-wise in algorithm 3. %This term comes from (9), and can be used to evaluate the possible gain from the substitution links $\left[ {0 \to i'@{t_1},i' \to j@{t_2}} \right]$
 
 
 \begin{algorithm}[ht]
 \caption{Suboptimal User Scheduling for Maritime Ship-to-Ship/Shore System}
 \begin{algorithmic}[1]
 \STATE Initialize ${{\mathbf{S}}_{\mathbf{2}}}={{\mathbf{S}}_{\mathbf{1}}}$
 \STATE Initialize ${\mathbf{R}} = \phi $ as group for all plausible ship-to-ship/shore links.
 \STATE Find all ship-to-ship/shore links combination that have higher transmission speed than the original ship-to-shore only links. Store them in ${\mathbf{R}}$.
 \FOR{all user $j$}
  \WHILE{${\left. {{V_{j,t}}} \right|_{t = t_j^{{\rm{Out}}}}} < 2{V^{\rm{QoS}}}$}
   \IF{there is no relay link with $j$ as target in  ${\mathbf{R}}$ }
    \STATE Break.
   \ENDIF
   \STATE Find ship-to-ship/shore links in ${\mathbf{R}}$ with lowest (best) composite power-to-rate ratio $ {\frac{{{P_0}}}{{{r_{0,i',{t_1}}}}} + \frac{{{P_{i'}}}}{{{r_{i',j,{t_2}}}}}} $. 
   \IF{`${V_{i',{t_2}}}$ is ENOUGH' AND `${i'}$ \& $j$ \& SYSTEM are FREE at ${t_2}$' AND `${i'}$ \& SYSTEM are FREE at ${t_1}$'}
    \STATE Add them from ${\mathbf{R}}$ to ${{\mathbf{S}}_{\mathbf{2}}}$.
   \ENDIF
   \ENDWHILE
 \ENDFOR

 \WHILE{${\left. {{V_{j,t}}} \right|_{t = t_j^{{\rm{Out}}}}} \ge {V^{\rm{QoS}}}$}
   \STATE Find original ship-to-shore link in ${{\mathbf{S}}_{\mathbf{2}}}$ with highest (worst) power-to-rate ratio ${\frac{P_0}{r_{0,j,{t_0}}}}$. 
   \STATE Remove it from ${{\mathbf{S}}_{\mathbf{2}}}$ for the substitution.
   
 \ENDWHILE
 \end{algorithmic}
 \end{algorithm}
 
 
 In order to make sure that system constraint in (5d) is met, and relays do not transmit more than they have currently, we have ``${V_{i',{t_2}}}$ is ENOUGH'' in the algorithm, which means 
 \begin{subnumcases}
 {}%\begin{align}
 {V_{j',{t_2} - 1}} \ge r_0^{\min }\Delta \tau ,{\text{if}}\;{t_1} > {t_2},\\
 {V_{j',{t_2} - 1}} + r_{0,j',{t_1}}\Delta \tau \ge r_0^{\min }\Delta \tau ,{\text{else}}.
 %\end{align}
 \end{subnumcases}
 In the above algorithm, ``${i'}$ \& $j$ \& SYSTEM are FREE at ${t_2}$'' means that
 \begin{subnumcases}
 {}%\begin{align}
 \sum\limits_{{i^*} \ne j} { {\delta _{{i^*},j,{t_2}}}} + \sum\limits_{{j^*} \ne j} {{\delta _{j,{j^*},{t_2}}}} \le 1,\\
 \sum\limits_{{i^*} \ne i'} {{\delta _{{i^*},i',{t_2}}}} + \sum\limits_{{j^*} \ne i'} {{\delta _{i',{j^*},{t_2}}}} \le 1 ,\\
 \sum\limits_{{j^*}} {\sum\limits_{{i^*}} {{\delta _{{i^*},{j^*},{t_2}}}} } \le N.
 %\end{align}
 \end{subnumcases}
 And ``${i'}$ \& SYSTEM are FREE at ${t_1}$'' means that
 \begin{subnumcases}
 {}%\begin{align}
 {\sum\limits_{{i^*} \ne i'} {{\delta _{{i^*},i',{t_1}}}} + \sum\limits_{{j^*} \ne i'} {{\delta _{i',{j^*},{t_1}}} \le 1}},\\
 {\sum\limits_{{j^*}} {\sum\limits_{{i^*}} {{\delta _{{i^*},{j^*},{t_1}}} } } \le N}.
 %\end{align}
 \end{subnumcases}
 Through (10) and (11) the half-duplex constraint (5b) and $N$-subcarrier constraint (5c) are met, at least for the ship-to-ship/shore substitution links. 


 \subsection{Step-3}
 
 After step-1, we achieved the optimal solution for problem (6), though we have not take the subcarrier constraint into the ship-to-shore only system. Thus, we might end up with some ship-to-shore links that do not satisfy the subcarrier constraint in (5c). Through step-2, we change part of the original links into ship-to-ship/shore links. This may relieve the conflict after step-1 since the system now has fewer links. In step-3, we deal with the remaining links that do not satisfy the subcarrier constraint.
 %The solution ${{\mathbf{S}}_{\mathbf{1}}}$ returned by step1 is not a practical one for the ship-to-shore system in (5a)-(5c) since (4b) has not been taken into account. We design an effective method to iteratively get the approximate solution ${{\mathbf{S}}_{\mathbf{2}}}$ which is applicable for the ship-to-shore system. 
 
 In step-3, we minimize the energy consumption gap 
 \begin{align}
   &\mathop {\min }\limits_{\left\{ {{\delta _{i,j,t}}} \right\}} \left\{ {\sum\limits_{i = 0}^J {\sum\limits_{j = 1}^J {\sum\limits_{t = 1}^T {\left( {\mathop {{\delta _{i,j,t}}}\limits_{\left( {i,j,t} \right) \in {{\bf{S}}_{\bf{3}}}}  - \mathop {{\delta _{i,j,t}}}\limits_{\left( {i,j,t} \right) \in {{\bf{S}}_{\bf{2}}}} } \right){P_i}\Delta \tau } } } } \right\}.
 \end{align}
 Since the ship-to-shore links returned by step-1 are the optimal choice, any changes in step-3 will result in higher energy consumption. Thus, we progressively approximate the optimal solution for the ship-to-ship/shore system by minimizing the energy consumption gap between ${{\mathbf{S}}_{\mathbf{2}}}$ and the result ${{\mathbf{S}}_{\mathbf{3}}}$. 
 In step-2, we considered the subcarrier constraint when making substitutions. Thus only the original links introduced by step-1 will violate the subcarrier constraint in (5c). Therefore we propose the following algorithm in step-3.
 
 If the constraint (5c) is not satisfied in time slot ${t}$, we find links in ${{\mathbf{S}}_{\mathbf{2}}}$ that have the highest (worst) power-to-rate ratio ${\frac{P_0}{r_{0,j,{t_0}}}}$. We drop those links out of ${{\mathbf{S}}_{\mathbf{2}}}$ and find substitution links to satisfy the QoS need under the $N$-subcarrier constraint in (5b) with minimal energy addition. 
 
 \begin{algorithm}[ht]
 \caption{Subcarrier Constraint Adjustments}
 \begin{algorithmic}[1]
 \STATE Initialize ${{\mathbf{S}}_{\mathbf{3}}}={{\mathbf{S}}_{\mathbf{2}}}$
 \FOR{ all $t$}
  \IF{$\sum\limits_j {{\eta _{0,j,t}}} \le N$ not met}
   \STATE Find original link in ${{\mathbf{S}}_{\mathbf{3}}}$ that have the highest (worst) power-to-rate ratio ${\frac{P_0}{r_{0,j,{t_0}}}}$. 
   \STATE Remove it from ${{\mathbf{S}}_{\mathbf{3}}}$.
   \WHILE {${{V_{j,T}} \ge {V_{j,QoS}}}$ not met}
    \STATE Find ship-to-shore only link or ship-to-ship/shore links (in the reamining ${\mathbf{R}}$) with lowest (best) composite power-to-rate ratio. 
    \STATE Add it/them to ${{\mathbf{S}}_{\mathbf{3}}}$.
   \ENDWHILE
  \ENDIF
 \ENDFOR
 \end{algorithmic}
 \end{algorithm}

 \subsection{Complexity Analysis}

 When carrying out the 3 consecutive offline algorithms, the link transmission speed (capacity) is pre-calculated based on predicted large-scale CSI. Since we have to consider links from $J+1$ transmitters to $J$ receivers in $T$ time slots, it takes $\left( {J + 1} \right) \times J \times T$ calculations. The time complexity here is $O\left[ {\left( {J + 1} \right)JT} \right] = O\left( {{J^2}T} \right)$.
 %Actually, since we only consider the transmissions for the users within the BS coverage, the calculations can be reduced to $O\left[ \left( {J + 1} \right) \times J \times \max \left\{ {t_j^{{\rm{Out}}} - t_j^{{\rm{In}}}} \right\} \right]$ complexity.

 In algorithm 1, we have to go through all $J$ users. For each user $j \in \left\{ {1,...,J} \right\}$, it takes $T$ iterations worst case to get the optimal solution for (6). Finding a ship-to-shore only link with highest transmission speed is an $O\left( \log \left( T \right) \right)$ operation, whereas adding a link to ${{\mathbf{S}}_{\mathbf{1}}}$ is a cheap $O\left( 1 \right)$ operation. The total complexity of algorithm 1 is $O\left( J T\log \left( T \right) \right)$.

 In algorithm 2, first, we have to acquire the `plausible' ship-to-ship/shore link set ${\mathbf{R}}$ (ship-to-ship/shore links in ${\mathbf{R}}$ have higher transmission speed than the optimal ship-to-shore only links). During this process, we store the composite power-to-rate ratio for the links, which is a $O\left( 1 \right)$ operation. 
 Then, we have to go through all $J$ users, adding ship-to-ship/shore links trying to meet the QoS constraint only with ship-to-ship/shore links. In the worst scenario, there are $J \times T^2$ links in ${\mathbf{R}}$ for each user. Hence, finding the ship-to-ship/shore link with lowest (best) composite power-to-rate ratio for user $j$ is an $O\left( \log \left( JT^2 \right) \right)$ operation. 
 After that in ${{\mathbf{S}}_{\mathbf{2}}}$ we have at most $N \times T$ ship-to-ship/shore links (since we have $N$ subcarrier constraint (5b)) and $J \times T$ ship-to-shore only links (inherit from ${{\mathbf{S}}_{\mathbf{1}}}$). 
 At last we go through all $\left( {N + J} \right)T$ links, finding links with highest (worst) composite power-to-rate ratio, and remove them from ${{\mathbf{S}}_{\mathbf{2}}}$. The removal of links has an $O\left[ \left( {N + J} \right)T \log \left( {\left( {N + J} \right)T} \right] \right)$ complexity. 
 Thus, the overall complexity for algorithm 2 is $O\left[ NT \log \left( JT^2 \right) + \left( {N + J} \right)T \log \left( {\left( {N + J} \right)T} \right) \right]$.

 In algorithm 3, since only ship-to-shore links inherit from ${{\mathbf{S}}_{\mathbf{1}}}$ may not satisfy the subcarrier constraint, there are at most $J \times T$ links to remove from ${{\mathbf{S}}_{\mathbf{3}}}$. This has an $O\left( JT\log \left( JT \right) \right)$ worst case complexity. Similar to algorithm 2, adding links so that the QoS constraint is satisfied takes at most $O\left( NT\log \left( JT^2 \right) \right)$ operations. Thus, the time complexity in algorithm 3 is $O\left[ JT \log \left( JT \right) + NT \log \left( {JT^2} \right) \right]$.

 In total, the overall time complexity of all 3 consecutive algorithms is 
 \begin{align}
  &  O\left[  J^2 T  + { N T \log \left( JT^2 \right) + \left( {N + J} \right)T \log \left( {\left( {N + J} \right)T} \right)} \right].
 \end{align}
 As can be seen from the above analysis, the pre-processing of the transmission speed and proposed 3 consecutive offline algorithms all have polynomial complexity. The total complexity is also polynomial. 
 
 \section{Simulation Results}\label{sec:4}
 
 In this section, we provide numerical results for the proposed ship-to-ship/shore 3-step progressive method, as well as a referential round-robin method. 
 %The referential method optimize the ship-to-shore system based on current CSI, therefore the optimization is in a \textbf{1-dimensional} subspace $J$, rather than the $J \times T$ \textbf{2-dimensional} subspace for the ship-to-shore system in subproblem 1 and 2 or the $\left( {J + 1} \right) \times J \times T$ \textbf{3-dimensional} subspace for IoV underlaid system in subproblem 3. 
 For the referential ship-to-shore round-robin method, we have zero information about CSI, thus we find up to $N$ ships in a round-robin method in each time slot. 
 
 \subsection{Parameters Settings}
 
 
 %We compare the large-scale CSI replacement in (2d) with the complete channel in the following simulations. Moreover, we take one step further by taking the expectation operator out from the original channel in (2a) and replace ${h_{i,j,t}} \sim \mathcal{CN}(0, \mathbf{I})$ with ${\left| {{h_0}} \right|^2} = 1$. Based on a low SNR assumption, we get 
 
 %\begin{align}
 %{r_{i,j,t}} \approx {B_s}{\log _2}\left( {1 + \frac{{{P_{i,j,t}}{\beta _{i,j,t}}{{\left| {{h_0}} \right|}^2}}}{{{\sigma ^2}}}} \right).
 %\end{align}
 %We further discuss the performance of this estimation and the large-scale CSI replacement in (2d) in this section.
 
 As for the system settings, the on-shore BS is located in the center of the plane and have a semicircular coverage shape in the sea, while the ships traverse along two intersecting shipping-lanes. 
 % since we focus on passenger ships scenarios in this study. Moreover, passenger ship assumption suits our study since their shipping-lanes are fixed and their position information can be easily determined. 
 Ships (users) leave the harbors every 15 minutes, and all sail at the speed of 36km/h. The time slot duration here is $\Delta \tau = 60$s. The QoS constraint is 1Gbits/ship if not specified. We assume that the system uses a carrier frequency of 1.9GHz, and has 32 subcarriers, each having a bandwidth of 2MHz. The on-shore BS's transmit power is 10W on any subcarrier, whereas the vessels' relay transmit power is 1W on any subcarrier since they are arguably smaller in size. The antenna height of the BS and the ships are 100m and 10m, respectively. The power density of the additive white Gaussian noise is ${-140}$dBm/Hz. 
 
 Of all the following simulations in Fig. 2 - 4, %the `round-robin' method refers to the ship-to-shore only method based on zero CSI. 
 %The legend `large-scale CSI' means the algorithms carried out based on large-scale CSI that can be practically obtained through shipping-lanes and timetables. 
 the legend `(genius-aided) full CSI' actually means the assumption that we can know full CSI for the whole service duration in advance. %, which is impossible
 They are brought into the following simulations to examine the feasibility of our large-scale CSI method. 
 
 
 As shown later in simulations, having full CSI indeed will be most feasible. In ship-to-ship/shore systems, the difference in system energy consumption between large-scale CSI replacement and the genius-aided `full CSI' is around 5\% - 15\%. These errors here come from assuming that ship $j$ stays in the same position and $\beta _{i,j,\tau }$ remains constant during each time slot $t$, as well as not knowing the full CSI.
 This 5\% - 15\% error in ship-to-ship/shore systems shows that the large-scale CSI replacement in (2c) is quite acceptable. %If we are less strict with the approximation, the ${h_0}$ constant estimation may be acceptable since the estimation in (12) shares similarity in trend and shape with the actual channel. 
  
 
 \subsection{Performance Evaluations}
 
 First, in Fig. 2, we study the impact of only considering a part of the service duration in user scheduling. 
 Next, we investigate the impact of different service needs in Fig. 3. 
 Last, in Fig. 4, we investigate the impact of different transmit powers, since our algorithms mainly focus on user scheduling with fixed transmit power on given subcarrier. 
 
 
 \begin{figure} [htb]
 \begin{center}
 \includegraphics*[width=8.8cm]{Tranges.eps}
 \end{center}
 \vspace*{-4mm} 
 \caption{System performance (energy consumption) when only considering a part of the system service duration in user scheduling algorithms} \label{fig:2}
 \vspace*{-2mm} 
 \end{figure}
 
 %\Fig.[t!](topskip=0pt, botskip=0pt, midskip=0pt){Tranges.png}
 %{Average energy consumption per user $E_{avg}$ versus the percentage of pre-acquired CSI.\label{fig4}}
 
 Fig. 2 demonstrates the relationship between average energy consumption and the ratio of ${T_{\rm{acq}}}$ to total service time duration. Here ${T_{\rm{acq}}}$ represents the time duration in user scheduling, i.e., we only consider at most $T{\rm{acq}}$ time slots in user scheduling. %The QoS constraint here is 1Gbits/user. 
 %The larger ${T_{\rm{acq}}}$ is, the longer can we predict the CSI, and hence the more feasible transmission time slots we can choose from in our method. As a result, we can get more improvement from our process-oriented D2D-aided or ship-to-shore method when ${T_{\rm{acq}}}$ approximate the total service time duration. 
 
 As we can see, the genius-aided full CSI curves are better than our large-scale CSI replacement energy-wise generally. Our proposed ship-to-ship/shore method outmatches the referential method, especially in ideal conditions (which means we consider full service duration, ${T_{\rm{acq}}}=T$). 
 Since we can take the cost of larger algorithm complexity (resulting from larger $T_{\rm{acq}}$), we can increase  $T_{\rm{acq}}$ in exchange for better performance energy-wise. The long-term (larger $T_{\rm{acq}}$) large-scale CSI result we get is better than the shorter-term (smaller $T_{\rm{acq}}$) genius-aided full CSI result, which justifies our large-scale replacement. 
 
 When ${T_{\rm{acq}}}$ approximates the total service time duration, we have maximum benefit from user scheduling. The large-scale CSI ship-to-ship/shore method consummates 60\% less energy than the round-robin method. 
 Under simulation settings, due to geographical closeness and better CSI, users tend to finish their communications with BS or relays during the first 70\% of their staying time inside the service zone (semicircle). The 60\% improvement is steady until the ratio of ${T_{\rm{acq}}}$ to total service time becomes less than 0.7. Thus we can set ${T_{\rm{acq}}=0.7T}$ for relatively large improvements at a relatively small complexity cost. 
 Energy consumption rises as the ratio of ${T_{\rm{acq}}}$ to total service time shrinks, mainly because we aim to satisfy as much as possible in fewer time slots. When ${T_{\rm{acq}}} = 0$, i.e. having zero CSI, our proposed method retrogresses to the referential method. 
 %When the QoS constraint is 1Gbits/user, 
 %These improvements prove the superiority in our 3-dimensional optimization over 2-dimensional and 1-dimensional ones. 
 %
 
 
 
 \begin{figure} [htb]
 \begin{center}
 \includegraphics*[width=8.8cm]{Cqos.eps}
 \end{center}
 \vspace*{-4mm} 
 \caption{System performance (energy consumption) under different QoS constraint (data volume constraint).}\label{fig:3}
 \vspace*{-2mm} 
 \end{figure}
 
 
 %\Fig.[htb]{Cqos.png}
 %{Average energy consumption per user $E_{avg}$ versus the QoS constraint ${V_{\rm{QoS}}}$.\label{fig3}}
 
 Fig. 3 shows the bit-wise average energy consumption under different QoS constraint (desired data volume for each user).
 Our proposed ship-to-ship/shore method outmatches the referential method. We achieve the best performance improvement when the QoS constraint is 10Mbits/user. The ship-to-ship/shore method consummates 71\% less energy than the referential method. When the QoS constraint is 1.5Gbits/user, we have the worst performance improvement, which is 46\% over the referential method energy-wise.
 
 The referential method's bit-wise energy consumption changes irregularly since it has zero information about CSI and chooses transmission links in a round-robin way. The proposed method's energy consumptions, on the other hand, increase as the data volume desired becomes larger. The rise in proposed method's energy consumption is because we end up choosing the time slots with low transmission speed in order to meet the increasing QoS constraint. This results in a larger energy consumption per user per Gbit. 
 
 
 
 \begin{figure} [htb]
 \begin{center}
 \includegraphics*[width=8.8cm]{snrs.eps}
 \end{center}
 \vspace*{-4mm} 
 \caption{System performance (energy consumption) under different BS's transmit power on given subcarrier.}\label{fig:5}
 \vspace*{-2mm} 
 \end{figure}

 Fig. 4 shows average energy consumption versus BS's transmit power under given QoS constraint. The transmit power ratio of BS and ship relays remains 10 during the change, i.e. the transmit powers of ship relays also change in direct proportion to the BS's transmit power. 
 As we can see, the energy consumption first decreases then increases. 
 The energy consumption increases first since the rise in transmit power results in the rise of transmission speed, and therefore the data distribution service can be done more quickly and in better CSI conditions. 
 The overall energy consumption then increases since the $log\left({}\right)$ operator in channel capacity (transmission speed) makes the increase in SNR less feasible when SNR is relatively large. Moreover, due to the greedy progressive methods for solving the user scheduling problem, the greedy choice we made focusing on links with best transmission speed also results in the increase energy-wise. 
% The inflection point is larger in ship-to-ship/shore system since the transmission speed gain from lower ship relay transmit power is larger than higher BS transmit power, thus similarly the relay transmission can be done more quickly and in better CSI conditions. 
 
 By going through different transmit powers, we can find the optimal transmit power in any specific scenario. In this case, the optimal transmit power is around 8W for the BS (0.8W for the ship relays), smaller than the 10W BS assumption we first made. The optimal transmit power is also slightly higher in large-scale CSI based user scheduling schemes than genius-aided perfect CSI scenario. 
 
 %The gap between large-scale CSI approximation (2d) and the actual channel remains minimal, whereas the gap between the ${h_0}$ constant estimation and the actual channel shrinks as the noise rises. %The Gaussian noise ${\sigma ^2 ={10^{ - 14}}}$ is more often the case in following simulations. For further tests, we increase the noise ${\sigma ^2}$. 
 %The worsening in SNR also result in the increase in energy consumption, while making our proposed methods more beneficial as the gap in energy consumption between referential method and proposed ship-to-ship/shore method increases. %The relationship between SNR and the gap are explored below. 
 
 %We estimate ${r_{i,j,t}}$ in (2d) and in (12) both through taking out the expectation operator. Since $x \ge {\log _2}\left( {1 + x} \right)$ when ${x \ge 0}$, the transmission speed or channel capacity ${r_{i,j,t}}$ is smaller in estimation (12) than the actual channel or large-scale CSI replacement in (2d). Therefore, the energy consumptions with the ${h_0}$ constant estimation are smaller than the others, just as showed in Fig. 5. 
 %Since the ${h_0}$ constant approximation is based on a low SNR assumption, as the noise function ${\sigma ^2}$ increases, the ${h_0}$ constant estimations approximate the large-scale CSI replacement in (2d) and the complete channel energy-wise.
 
 %As the SNR decreases, the channel capacity worsens and the greedy referential method ends up with more low-speed links. Thus we can see an improvement of over 60\% in energy consumption from the proposed ship-to-ship/shore method when ${\sigma ^2 \le{10^{ - 10}}}$.
 
 To sum up, the followings can be seen from the simulations. 
 First, considering a longer time in user scheduling surely will improve system performance, but the sweetpoint is 70\% of service duration. 
 Second, a larger QoS data volume constraint will result in larger bit-wise energy consumption. 
 Last, we can go through different powers to choose the optimal transmit power for given QoS constraint. As for performance, the total energy improvement is up to 71\%, while the worst case improvement is 46\%. 

 The approximate solution we deduced from step 1 - 3 with polynomial time complexity show its potential as well as its simplicity. Therefore, the 3-step progressive approach in Section III is justified. 
 
 \section{Conclusion}\label{sec:5}
 
 In this paper, we focused on user scheduling in a maritime ship-to-ship/shore communication system, aiming to reduce the system energy consumption. 
 
 In order to reduce the system energy consumption, we introduced ship-to-ship relay communication. 
 %By exploiting large-scale CSI for the whole service duration, we proposed our user scheduling algorithms. 
 Since large-scale CSI is the dominant factor in maritime scenarios, and the large-scale CSI can be easily predicted based on shipping lanes and timetables, we used large-scale CSI instead of full CSI to avoid the heavy overhead. Further, we progressively approximated the user scheduling problem through a 3-step offline approach, which has a polynomial time complexity. 
 
 As for the performance of proposed 3-step offline approach, it's shown by simulations that the proposed user scheduling scheme significantly enhance the system performance by up to 71\% in terms of energy consumption. %Our large-scale CSI replacement is also justified by the improvements under polynomial complexity. 
 Moreover, considering a longer time in offline user scheduling surely will improve system performance, but the sweetpoint is 70\% of service duration. 
 In addition, a larger QoS data volume constraint will result in larger bit-wise energy consumption. 

 The main limitation of our progressive offline approach is that we assumed the ships to stick to their shipping lanes and timetables. If there are some deviations in the ship routes, our large-scale CSI prediction will have larger errors from the genius aided perfect CSI, and further leads to worsening in proposed user scheduling scheme's performance. 
 
 It should be noted that transmit power allocation and user scheduling jointly  influence the system energy consumption. In this paper, we assumed fixed transmit power on any given subcarrier, and repeated the user scheduling algorithms under different power constraints to get the optimal transmit power. In the near furture, we plan to design a joint power allocation and user scheduling algorithm for the maritime ship-to-ship/shore communication system. 

 %\appendices
 
 %\section*{Appendix A \\ Proof of Theorem 1}
 %The problem (13) is a Nonlinear Integer Programming problem, which has been proved to be NP-hard \cite{p420}. 
 %Nonlinear Integer Programming ∗
 %Therefore, the equivalent problem (4) is NP-hard.
 
 
 \section*{Acknowledgement}
 
 This work was partially supported by the National Science Foundation of China under Grant No. 61771286 and Grant No. 91638205 and Grant No. 61621091. The authors Yunzhong Hou and Te Wei contributed equally to this work.
 
 
 \begin{thebibliography}{10}
  
  %maritime
  %\bibitem{p322}
  %T. Roste, K. Yang, and F. Bekkadal, ``Coastal coverage for maritime broadband communications,'' in
  %\emph{Proc. MTS/IEEE--Bergen}, Jun. 2013, pp.~1--8.
  
  %\bibitem{p101}
  %D. Liu, Y. Yu, C. Wen, and Z. Zhang, ``The GDUT Maritime Silk Road project (2014--2015) as a case study for VSMM in museum settings in China,'' in
  %\emph{Proc. Intern. Conf. Virtual System \& Multimedia}, Oct. 2016, pp.~1--9.
  
  \bibitem{p321}
  R. Campos, T. Oliveira, N. Cruz, A. Matos, and J. M. Almeida,
  ``BLUECOM+: Cost-effective broadband communications in remote ocean areas,'' in
  \emph{Proc. OCEANS}, Apr. 2016, pp.~1--6.
  
  \bibitem{p323}
  F. Bekkadal, ``Innovative maritime communications technologies,'' in
  \emph{Proc. Intern. Conf. Microwaves, Rader, \& Wireless Commun.}, June 2010, pp.~1--6.
  
  \bibitem{p32}
  M. Zhou, et al., ``TRITON: high-speed maritime wireless mesh network,''
  \emph{IEEE Wireless Commun.}, vol. 20, no. 5, pp.~134--142, 2013.
  
  
  \bibitem{p33}
  S. Buzzi, et al., ``A survey of energy-efficient techniques for 5G networks and challenges ahead,''
  \emph{IEEE J. Sel. Areas Commun.}, vol. 34, no. 4, pp.~697--709, 2016.
  
  \bibitem{p51}
  M. Shreedhar and G. Varghese, ``Efficient fair queuing using deficit round-robin,''
  \emph{IEEE/ACM Trans. Networking}, vol. 4, no. 3, pp.~375--385, 1996.
  
  \bibitem{p52}
  Q. Cao, Y. Sun, Q. Ni, S. Li, and Z. Tan, ``Statistical CSIT aided user scheduling for broadcast MU-MISO system,''
  \emph{IEEE Trans. Veh. Tech.}, vol. 66, no. 7, pp.~6102--6114, 2017.
  
  \bibitem{p53}
  J. Wang, M. Matthaiou, S. Jin, and X. Gao, ``Precoder design for multiuser MISO systems exploiting statistical and outdated CSIT,''
  \emph{IEEE Trans. Commun.}, vol. 61, no. 11, pp.~4551--4564, 2013.
  
  %\bibitem{p3}
  %X. Li, et al., ``Energy efficiency optimization: joint antenna-subcarrier-power allocation in OFDM-DASs,''
  %\emph{IEEE Trans. Wireless Commun.}, vol. 15, no. 11, pp.~7470--7483, 2016.
  
  %\bibitem{p6}
  %B. Di, L. Song, and Y. Li, ``Sub-channel assignment, power allocation, and user scheduling for non-orthogonal multiple access networks,''
  %\emph{IEEE Trans. Wireless Commun.}, vol. 15, no. 11, pp.~7686--7698, 2016.
  
  \bibitem{p4}
  L. Shan and R. Miura, ``Energy-efficient scheduling under hard delay constraints for multi-user MIMO System,'' in
  \emph{Proc. Intern. Symp. Wireless Personal Multimedia Commun.}, Sept. 2014, pp.~696--699.
  
  \bibitem{p5}
  S. Cao, Q. Cui, Y. Shi, H. Wang, and X. Ma, ``Cross-layer cooperative delay-energy tradeoff scheme for hybrid services in cellular networks,'' in
  \emph{Proc. IEEE Veh. Tech. Conf.}, May 2014, pp.~1--5.
  
  \bibitem{p7}
  X. Xiong, B. Jiang, X. Gao and X. You, ``QoS-guaranteed user scheduling and pilot assignment for large-scale MIMO-OFDM systems,''
  \emph{IEEE Trans. Veh. Tech.}, vol. 65, no. 8, pp.~6275--6289, 2016.
  
  %\bibitem{p8}
  %Rahul Singh, Alexander Stolyar, ``MaxWeight scheduling: Smoothness of the service process'',
  %\emph{IEEE 35th Annual IEEE International Conference on Computer Communications (INFOCOM)}, pp.~1-9, 2016.
  
 
  % 
  \bibitem{p300}
  T. Yang, H. Liang, N. Cheng, and X. Shen, 
  ``Towards video packets store-carry-and-forward scheduling in maritime wideband communication,'' in
  \emph{IEEE Global Commun. Conf. (GLOBECOM)}, Dec. 2013, pp.~4032--4037.
 
  \bibitem{p301}
  T. Yang, H. Liang, N. Cheng, R. Deng, and X. Shen, ``Efficient scheduling for video transmissions in maritime wireless communication networks,'' 
  \emph{IEEE Trans. Veh. Tech.}, vol. 64, no. 9, pp.~4215--4229, 2015.
 
  %\bibitem{p302}
  %A. Bejan, R. Gibbens, Y. Lim, and D. Towsley, 
  %``A performance analysis study of multipath routing in a hybrid network with mobile users,'' in 
  %\emph{Proc. of the 2013 25th International Teletraffic Congress (ITC)}, Sept. 2013, pp.~1--9.
 
  \bibitem{p303}
  Y. Bai, Y. Zhai, and D. Wang, 
  ``Research on optimum cooperative relay model for moving targets based on ant colony algorithm,'' in
  \emph{4th International Conference on Information, Cybernetics and Computational Social Systems (ICCSS)}, Nov. 2017, pp.~539--543.
 
 
  %\bibitem{p400}
  %T. Wei, W. Feng, N. Ge, and J. Lu, ``Optimized time-shifted pilots for maritime massive MIMO communication systems,'' in
  %\emph{Proc. 26th Wireless \& Optical Commun. Conf.}, Apr. 2017, pp. 1--5.
 
 
  %\bibitem{p402}
  %M. T. Zhou, H. Harada, P. Y. Kong, and J. S. Pathmasuntharama, ``Interference range analysis and scheduling among three-hop neighborhood in maritime WiMAX mesh networks,'' in 
  %\emph{Proc. IEEE Wireless Commun. \& Networking Conf.}, Apr. 2010, pp. 1--6.
 
  \bibitem{p403}
  W. Feng, Y. Wang, D. Lin, N. Ge, J. Lu, and S. Li, ``When mmWave communications meet network densification: a scalable interference coordination perspective,''
  \emph{IEEE J. Sel. Areas Commun. }, vol. 35, no. 7, pp. 1459--1471, 2017.
 
  \bibitem{p404}
  W. Feng, Y. Chen, N. Ge, and J. Lu, ``Optimal energy-efficient power allocation for distributed antenna systems with imperfect CSI,''
  \emph{IEEE Trans. Veh. Tech.}, vol. 65, no. 9, pp. 7759-7763, 2016.
 
  \bibitem{p405}
  W. Feng, Y. Wang, N. Ge, J. Lu, and J. Zhang, ``Virtual MIMO in Multi-Cell Distributed Antenna Systems: Coordinated Transmissions with Large-Scale CSIT,'' 
  \emph{IEEE J. Sel. Areas in Commun.}, vol. 31, no. 10, pp. 2067-2081, 2013.
 
 
  \bibitem{p410}
  J. S. Pathmasuntharam, J. Jurianto, P. Y. Kong, Y. Ge, M. Zhou, and R. Miura, ``High speed maritime ship-to-ship/shore mesh networks,'' in
  \emph{Proc. 7th Intern. Conf. ITS Telecommun.}, June 2007, pp. 1--6.
 
  %\bibitem{p411}
  %R. L. Moe, "Networking and ship-to-shore ship-to-ship communication," 
  %\emph{OCEANS '88. A Partnership of Marine Interests}, pp. 532-536 vol.2 , 1988.
 
  %\bibitem{p420}
  %R. Hemmecke, M. Köppe, J. Lee, and R. Weismantel, ``Nonlinear Integer Programming,'' in
  %\emph{Jünger M. et al. (eds) 50 Years of Integer Programming 1958-2008}, Springer, Berlin, Heidelberg, 2010.
 
 
  
 %\bibitem{p120}
 %M. Jung, K. Hwang, and S. Choi, ``Joint mode selection and power allocation scheme for power-efficient device-to-device (D2D) communication,'' in
 %\emph{Proc. IEEE Veh. Tech. Conf.}, May 2012, pp.~1--5.
 
 %\bibitem{p230}
 %H. Shin and J. H. Lee, ``Capacity of multiple-antenna fading channels: spatial fading correlation, double scattering, and keyhole'',
 %\emph{IEEE Trans. Info. Theory}, vol. 49, no. 10, pp.~2636--2647, 2003.
 
 \bibitem{p0}
 S. Balkees P A, K. Sasidhar, and S. Rao, ``A survey based analysis of propagation models over the sea,'' in
 \emph{Proc. Intern. Conf. Advances in Computing, Commun. \& Informatics}, Aug. 2015, pp.~69--75.
 
 \bibitem{p1}
 Y. Zhao, J. Ren, and X. Chi, ``Maritime mobile channel transmission model based on ITM,''
 \emph{Proc. Intern. Symp. Computer Commun. Control \& Automation}, vol. 68, no. 3, pp.~378--383, 2013.
 
 \bibitem{p2}
 J. C. Reyes-Guerrero, M. Bruno, L. A. Mariscal, and A. Medouri, ``Buoy-to-ship experimental measurements over sea at 5.8 GHz near urban environments,'' in
 \emph{Proc. Mediterranean Microwave Symp.}, Sept. 2011, pp.~320--324.
 
 
  %\bibitem{p22}
  %C. He, G. Y. Li, F. Zheng, X. You, ``Power Allocation Criteria for Distributed Antenna Systems'',
  %\emph{IEEE Transactions on Vehicular Technology (TVT)}, vol. 64, no. 11, pp.~5083-5090, 2015.
 
  \bibitem{p41}
  H. Shin and J. H. Lee, ``Capacity of multiple-antenna fading channels: spatial fading correlation, double scattering, and keyhole'',
  \emph{IEEE Trans. Info. Theory}, vol. 49, no. 10, pp.~2636--2647, 2003.
 
 % \bibitem{p9}
 % F. Fernandes, A. Ashikhmin, T. L. Marzetta, ``Inter-Cell Interference in Noncooperative TDD Large Scale Antenna Systems'',
 % \emph{IEEE Journal on Selected Areas in Communications (JSAC)}, vol. 31, no. 2, pp.~192-201, 2013.
 
  % \bibitem{p11}
  % N. Souto, R. Dinis, ``MIMO Detection and Equalization for Single-Carrier Systems Using the Alternating Direction Method of Multipliers'',
  % \emph{IEEE Signal Processing Letters (SPL)}, vol. 23, no. 12, pp.~1751-1755, 2016.
 
  % \bibitem{p123}
  % H. Zhang, C. Jiang, N. C. Beaulieu, X. Chu, X. Wen, M. Tao, ``Resource Allocation in Spectrum-Sharing OFDMA Femtocells With Heterogeneous Services'',
  % \emph{IEEE Transactions on Communications}, vol. 62, no. 7, pp.~2366-2377, 2014.
 
  % \bibitem{p8}
  % T. Yang, H. Liang, N. Cheng, R. Deng, X. Shen, ``Efficient Scheduling for Video Transmissions in Maritime Wireless Communication Networks'',
  % \emph{IEEE Transactions on Vehicular Technology (TVT)}, vol. 64, no. 9, pp.~4215-4229, 2015.
 
 
 \end{thebibliography}
 
 
 \end{document}
 